\documentclass[a4paper]{article}

\usepackage[ngerman]{babel}
\usepackage[utf8]{inputenc}
\usepackage{amsthm}
\usepackage{amsmath}
\usepackage{amssymb}
\usepackage{tikz,tkz-euclide}
\usepackage{titlesec}
\usepackage{gensymb}
\usepackage{textcomp}
\usepackage[titles]{tocloft}
\usepackage{csquotes}
\usepackage[babel]{microtype}
\usepackage{MnSymbol}
\usepackage{stmaryrd}
\usepackage{mathtools}
\usepackage{ulem}
\usepackage[shortlabels]{enumitem}
\usepackage{scalerel}
\usepackage{stackengine}
\usepackage[
  separate-uncertainty = true,
  multi-part-units = repeat
]{siunitx}

\usetkzobj{all}
\usetikzlibrary{shapes.misc}

\MakeOuterQuote{"}

\setcounter{secnumdepth}{4}

\renewcommand\hateq{\mathrel{\stackon[1.5pt]{=}{\stretchto{%
				\scalerel*[\widthof{=}]{\wedge}{\rule{1ex}{3ex}}}{0.5ex}}}}

\newcommand*\circled[1]{
  \tikz[baseline=(C.base)]\node[draw,circle,inner sep=0.75pt](C) {#1};\!
}

\newcommand*{\obot}{\perp\mkern-20.7mu\bigcirc}

\DeclarePairedDelimiter\abs{\lvert}{\rvert}
\DeclarePairedDelimiter\norm{\lVert}{\rVert}
\makeatletter
\let\oldabs\abs
\def\abs{\@ifstar{\oldabs}{\oldabs*}}
\let\oldnorm\norm
\def\norm{\@ifstar{\oldnorm}{\oldnorm*}}
\makeatother

\newcommand{\ul}{\underline}
\renewcommand{\proof}{\ul{Beweis:}\\}
\renewcommand{\qed}{\begin{flushright}
\ul{\(q.e.d.\)}
\end{flushright}}
\let\origphi\phi
\let\phi\varphi
\let\origepsilon\epsilon
\let\epsilon\varepsilon

\title{Analysis 2}
\author{Nico Mexis}
\date{\today}

\begin{document}
\maketitle
\newpage

\tableofcontents
\newpage

\setcounter{section}{-1}
\section{Wiederholung}
\ul{Satz von Taylor:}\\
Sei $f\in C^{n+1}(I)$, $x,x_0\in I$. Dann gilt:
$$f(x)=\sum_{k=0}^n\frac{f^{(k)}(x_0)}{k!}(x-x_0)^k+\frac{1}{n!}\int_{x_0}^{x}(x-t)^nf^{(n+1)}(t)\text{d}t$$
\section{Metrik \& Topologie}
Sei $X$ eine Menge.\\
\ul{Metrik}: $d:X\times X\rightarrow\mathbb{R}_+$ Metrik, wenn:
\begin{enumerate}[1)]
	\item $d(x,y)=0\Leftrightarrow x=y$
	\item $d(x,y)=d(y,x)$
	\item $d(x,y)\leq d(x,z)+d(z,y)$
\end{enumerate}
\ul{Norm:} $\norm{\cdot}:X\rightarrow\mathbb{R}$ Norm, wenn:
\begin{enumerate}[1)]
	\item $\norm{x}\geq 0$ und $\norm{x}=0\Leftrightarrow x=0$
	\item $\norm{cx}=\abs{x}\cdot\norm{x},\ c\in\mathbb{R}$
	\item $\norm{x+y}\leq\norm{x}+\norm{y},\ x,y\in X$
\end{enumerate}
$\norm{\cdot}$ Norm $\Rightarrow$ $d(x,y)=\norm{x-y}$ induzierte Metrik
\begin{tabbing}
	\ul{Kugel}: \= $x\in X,\ r\in\mathbb{R}_+$\\
	\> abgeschlossene Kugel: \=$B_{\leq r}(x):=\{y\in X:d(x,y)\leq r\}$ (mit Rand)\\
	\> offene Kugel: \>$B_{<r}(x):=\{y\in X:d(x,y)<r\}$ (ohne Rand)
\end{tabbing}
\ul{Mengen}:
\begin{enumerate}[1)]
	\item $Y\subseteq X$ \ul{offen}, wenn $\forall y\in Y:\exists \epsilon>0:B_{<\epsilon}(y)\subseteq Y$
	\item $Y$ \ul{abgeschlossen}, wenn $X\backslash Y$ offen
\end{enumerate}
Offene Kugel = offene Menge\\
Abgeschlossene Kugel = abgeschlossene Menge\\
$X$ metrischer Raum $\Rightarrow$
\begin{enumerate}[1)]
	\item $X$ und $\varnothing$ offen
	\item Jede \ul{abzählbare} Vereinigung von offenen Mengen ist offen
	\item Jeder \ul{endliche} Durchschnitt ist offen
	\item $X$ und $\varnothing$ abgeschlossen
	\item $\bigcup_{<\infty}\text{abg}=\text{abg}$
	\item $\bigcap_{\infty}\text{abg}=\text{abg}$
\end{enumerate}
\ul{Umgebung}: $U\subseteq X$ von $x$: $\exists\epsilon>0:B_{<\epsilon}(x)\subseteq U$\\
\ul{Abschluss}:
\begin{enumerate}[1)]
	\item $x\in X$ \ul{Berührpunkt}/\ul{Kontaktpunkt} zu $Y\subseteq X$, wenn $\forall \text{Umgebung }U\text{ von }x:U\cap Y\neq\varnothing$
	\item \ul{Abschluss} von $Y$: $\overline{Y}=\{\text{Kontaktpunkte an }Y\}$
\end{enumerate}
$Y,Y'\subseteq X$ $\Rightarrow$
\begin{enumerate}[1)]
	\item $Y\subseteq\overline{Y},\ \overline{\overline{Y}}=\overline{Y}$
	\item $\overline{Y\cup Y'}=\overline{Y}\cup\overline{Y'}$
	\item $\overline{Y}$ abgeschlossen
\end{enumerate}
\ul{Randpunkt:} $x\in X$ Randpunkt zu $Y\subseteq X$, wenn $\forall \text{Umgebung }U\text{ von }x:U\cap Y\neq\varnothing\neq U\cap(X\backslash Y)$\\
\ul{Rand}: $\partial Y=\{\text{Randpunkte von }Y\}$\\
$Y\subseteq X$ $\Rightarrow$ Rand $\partial Y$ erfüllt
\begin{enumerate}[1)]
	\item $Y\backslash\partial Y$ offen
	\item $Y\cup\partial Y$ abgeschlossen
	\item $\partial Y$ abgeschlossen
\end{enumerate}
\ul{Topologie}: $\mathcal{T}$ System von Teilmengen von $X$ ($\mathcal{T}\subseteq 2^X=\{0,1\}^X=\mathcal{P}(X)$)\\
Topologie, wenn
\begin{enumerate}[1)]
	\item $X,\varnothing\in\mathcal{T}$
	\item $T,T'\in\mathcal{T}\Rightarrow T\cap T'\in\mathcal{T}$
	\item $T_z\in\mathcal{T},\ z\in I\Rightarrow\bigcup_{i\in I}T_i\in\mathcal{T}$
\end{enumerate}
$X$ topologischer Raum, $\mathcal{T}$ auf $X$\\
$T\in\mathcal{T}$ "offene Menge"\\
$U\subseteq X$ Umgebung von $x$, wenn $x\in T\subseteq U$\\
$Y\subseteq X$ "offen" $\Leftrightarrow$ $\forall y\in Y:\exists\text{Umgebung }U\text{ von }Y:U\subseteq Y$\\
$Y$ abgeschlossen $\Leftrightarrow$ $X\backslash Y$ offen\\
\begin{tabbing}
	\ul{Hausdorffraum}: \=topologischer Raum\\
	\>$x\neq x'\Rightarrow U,U'$ mit $x\in U,x'\in U'$ und $U\cap U'=\varnothing$ "punkttrennend"
\end{tabbing}
Metrischer Raum $\Rightarrow$ Hausdorffraum\\
$X$ topologischer Raum und $Y\subseteq X$ $\Rightarrow$ $\overline{Y}$ abgeschlossen\\
\ul{Diskret}: $\mathcal{T}=\{T\subseteq X\}$\\
\ul{Indiskret}: $\mathcal{T}=\{\varnothing,X\}$\\
$y\in Y\subseteq X$ \ul{innerer Punkt}, wenn es Umgebung $U$ von $y$ gibt mit $U\subseteq Y$\\
\ul{Inneres} von $Y$: $Y^\circ=\{\text{innere Punkte von }Y\}$\\
Sei $Y\subseteq X$: $\partial Y=\overline{Y}\backslash Y^\circ$\\
Die Folge $a=(a_n:n\in\mathbb{N})$ heißt \ul{konvergent} mit Grenzwert $a^*\in X$, wenn es zu jeder Umgebung $U$ von $a^*$ ein $n_0$ gibt, mit $a_n\in U$, $u\geq u_0$.\\
$a^*$ Häufungspunkt: $\#\{n:a_n\in U\}=\infty$ für alle $U(a^*)$\\
$Y\subseteq X$ $\Rightarrow$
\begin{enumerate}[1)]
	\item $Y$ abgeschlossen, $a$ konvergente Folge in $Y$ $\Rightarrow$ $a^*\in Y$
	\item $a$ konvergent $\Rightarrow$ $a^*\in \overline{Y}$
	\item $X$ metrisch: $Y$ abgeschlossen $\Leftrightarrow$ $a^*\in Y$
\end{enumerate}
Hausdorffraum $\Rightarrow$ Grenzwert eindeutig\\
$f:X\rightarrow X'$ heißt "\ul{stetig}", wenn für alle $x_n\rightarrow x^*$ auch $f(x_n)\rightarrow f(x^*)$ gilt.\\
$f:X\rightarrow X'$, $Y'\subseteq X'$: $f^{-1}(X'):=\{x\in X\ \vert\ f(x)\in Y'\}$ \ul{Urbild}\\
$f:X\rightarrow X'$ stetig $\Leftrightarrow$ Urbilder abgeschlossener Mengen sind abgeschlossen\\
$f:X\rightarrow X'$ stetig $\Leftrightarrow$ Urbilder offener Mengen sind offen\\
$a=(a_n:n\in\mathbb{N})$ Cauchy-Folge, wenn es zu $\epsilon>0$ ein $n_0$ gibt, sodass $d(x_n,x_m)<\epsilon$, $n,m\geq n_0$\\
$X$ vollständig, wenn Cauchy $\Rightarrow$ Konvergenz\\
Vervollständigung: $\{\text{Cauchy-Folgen}\}$\\
$a\sim a'\Leftrightarrow \lim_{n\rightarrow 0}d(a_n,a_n')=0$\\
$Y\subset X$ \ul{dicht}, wenn $\overline{Y}=X$\\
$X$ ist vollständig $\Leftrightarrow$ jede Folge $B_{\leq \epsilon_1}(x_1)\supset B_{\leq \epsilon_2}(x_2)$ mit $\epsilon_1>\epsilon_2>\dots$ und $\epsilon_n\rightarrow 0$ erfüllt:
$$\bigcap_{j=1}^\infty B_{\leq \epsilon_j}(x_j)\neq\varnothing$$\\
\ul{Kontraktion}: $f:X\rightarrow X$, $d(f(x),f(x'))\leq \rho\cdot d(x,x')$, $0<\rho<1$\\
$f:X\rightarrow X$ Kontraktion mit $X$ vollständig $\Rightarrow$ $f$ hat \ul{genau einen} Fixpunkt $f(x^*)=x^*\in X$\\\\
$Y\subseteq X$
\begin{enumerate}[1)]
	\item \ul{Überdeckungskompakt}: Für jede Überdeckung $Y\subseteq\bigcup_{U\in \mathcal{U}}U$ von offenen Umgebungen gibt es $U_1,\dots,U_n$ mit $Y\subseteq U_1\cup\dots\cup U_n$
	\item \ul{Folgenkompakt}: Jede Folge in $Y$ enthält konvergente Teilfolge
\end{enumerate}
$Y\subseteq X$ metrisch ist überdeckungskompakt $\Leftrightarrow$ $Y$ folgenkompakt\\
\begin{enumerate}[1)]
	\item $A\subseteq Y$ heißt ein $\epsilon$-Netz für $Y$, wenn $Y\ni y\in B_{<\epsilon}(a)$, $a\in A$
	\item $Y$ \ul{total beschränkt}: $\exists \text{endliches }\epsilon\text{-Netz }A\text{ für jedes }\epsilon>0$
	\item $Y$ \ul{beschränkt}: $d(y,y')<M$, $y,y'\in Y$\\
	\ul{Durchmesser}: $d(Y):=\sup_{y,y'}d(y,y')$\\
	\ul{Radius}: $r(Y):=\inf_y\sup_{y'}d(y,y')$
\end{enumerate}
Jede total beschränkte Menge ist beschränkt.\\
$Y\subset X$, $Y$ abgeschlossen, $X$ vollständig, dann: $Y$ folgenkompakt $\Leftrightarrow$ $Y$ total beschränkt\\
$Y\subset X$ metrisch, $Y$ überdeckungskompakt $\Rightarrow$ $Y$ abgeschlossen, beschränkt
\end{document}






























