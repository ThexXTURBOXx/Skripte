\documentclass[a4paper]{article}

\usepackage[ngerman]{babel}
\usepackage[utf8]{inputenc}
\usepackage{amsthm}
\usepackage{amsmath}
\usepackage{amssymb}
\usepackage{tikz,tkz-euclide}
\usepackage{titlesec}
\usepackage{gensymb}
\usepackage{textcomp}
\usepackage[titles]{tocloft}
\usepackage{csquotes}
\usepackage[babel]{microtype}
\usepackage{MnSymbol}
\usepackage{stmaryrd}
\usepackage{mathtools}
\usepackage{ulem}
\usepackage[shortlabels]{enumitem}
\usepackage{scalerel}
\usepackage{stackengine}
\usepackage[
  separate-uncertainty = true,
  multi-part-units = repeat
]{siunitx}

\usetkzobj{all}
\usetikzlibrary{shapes.misc}

\MakeOuterQuote{"}

\setcounter{secnumdepth}{4}

\renewcommand\hateq{\mathrel{\stackon[1.5pt]{=}{\stretchto{%
				\scalerel*[\widthof{=}]{\wedge}{\rule{1ex}{3ex}}}{0.5ex}}}}

\newcommand*\circled[1]{
  \tikz[baseline=(C.base)]\node[draw,circle,inner sep=0.75pt](C) {#1};\!
}

\newcommand*{\obot}{\perp\mkern-20.7mu\bigcirc}

\DeclarePairedDelimiter\abs{\lvert}{\rvert}
\DeclarePairedDelimiter\norm{\lVert}{\rVert}
\makeatletter
\let\oldabs\abs
\def\abs{\@ifstar{\oldabs}{\oldabs*}}
\let\oldnorm\norm
\def\norm{\@ifstar{\oldnorm}{\oldnorm*}}
\makeatother

\renewcommand{\thesubsection}{\arabic{subsection}}
\titleformat{\section}{\normalfont\Large\bfseries}{Kapitel \arabic{section}: }{0em}{}
\titleformat{\subsection}{\normalfont\large\bfseries}{§\arabic{subsection} }{0em}{}
\titleformat{\subsubsection}{\normalfont\bfseries}{\arabic{subsection}.\arabic{subsubsection} }{0em}{}
\renewcommand{\cftsubsecpresnum}{§}
\newlength\mylength
\settowidth\mylength{\cftsubsecpresnum}
\settowidth\mylength{\cftsubsecaftersnum}
\addtolength\cftsubsecnumwidth{\mylength}
\renewcommand{\cftsecpresnum}{Kapitel }
\renewcommand{\cftsecaftersnum}{: }
\settowidth\mylength{\cftsecpresnum}
\addtolength\cftsecnumwidth{\mylength}

\newcommand{\ul}{\underline}
\renewcommand{\proof}{\ul{Beweis:}\\}
\renewcommand{\qed}{\begin{flushright}
\ul{\(q.e.d.\)}
\end{flushright}}
\let\origphi\phi
\let\phi\varphi
\let\origepsilon\epsilon
\let\epsilon\varepsilon

\title{Analysis 2}
\author{Nico Mexis}
\date{\today}

\begin{document}
\maketitle
\newpage

\tableofcontents
\newpage

\section{Metrik \& Topologie}
Sei $X$ eine Menge.\\
\ul{Metrik}: $d:X\times X\rightarrow\mathbb{R}_+$ Metrik, wenn:
\begin{enumerate}[1)]
	\item $d(x,y)=0\Leftrightarrow x=y$
	\item $d(x,y)=d(y,x)$
	\item $d(x,y)\leq d(x,z)+d(z,y)$
\end{enumerate}
\ul{Norm:} $\norm{\cdot}:X\rightarrow\mathbb{R}$ Norm, wenn:
\begin{enumerate}[1)]
	\item $\norm{x}\geq 0$ und $\norm{x}=0\Leftrightarrow x=0$
	\item $\norm{cx}=\abs{x}\cdot\norm{x},\ c\in\mathbb{R}$
	\item $\norm{x+y}\leq\norm{x}+\norm{y},\ x,y\in X$
\end{enumerate}
$\norm{\cdot}$ Norm $\Rightarrow$ $d(x,y)=\norm{x-y}$ induzierte Metrik
\begin{tabbing}
	\ul{Kugel}: \= $x\in X,\ r\in\mathbb{R}_+$\\
	\> abgeschlossene Kugel: \=$B_{\leq r}(x):=\{y\in X:d(x,y)\leq r\}$ (mit Rand)\\
	\> offene Kugel: \>$B_{<r}(x):=\{y\in X:d(x,y)<r\}$ (ohne Rand)
\end{tabbing}
\ul{Mengen}:
\begin{enumerate}[1)]
	\item $Y\subseteq X$ \ul{offen}, wenn $\forall y\in Y:\exists \epsilon>0:B_{<\epsilon}(y)\subseteq Y$
	\item $Y$ \ul{abgeschlossen}, wenn $X\backslash Y$ offen
\end{enumerate}
Offene Kugel = offene Menge\\
Abgeschlossene Kugel = abgeschlossene Menge\\
$X$ metrischer Raum $\Rightarrow$
\begin{enumerate}[1)]
	\item $X$ und $\varnothing$ offen
	\item Jede \ul{abzählbare} Vereinigung von offenen Mengen ist offen
	\item Jeder \ul{endliche} Durchschnitt ist offen
	\item $X$ und $\varnothing$ abgeschlossen
	\item $\bigcup_{<\infty}\text{abg}=\text{abg}$
	\item $\bigcap_{\infty}\text{abg}=\text{abg}$
\end{enumerate}
\ul{Umgebung}: $U\subseteq X$ von $x$: $\exists\epsilon>0:B_{<\epsilon}(x)\subseteq U$\\
\ul{Abschluss}:
\begin{enumerate}[1)]
	\item $x\in X$ \ul{Berührpunkt}/\ul{Kontaktpunkt} zu $Y\subseteq X$, wenn $\forall \text{Umgebung }U\text{ von }x:U\cap Y\neq\varnothing$
	\item \ul{Abschluss} von $Y$: $\overline{Y}=\{\text{Kontaktpunkte an }Y\}$
\end{enumerate}
$Y,Y'\subseteq X$ $\Rightarrow$
\begin{enumerate}[1)]
	\item $Y\subseteq\overline{Y},\ \overline{\overline{Y}}=\overline{Y}$
	\item $\overline{Y\cup Y'}=\overline{Y}\cup\overline{Y'}$
	\item $\overline{Y}$ abgeschlossen
\end{enumerate}
\ul{Randpunkt:} $x\in X$ Randpunkt zu $Y\subseteq X$, wenn $\forall \text{Umgebung }U\text{ von }x:U\cap Y\neq\varnothing\neq U\cap(X\backslash Y)$\\
\ul{Rand}: $\partial Y=\{\text{Randpunkte von }Y\}$\\
$Y\subseteq X$ $\Rightarrow$ Rand $\partial Y$ erfüllt
\begin{enumerate}[1)]
	\item $Y\backslash\partial Y$ offen
	\item $Y\cup\partial Y$ abgeschlossen
	\item $\partial Y$ abgeschlossen
\end{enumerate}
\ul{Topologie}: $\mathcal{T}$ System von Teilmengen von $X$ ($\mathcal{T}\subseteq 2^X=\{0,1\}^X=\mathcal{P}(X)$)\\
Topologie, wenn
\begin{enumerate}[1)]
	\item $X,\varnothing\in\mathcal{T}$
	\item $T,T'\in\mathcal{T}\Rightarrow T\cap T'\in\mathcal{T}$
	\item $T_z\in\mathcal{T},\ z\in I\Rightarrow\bigcup_{i\in I}T_i\in\mathcal{T}$
\end{enumerate}
$X$ topologischer Raum, $\mathcal{T}$ auf $X$\\
$T\in\mathcal{T}$ "offene Menge"\\
$U\subseteq X$ Umgebung von $x$, wenn $x\in T\subseteq U$\\
$Y\subseteq X$ "offen" $\Leftrightarrow$ $\forall y\in Y:\exists\text{Umgebung }U\text{ von }Y:U\subseteq Y$\\
$Y$ abgeschlossen $\Leftrightarrow$ $X\backslash Y$ offen\\
\begin{tabbing}
	\ul{Hausdorffraum}: \=topologischer Raum\\
	\>$x\neq x'\Rightarrow U,U'$ mit $x\in U,x'\in U'$ und $U\cap U'=\varnothing$ "punkttrennend"
\end{tabbing}
Metrischer Raum $\Rightarrow$ Hausdorffraum\\
$X$ topologischer Raum und $Y\subseteq X$ $\Rightarrow$ $\overline{Y}$ abgeschlossen
\end{document}






























