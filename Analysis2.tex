\documentclass[a4paper]{article}

\usepackage[ngerman]{babel}
\usepackage[utf8]{inputenc}
\usepackage{amsthm}
\usepackage{amsmath}
\usepackage{amssymb}
\usepackage{tikz,tkz-euclide}
\usepackage{titlesec}
\usepackage{gensymb}
\usepackage{textcomp}
\usepackage[titles]{tocloft}
\usepackage{csquotes}
\usepackage[babel]{microtype}
\usepackage{MnSymbol}
\usepackage{stmaryrd}
\usepackage{mathtools}
\usepackage{ulem}
\usepackage[shortlabels]{enumitem}
\usepackage{scalerel}
\usepackage{stackengine}
\usepackage[
  separate-uncertainty = true,
  multi-part-units = repeat
]{siunitx}

\usetkzobj{all}
\usetikzlibrary{shapes.misc}

\MakeOuterQuote{"}

\setcounter{secnumdepth}{4}

\renewcommand\hateq{\mathrel{\stackon[1.5pt]{=}{\stretchto{%
				\scalerel*[\widthof{=}]{\wedge}{\rule{1ex}{3ex}}}{0.5ex}}}}

\newcommand*\circled[1]{
  \tikz[baseline=(C.base)]\node[draw,circle,inner sep=0.75pt](C) {#1};\!
}

\newcommand*{\obot}{\perp\mkern-20.7mu\bigcirc}

\DeclarePairedDelimiter\abs{\lvert}{\rvert}
\DeclarePairedDelimiter\norm{\lVert}{\rVert}
\makeatletter
\let\oldabs\abs
\def\abs{\@ifstar{\oldabs}{\oldabs*}}
\let\oldnorm\norm
\def\norm{\@ifstar{\oldnorm}{\oldnorm*}}
\makeatother

\newcommand{\ul}{\underline}
\renewcommand{\proof}{\ul{Beweis:}\\}
\renewcommand{\qed}{\begin{flushright}
\ul{\(q.e.d.\)}
\end{flushright}}
\let\origphi\phi
\let\phi\varphi
\let\origepsilon\epsilon
\let\epsilon\varepsilon

\title{Analysis 2}
\author{Nico Mexis}
\date{\today}

\begin{document}
\maketitle
\newpage

\tableofcontents
\newpage

\setcounter{section}{-1}
\section{Wiederholung}
\ul{Satz von Taylor:}\\
Sei $f\in C^{n+1}(I)$, $x,x_0\in I$. Dann gilt:
$$f(x)=\sum_{k=0}^n\frac{f^{(k)}(x_0)}{k!}(x-x_0)^k+\frac{1}{n!}\int_{x_0}^{x}(x-t)^nf^{(n+1)}(t)\text{d}t$$
\section{Metrik \& Topologie}
Sei $X$ eine Menge.\\
\ul{Metrik}: $d:X\times X\rightarrow\mathbb{R}_+$ Metrik, wenn:
\begin{enumerate}[1)]
	\item $d(x,y)=0\Leftrightarrow x=y$
	\item $d(x,y)=d(y,x)$
	\item $d(x,y)\leq d(x,z)+d(z,y)$
\end{enumerate}
\ul{Norm:} $\norm{\cdot}:X\rightarrow\mathbb{R}$ Norm, wenn:
\begin{enumerate}[1)]
	\item $\norm{x}\geq 0$ und $\norm{x}=0\Leftrightarrow x=0$
	\item $\norm{cx}=\abs{x}\cdot\norm{x},\ c\in\mathbb{R}$
	\item $\norm{x+y}\leq\norm{x}+\norm{y},\ x,y\in X$
\end{enumerate}
$\norm{\cdot}$ Norm $\Rightarrow$ $d(x,y)=\norm{x-y}$ induzierte Metrik
\begin{tabbing}
	\ul{Kugel}: \= $x\in X,\ r\in\mathbb{R}_+$\\
	\> abgeschlossene Kugel: \=$B_{\leq r}(x):=\{y\in X:d(x,y)\leq r\}$ (mit Rand)\\
	\> offene Kugel: \>$B_{<r}(x):=\{y\in X:d(x,y)<r\}$ (ohne Rand)
\end{tabbing}
\ul{Mengen}:
\begin{enumerate}[1)]
	\item $Y\subseteq X$ \ul{offen}, wenn $\forall y\in Y:\exists \epsilon>0:B_{<\epsilon}(y)\subseteq Y$
	\item $Y$ \ul{abgeschlossen}, wenn $X\backslash Y$ offen
\end{enumerate}
Offene Kugel = offene Menge\\
Abgeschlossene Kugel = abgeschlossene Menge\\
$X$ metrischer Raum $\Rightarrow$
\begin{enumerate}[1)]
	\item $X$ und $\varnothing$ offen
	\item Jede \ul{abzählbare} Vereinigung von offenen Mengen ist offen
	\item Jeder \ul{endliche} Durchschnitt ist offen
	\item $X$ und $\varnothing$ abgeschlossen
	\item $\bigcup_{<\infty}\text{abg}=\text{abg}$
	\item $\bigcap_{\infty}\text{abg}=\text{abg}$
\end{enumerate}
\ul{Umgebung}: $U\subseteq X$ von $x$: $\exists\epsilon>0:B_{<\epsilon}(x)\subseteq U$\\
\ul{Abschluss}:
\begin{enumerate}[1)]
	\item $x\in X$ \ul{Berührpunkt}/\ul{Kontaktpunkt} zu $Y\subseteq X$, wenn $\forall \text{Umgebung }U\text{ von }x:U\cap Y\neq\varnothing$
	\item \ul{Abschluss} von $Y$: $\overline{Y}=\{\text{Kontaktpunkte an }Y\}$
\end{enumerate}
$Y,Y'\subseteq X$ $\Rightarrow$
\begin{enumerate}[1)]
	\item $Y\subseteq\overline{Y},\ \overline{\overline{Y}}=\overline{Y}$
	\item $\overline{Y\cup Y'}=\overline{Y}\cup\overline{Y'}$
	\item $\overline{Y}$ abgeschlossen
\end{enumerate}
\ul{Randpunkt:} $x\in X$ Randpunkt zu $Y\subseteq X$, wenn $\forall \text{Umgebung }U\text{ von }x:U\cap Y\neq\varnothing\neq U\cap(X\backslash Y)$\\
\ul{Rand}: $\partial Y=\{\text{Randpunkte von }Y\}$\\
$Y\subseteq X$ $\Rightarrow$ Rand $\partial Y$ erfüllt
\begin{enumerate}[1)]
	\item $Y\backslash\partial Y$ offen
	\item $Y\cup\partial Y$ abgeschlossen
	\item $\partial Y$ abgeschlossen
\end{enumerate}
\ul{Topologie}: $\mathcal{T}$ System von Teilmengen von $X$ ($\mathcal{T}\subseteq 2^X=\{0,1\}^X=\mathcal{P}(X)$)\\
Topologie, wenn
\begin{enumerate}[1)]
	\item $X,\varnothing\in\mathcal{T}$
	\item $T,T'\in\mathcal{T}\Rightarrow T\cap T'\in\mathcal{T}$
	\item $T_z\in\mathcal{T},\ z\in I\Rightarrow\bigcup_{i\in I}T_i\in\mathcal{T}$
\end{enumerate}
$X$ topologischer Raum, $\mathcal{T}$ auf $X$\\
$T\in\mathcal{T}$ "offene Menge"\\
$U\subseteq X$ Umgebung von $x$, wenn $x\in T\subseteq U$\\
$Y\subseteq X$ "offen" $\Leftrightarrow$ $\forall y\in Y:\exists\text{Umgebung }U\text{ von }Y:U\subseteq Y$\\
$Y$ abgeschlossen $\Leftrightarrow$ $X\backslash Y$ offen\\
\begin{tabbing}
	\ul{Hausdorffraum}: \=topologischer Raum\\
	\>$x\neq x'\Rightarrow U,U'$ mit $x\in U,x'\in U'$ und $U\cap U'=\varnothing$ "punkttrennend"
\end{tabbing}
Metrischer Raum $\Rightarrow$ Hausdorffraum\\
$X$ topologischer Raum und $Y\subseteq X$ $\Rightarrow$ $\overline{Y}$ abgeschlossen\\
\ul{Diskret}: $\mathcal{T}=\{T\subseteq X\}$\\
\ul{Indiskret}: $\mathcal{T}=\{\varnothing,X\}$\\
$y\in Y\subseteq X$ \ul{innerer Punkt}, wenn es Umgebung $U$ von $y$ gibt mit $U\subseteq Y$\\
\ul{Inneres} von $Y$: $Y^\circ=\{\text{innere Punkte von }Y\}$\\
Sei $Y\subseteq X$: $\partial Y=\overline{Y}\backslash Y^\circ$\\
Die Folge $a=(a_n:n\in\mathbb{N})$ heißt \ul{konvergent} mit Grenzwert $a^*\in X$, wenn es zu jeder Umgebung $U$ von $a^*$ ein $n_0$ gibt, mit $a_n\in U$, $u\geq u_0$.\\
$a^*$ Häufungspunkt: $\#\{n:a_n\in U\}=\infty$ für alle $U(a^*)$\\
$Y\subseteq X$ $\Rightarrow$
\begin{enumerate}[1)]
	\item $Y$ abgeschlossen, $a$ konvergente Folge in $Y$ $\Rightarrow$ $a^*\in Y$
	\item $a$ konvergent $\Rightarrow$ $a^*\in \overline{Y}$
	\item $X$ metrisch: $Y$ abgeschlossen $\Leftrightarrow$ $a^*\in Y$
\end{enumerate}
Hausdorffraum $\Rightarrow$ Grenzwert eindeutig\\
$f:X\rightarrow X'$ heißt "\ul{stetig}", wenn für alle $x_n\rightarrow x^*$ auch $f(x_n)\rightarrow f(x^*)$ gilt.\\
$f:X\rightarrow X'$, $Y'\subseteq X'$: $f^{-1}(X'):=\{x\in X\ \vert\ f(x)\in Y'\}$ \ul{Urbild}\\
$f:X\rightarrow X'$ stetig $\Leftrightarrow$ Urbilder abgeschlossener Mengen sind abgeschlossen\\
$f:X\rightarrow X'$ stetig $\Leftrightarrow$ Urbilder offener Mengen sind offen\\
$a=(a_n:n\in\mathbb{N})$ Cauchy-Folge, wenn es zu $\epsilon>0$ ein $n_0$ gibt, sodass $d(x_n,x_m)<\epsilon$, $n,m\geq n_0$\\
$X$ vollständig, wenn Cauchy $\Rightarrow$ Konvergenz\\
Vervollständigung: $\{\text{Cauchy-Folgen}\}$\\
$a\sim a'\Leftrightarrow \lim_{n\rightarrow 0}d(a_n,a_n')=0$\\
$Y\subset X$ \ul{dicht}, wenn $\overline{Y}=X$\\
$X$ ist vollständig $\Leftrightarrow$ jede Folge $B_{\leq \epsilon_1}(x_1)\supset B_{\leq \epsilon_2}(x_2)$ mit $\epsilon_1>\epsilon_2>\dots$ und $\epsilon_n\rightarrow 0$ erfüllt:
$$\bigcap_{j=1}^\infty B_{\leq \epsilon_j}(x_j)\neq\varnothing$$\\
\ul{Kontraktion}: $f:X\rightarrow X$, $d(f(x),f(x'))\leq \rho\cdot d(x,x')$, $0<\rho<1$\\
$f:X\rightarrow X$ Kontraktion mit $X$ vollständig $\Rightarrow$ $f$ hat \ul{genau einen} Fixpunkt $f(x^*)=x^*\in X$\\\\
$Y\subseteq X$
\begin{enumerate}[1)]
	\item \ul{Überdeckungskompakt}: Für jede Überdeckung $Y\subseteq\bigcup_{U\in \mathcal{U}}U$ von offenen Umgebungen gibt es $U_1,\dots,U_n$ mit $Y\subseteq U_1\cup\dots\cup U_n$
	\item \ul{Folgenkompakt}: Jede Folge in $Y$ enthält konvergente Teilfolge
\end{enumerate}
$Y\subseteq X$ metrisch ist überdeckungskompakt $\Leftrightarrow$ $Y$ folgenkompakt\\
\begin{enumerate}[1)]
	\item $A\subseteq Y$ heißt ein $\epsilon$-Netz für $Y$, wenn $Y\ni y\in B_{<\epsilon}(a)$, $a\in A$
	\item $Y$ \ul{total beschränkt}: $\exists \text{endliches }\epsilon\text{-Netz }A\text{ für jedes }\epsilon>0$
	\item $Y$ \ul{beschränkt}: $d(y,y')<M$, $y,y'\in Y$\\
	\ul{Durchmesser}: $d(Y):=\sup_{y,y'}d(y,y')$\\
	\ul{Radius}: $r(Y):=\inf\sup_{y,y'}d(y,y')$
\end{enumerate}
Jede total beschränkte Menge ist beschränkt.\\
$Y\subset X$, $Y$ abgeschlossen, $X$ vollständig, dann: $Y$ folgenkompakt $\Leftrightarrow$ $Y$ total beschränkt\\
$Y\subset X$ metrisch, $Y$ überdeckungskompakt $\Rightarrow$ $Y$ abgeschlossen, beschränkt\\
$Y$ kompakt $\Rightarrow$ abgeschlossen + beschränkt\\
abgeschlossen $Y'\subseteq Y$ kompakt $\Rightarrow$ $Y'$ kompakt\\
$f:X\rightarrow X'$, $Y\subseteq X$ kompakt $\Rightarrow$ $f(Y)=:Y'\subseteq X'$ kompakt\\
$f:X\rightarrow X'$ gleichmäßig stetig, wenn es zu jedem $\epsilon>0$ ein $\delta>0$ gibt, sodass $d(x,x')<\delta\ \Rightarrow\ d(f(x),f(x'))<\epsilon$\\
$f:X\rightarrow X'$ stetig, $Y\subseteq X$ kompakt $\Rightarrow$ $f:Y\rightarrow X'$ gleichmäßig stetig\\
$\mathbb{R}^n$ endlich-dimensionaler, vollständiger, normierter Raum\\
$\norm{x}_p=(\sum_{j=1}^n\abs{x_j}^p)1\frac{1}{p}$ "$p$-Norm"\\
$\norm{x}_1=\sum_{j=1}^n\abs{x_j}$, $\norm{x}_2=\sqrt{\sum_{j=1}^n\abs{x_j}^2}$, $\norm{x}_\infty=\max_{j=1,\dots,n}\abs{x_j}$\\
Einheitskugeln: $\{x:\norm{x}\leq 1\}$\\
\ul{Quader}: $\left[a,b\right]=\left[a_1,b_1\right]\times\dots\times\left[a_n,b_n\right]$\\
$\Omega\subset\mathbb{R}^n$ beschränkt: $x\in\Omega$: $\{\norm{x-y}:y\in\Omega\}\leq M$\\
Abgeschlossene Quader im $\mathbb{R}^n$ sind kompakt\\
$\Omega\subset\mathbb{R}^n$: kompakt $\Leftrightarrow$ abgeschlossen + beschränkt
\section{Kurven}
\ul{Kurve}: $f:I\mathbb{R}^d$ stetig\\
$f(I)\subseteq\mathbb{R}^d$, $f$ \ul{Parametrisierung}\\
$f$ als "Vektor": $f=(f_j:j=1,\dots,d)$\\
Polynomiale Kurven: $f(x)=\begin{pmatrix}
f_1(x)\\
\vdots\\
f_d(x)
\end{pmatrix}=\begin{pmatrix}
\sum_{k=0}^n a_{1k}x^k\\
\vdots\\
\sum_{k=0}^n a_{dk}x^k
\end{pmatrix}=\sum_{k=0}^n\begin{pmatrix}
a_{1k}\\
\vdots\\
a_{dk}
\end{pmatrix}x^k=\sum_{k=0}^na_kx^k$\\
$f$ differenzierbar, wenn alle $f_j$ differenzierbar sind\\
\ul{Tangente}: $f'=\begin{pmatrix}
f_1'\\
\vdots\\
f_d'
\end{pmatrix}$\\
$x$ \ul{singulär}, wenn $f'(x)=0$. $f$ heißt "regulär", wenn $f'(x)\neq 0$, $x\in I$\\
\ul{Bogenlänge}: $f:I=\left[a,b\right]\rightarrow\mathbb{R}^d$, $x_0<\dots<x_N$, $x_0=a$, $x_N=b$, $Lf(x_1,\dots,x_N)=\sum_{k=1}^N\norm{f(x_k)-f(x_{k-1})}_2$\\
$f:I\rightarrow\mathbb{R}^d$ "\ul{rektifizierbar}" mit Länge $L$, wenn $\forall\epsilon > 0:\exists\delta>0:\forall a=x_0<\dots<x_N=b:\abs{x_{j+1}-x_j}<\delta$\\
$\Rightarrow$ $\abs{Lf(x_0,\dots,x_N)-L}<\epsilon$\\
$f\in C^1(I)$ $\Rightarrow$ $f$ rektifizierbar und $L=\int_a^b\norm{f'(x)}_2dx$\\
$\int_a^b\norm{\begin{pmatrix}
\cos(x)\\
\sin(x)
\end{pmatrix}}_2dx=\int_a^b\sqrt{\sin^2x+\cos^2x}dx=b-a$\\
$f\in C^1(I)$ $\Rightarrow$ $\int_a^b\sqrt{1+(f'(x))^2}dx=L$\\
$\forall f:I\rightarrow\mathbb{R}^d$ stetig diff'bar, $I$ kompakt, $\epsilon>0:\exists\delta >0:x\neq x',\abs{x-x'}<\delta\Rightarrow\norm{\frac{f(x)-f(x')}{x-x'}-f'(x)}_2<\epsilon$\\
$\phi:I'\rightarrow I$ \ul{Reparametrisierung}, wenn $\phi$ stetig+bijektiv\\
$\phi$ \ul{regulär}, wenn $\phi\in C^1(I')$, $\phi'(x)\neq 0$\\
\ul{Tangente}:\\
$(f\circ\phi)'=f'(\phi(x))\phi'(x)$\\
$\phi'>0$: orientierungstreu\\
$f$ regulär + $\phi$ regulär $\Rightarrow$ $(f\circ\phi)$ regulär\\
$x\in I$ $\Rightarrow$ $l(x)=Lf(x)=\int_a^x\norm{f'(t)}_2dt$\\
$l'(x)=\norm{f'(t)}_2>0$\\
$l:I\rightarrow\left[0,L\right]$ $\Rightarrow$ ex. $l^{-1}=:\phi=Lf^{-1}:\left[0,L\right]\rightarrow I$\\
$\left[0,L\right]\ni S=\int_a^{\phi(S)}\norm{f'(t)}_2dt$\\
$\phi'(x)=\frac{1}{l'(\phi(x))}=\norm{f'(\phi(x))}_2^{-1}$\\
$f_\phi=f\circ\phi:\left[0,L\right]\rightarrow\mathbb{R}^d$\\
$f$ heißt \ul{bogenparametrisiert}, wenn $I=\left[0,L\right]$ und $Lf(x)=x$\\
$f\in C^1$ bogenlängenparametrisiert $\Leftrightarrow$ $\norm{f'(x)}=1$\\
$f\phi(x)=(f\circ\phi)(x)$
\section{Differentialrechnung in mehreren Variablen}
$\frac{\partial f}{\partial x_j}:=f(x_1,\dots,x_{j-1},\cdot,x_{j+1},\dots,x_n)'$ partielle Ableitung\\
\begin{enumerate}[1)]
	\item $e_j\in\mathbb{R}^n: (e_j)_k=\delta_{jk},j,k=1,\dots,n$ j-ter Einheitsvektor im $\mathbb{R}^n$
	\item $f:U\rightarrow\mathbb{R}$, $U\subseteq\mathbb{R}^n$ offen, heißt \ul{"partiell differenzierbar"} nach $x_j$ an $x\in U$, wenn $\frac{\partial f}{\partial x_j}(x):=\lim_{h\rightarrow 0} \frac{f(x+he_j)-f(x)}{h}$ existiert.
	\item Partielle Ableitung stetig $\hateq$ \ul{stetig partiell diff'bar}
	\item $C^1(U)=\{\text{stetig partiell diff'bar}\}$
	\item $\nabla f(x)=\begin{pmatrix}
	\frac{\partial f}{\partial x_1}\\
	\vdots\\
	\frac{\partial f}{\partial x_n}
	\end{pmatrix}(x)=(\frac{\partial f}{\partial x_j}(x):j=1,\dots,n)$ \ul{"Gradient"}
\end{enumerate}
$f:U\rightarrow\mathbb{R}$ "richtungsdiff'bar" an x in \ul{Richtung} $y\in\mathbb{R}^n$, wenn $D_yf(x):=\lim_{h\rightarrow 0}\frac{f(x+h)-f(x)}{h}$ existiert $\Rightarrow$ \ul{Richtungsableitung}\\
$x,y\in\mathbb{R}^n:\left[x,y\right]=\{\alpha x+(1-\alpha)y:0\leq\alpha\leq 1\}$ \ul{"konvexe Hülle"}\\
$f:U\rightarrow\mathbb{R}$, $x\in U$, $y\in U$, $\left[x,y\right]\subset U$
\begin{enumerate}[1)]
	\item $f(y)-f(x)=\int_0^1 D_{y-x}f(x+t(y-x))dt$
	\item $f(y)-f(x)=D_{y-x}f(\xi)$, $\xi\in\left[x,y\right]$
\end{enumerate}
$D_{\lambda y}f(x)=\lambda D_yf(x),\lambda\in\mathbb{R}$\\
$\frac{\partial^2}{\partial x\partial y}f:=\frac{\partial f}{\partial x}\frac{\partial f}{\partial y}f$\\
Existieren $\frac{\partial^2 f}{\partial x_j\partial x_k}$ und $\frac{\partial^2 f}{\partial x_k\partial x_j}$ von $f$ in $U\ni x$ und sind stetig in $x$ $\Rightarrow$ $\frac{\partial^2 f}{\partial x_j\partial x_k}=\frac{\partial^2 f}{\partial x_k\partial x_j}$\\
$$Hf(x)=\left(\frac{\partial^2f}{\partial x_j\partial x_k}:j,k=1,\dots,n\right)=\begin{pmatrix}
\frac{\partial^2f}{\partial x_1\partial x_1} & \hdots & \frac{\partial^2f}{\partial x_1\partial x_n}\\
\vdots & \ddots & \vdots\\
\frac{\partial^2f}{\partial x_n\partial x_1} & \hdots & \frac{\partial^2f}{\partial x_n\partial x_n}
\end{pmatrix} \text{ (Hesse'sche Matrix)}$$
\ul{Multiindex}:
\begin{itemize}
	\item $\alpha=(\alpha_1,\dots,\alpha_n)\in\mathbb{N}_0^n$, $\abs{\alpha}=\sum_{j=1}^n\alpha_j$
	\item $\epsilon_j=(\delta_{jk}:k=1,\dots,n)\in\mathbb{N}_0^n$
	\item $\alpha!=\alpha_1!\cdot\dots\cdot\alpha_n!$
\end{itemize}
$$\frac{\partial^{\abs{\alpha}}}{\partial x^\alpha}=\frac{\partial^{\abs{\alpha}}}{\partial x_1^{\alpha_1}\partial x_n^{\alpha_n}}=\left(\frac{\partial}{\partial x_1}\right)^{\alpha_1}\cdot\dots\cdot\left(\frac{\partial}{\partial x_n}\right)^{\alpha_n}$$
$C^k(D)$, $D\subseteq\mathbb{R}^n$ $:=$ Menge aller Funktionen, für die $\frac{\partial^k f}{\partial x_{j_1}\cdot\dots\cdots\partial x_{j_k}}$, $j_1,\dots,j_k\in\{1,\dots,n\}$ stetig sind.\\
$p(x)=\sum_\alpha p_\alpha x^\alpha$ $\rightarrow$ $p(D)=p(\frac{\partial}{\partial x})=\sum p_\alpha\frac{\partial^{\abs{\alpha}}}{\partial x^\alpha}$\\
$\langle p,q\rangle=(p(D)q)(0)$ Skalarprodukt\\
\ul{Satz von Schwarz}: Für allee $f\in C^k(D)$ sind alle partielle Ableitungen von der Reihenfolge unabhängig.
\begin{enumerate}[1)]
	\item $f:\mathbb{R}^n\rightarrow\mathbb{R}$ linear, $f(x)=a^Tx$\\
	affin, $f(x)=a^Tx+b$, $a\in\mathbb{R}^n$, $b\in\mathbb{R}$
	\item $f:U\rightarrow\mathbb{R}$, $x\in U\subset\mathbb{R}^n$, wenn es $l(x):=a^Tx$ gibt, sodass $\lim_{\norm{h}\rightarrow 0}\frac{\abs{f(x+h)-f(x)-l(h)}}{\norm{h}}=0$
	\item \ul{Ableitung} von $f$ an $x$: $f'=Df=l$
\end{enumerate}
$f:U\rightarrow\mathbb{R}$ sei in $x\in U$ diff'bar.
\begin{enumerate}[1)]
	\item $f$ ist stetig
	\item $Df$ ist eindeutig
	\item $f$ ist partiell diff'bar
	\item $Df(x)$ hat die Form: $y\mapsto y^T\nabla f(x)$
	\item $f$ ist in \ul{jede Richtung} richtungsdiff'bar: $D_yf(x)=y^T\nabla f(x)=(Df(x))(y)$
\end{enumerate}
$f:U\rightarrow\mathbb{R}$, $x\in U'\subseteq U$, $f$ partiell diff'bar an $X$ und $\frac{\partial f}{\partial x_j}$ stetig in $x$ $\Rightarrow$ $f$ ist an $x$ diff'bar.\\
stetig diff'bar ($\Rightarrow\ \nabla f(x)$ stetig $\Rightarrow$ $y\mapsto y^T\nabla f(x)$ stetig in $x$) $\Rightarrow$ stetig partiell diff'bar $\Rightarrow$ diff'bar $\Rightarrow$ partiell diff'bar\\
$f:U\rightarrow\mathbb{R}$ heißt \ul{$k$-mal stetig diff'bar}, wenn $\frac{\partial^{\abs{\alpha}}}{\partial x^\alpha}f$, $\abs{\alpha}\leq k$, existieren + stetig sind\\
$f:U\rightarrow\mathbb{R}$, $k$-mal stetig diff'bar an $x$ $$\Rightarrow\ D_y^kf(x)=\sum_{\abs{\alpha}=k}\frac{k!}{\alpha!}\frac{\partial^kf}{\partial x^\alpha}(x)y^\alpha$$
$f:U\rightarrow\mathbb{R}$, $k$-mal stetig diff'bar, $y\in\mathbb{R}^n$ mit $\left[x,x+y\right]\subset U$. Dann gilt für $g:t\rightarrow f(x+ty)$, $t\in\left[0,1\right]$ $\Rightarrow$ $g^{(j)}(t)=D_y^j f(x+ty)$\\
$f\in C^{k+1}(\Omega)$, $x,y\in\Omega$ mit $\left[x,x+y\right]\subseteq\Omega$. Es gibt $\xi\in\left[x,y\right]$ mit $$f(x+y)=\sum_{\abs{\alpha}\leq k}\frac{1}{\alpha!}\frac{\partial^{\abs{\alpha}}}{\partial x^\alpha}f(x)y^\alpha+\sum_{\abs{\alpha}=k+1}\frac{1}{\alpha!}\frac{\partial^{\abs{\alpha}}f}{\partial x^\alpha}(\xi)y^\alpha$$
$f$ an $x$ $k$-mal stetig differenzierbar $\Rightarrow$
\begin{enumerate}[1)]
	\item $D^kf$ symmetrisch: $D^kf(x)(y^{(1)},\dots,y^{(k)})=D^kf(x)(y^{\sigma(1)},\dots,y^{\sigma(k)})$ mit $\sigma$ Permutation
	\item $D^kf$ linear in jeder Variablen: $D^kf(x)(\dots,\alpha y^{(j)}+\beta z^{(j)},\dots)=\alpha D^kf(x)(\dots,y^{(j)},\dots)+\beta D^kf(x)(\dots,z^{(j)},\dots)$ (Multilinearform, $k$-Linearform)
	\item Diagonale: $D^kf(x)=(y,\dots,y)$ ist $k$-homogen in $y$
	\item $D^kf(x)(\underbrace{e_1,\dots,e_1}_{\alpha_1},\dots,\underbrace{e_n,\dots,e_n}_{\alpha_n})=\frac{\partial^kf}{\partial x^\alpha}$, $\abs{\alpha}=k$
\end{enumerate}
$D^kf(x)$ ist eine symmetrische Multilinearform\\
1-Linearform: $y\mapsto a^Ty$\\
Bilinearform: $(y,z)\mapsto y^TAz$\\
\ul{Polarisierungsformel/Blossoming}: Zu jedem $k$-homogenen Polynom $p:\mathbb{R}^n\rightarrow\mathbb{R}$ gibt es genau eine $k$-Linearform $P$ mit $p(x)=P(x,\dots,x)$ und umgekehrt.
\begin{enumerate}[1)]
	\item $k$-tes Taylorpolynom an $x$: $T_kf:y\mapsto\sum_{\abs{\alpha}\leq k}\frac{1}{\alpha}\frac{\partial^\alpha f}{\partial x^\alpha}(x)(y-x)^\alpha=\sum_{j=0}^k\sum_{\abs{\alpha}=j}p_\alpha x^\alpha=: \sum_{j=0}^k p_j(x)$
	\item $p$ hat Totalgrad $k$: $p(x)=\sum_{\abs{\alpha}\leq k}p_\alpha x^\alpha=\sum_{j=0}^kp_j(x)$
\end{enumerate}
$f:U\rightarrow\mathbb{R}$ $k$-mal diff'bar an $x\in U$, wenn es $p=p_x$ vom Grad $k$ gibt, sodass $$\lim_{h\rightarrow 0}\frac{\abs{f(x+h)-p_x(h)}}{\norm{h}_2^k}=0$$
$$\frac{\abs{f(x+h)-f(x)-l(h)}}{\norm{h}_2}\rightarrow 0$$
Leitform von $p$ $\rightarrow$ $\sum_{\abs{\alpha} = k}p_\alpha x^\alpha$ $\rightarrow$ $D^kf(x)$\\
$f:U\rightarrow\mathbb{R}$, $k$-mal diff'bar an $x$
\begin{enumerate}[1)]
	\item $f$ ist $j$-mal diff'bar ($0\leq j \leq k$)
	\item $p_x$ ist eindeutig
	\item $f\in C^k(\Omega)$ $\Rightarrow$ $p_x=T_kf$
\end{enumerate}
\ul{Parametrische Flächen}: $F:\mathbb{R}^n\rightarrow\mathbb{R}^d$\\
$f:U\subseteq\mathbb{R}^n\rightarrow\mathbb{R}^d$ heißt
\begin{enumerate}[1)]
	\item partiell diff'bar nach $x_j$ wenn $\frac{\partial f}{\partial x_j}:=\lim_{h\rightarrow0}\frac{f(x+he_j)-f(x)}{h}\in\mathbb{R}^d$ existiert.
	\item richtungsdiff'bar: $D_yf(x)=\lim_{h\rightarrow0}\frac{f(x+hy)-f(x)}{h}\in\mathbb{R}^d$
	\item diff'bar, wenn $y\mapsto D_yf(x)$ linear
\end{enumerate}
TODO VL\\
\ul{Minimum/Minimalstelle} $x$ von $f(x)$.\\
Es gibt $U\ni x$ mit $f(x)\leq f(y)$, $y\in U$.\\
Strikt: $f(x)<f(y)$, $x\neq y\in U$\\
$f'(x)=0$\\
$x$ ist Minimalstelle von $f$ $\Leftrightarrow$ $x$ ist Maximalstelle von $-f$\\
$f:U\rightarrow\mathbb{R}$, $U\subseteq\mathbb{R}^n$ offen, partiell diff'bar, $x$ lokales Extremum $\Rightarrow$ $\nabla f(x)=0$\\
$A\in\mathbb{R}^{n\times n}$ positiv definit, wenn $y^TAy>0$, $y\in\mathbb{R}^n\backslash\{0\}$ (positiv semidefinit "$\geq$").\\
\ul{indefinit}, wenn es zwei EW gibt mit $\lambda_1<0$ und $\lambda_2>0$.\\
$A\in\mathbb{R}^{n\times n}$ ist positiv (semi-)definit $\Leftrightarrow$ $\det\begin{bmatrix}
a_{11} & \hdots & a_{1k}\\
\vdots & \ddots & \vdots\\
a_{k1} & \hdots & a_{kk}
\end{bmatrix}>0$ (oder $\geq$)\\
$f\in\mathcal{C}^2(D)$, $x\in U\subset D$, $U$ offen,
\begin{enumerate}[1)]
	\item $f$ an $x$ \ul{lokales Minimum} $\Rightarrow$ $\nabla f(x)=0$, $Hf(x)$ positiv semidefinit
	\item $\nabla f(x)=0$, $Hf(x)$ positiv definit $\Rightarrow$ $f$ hat an $x$ ein striktes lokales Minimum
\end{enumerate}
$f\in\mathcal{C}^2(D),\dots,\nabla f(x)=0$
\begin{enumerate}
	\item $Hf(x)$ hat $\lambda>0$ $\Rightarrow$ $x$ \ul{kein} lokales Maximum
	\item $HF(x)$ hat $\lambda<0$ $\Rightarrow$ $x$ \ul{kein} lokales Minimum
\end{enumerate}
\ul{Konvexität}:
\begin{enumerate}
	\item $\Omega\subseteq\mathbb{R}^n$ \ul{konvex}, wenn $x,x'\in\Omega$ $\Rightarrow$ $\Omega\supset\left[x,x'\right]=\{(1-\alpha)x+\alpha x'\ \vert\ \alpha\in\left[0,1\right]\}$
	\item $f:\Omega\rightarrow\mathbb{R}$, $f((1-\alpha)x+\alpha x)\leq (1-\alpha)f(x)+\alpha f(x')$, $\alpha\in\left[0,1\right]$, strikt: "<" für $\alpha\in(0,1)$
\end{enumerate}
$\Omega\subseteq\mathbb{R}^n$ konvex + offen, $f:\Omega\rightarrow\mathbb{R}$ $\Rightarrow$ $Hf(x)$ positiv semidefinit, $x\in\Omega$\\
Einheitskreis: $\mathbb{S}^2=\{(x,y)\ \vert\ x^2+y^2=1\}$\\
Einheitsscheibe: $\mathbb{D}^2=\{(x,y)\ \vert\ x^2+y^2\leq1\}$\\
$\nabla f(x)=\begin{pmatrix}
a\\b
\end{pmatrix}\neq0$\\
\ul{Tangentialkegel}: $\Omega\subset\mathbb{R}^n$, $x\in\Omega$. Menge aller $y\in\mathbb{R}^n$ mit: $\exists x_k\rightarrow x,\alpha_k\in\mathbb{R}_+:\lim_{k\rightarrow\infty}\alpha_k(x_k-x)=y$\\
$f\in\mathcal{C}^1(\Omega)$, $\Omega\subseteq\mathbb{R}^n$, lokales Minimum/Maximum $x\in\Omega$. Dann: $D_yf(x)=y^T\nabla f(x)\geq 0\ / \leq 0$, $y\in T(\Omega,x)$\\
$g:\mathbb{R}^n\rightarrow\mathbb{R}^p$, $h:\mathbb{R}^n\rightarrow\mathbb{R}^q$, $g,h\in\mathcal{C}^1$, $\Omega=\{x:g(x)=0,h(x)\geq 0\}$\\
$h(x)\geq 0$ $\Leftrightarrow$ $h_1(x)\geq 0,\dots,h_q(x)\geq 0$.\\
$u\leq v$ $\Leftrightarrow$ $u_j\leq v_j$\\
\ul{Aktive Nebenbedingungen}: $A_k(x)=\{j\ \vert\ k_j(x)=0\}$\\
\ul{Farkas-Lemma}: Für $A\in\mathbb{R}^{k\times l}$ und $b\in\mathbb{R}^k$ gibt es genau dann ein $0\leq x\in\mathbb{R}^l$ mit $Ax=b$ wenn $A^Ty\geq 0$ $\Rightarrow$ $b^Ty\geq 0$\\
$L(x)=\{y\in\mathbb{R}^n\ \vert\ (\nabla g_j(x)y\geq 0 \wedge j=k,\dots,p)\vee(\nabla h_j(x)y\geq 0\wedge j\in A_k(x))\}$\\
$x\in\Omega$ Minimalstelle von $f\in\mathcal{C}^1(\Omega')$, $\Omega'\supseteq\Omega$ offen $\{y\ \vert\ y^Tz\geq 0,z\in Z(\Omega,x)\}=\{y\ \vert\ y^Tz\geq 0,z\in L(x)\}$. Dann existieren $\lambda\in\mathbb{R}^p$, $\mu\in\mathbb{R}_+^q$ mit $f'(x)-\lambda^Tg'(x)-\mu^Th'(x)=0$\\
\ul{Lagrangefunktion}: $F(x)=f(x)-\lambda^Tg(x)-\mu^Th(x)$\\
$\lambda,\mu$ Lagrangemultiplikatoren\\
TODO VL\\

\end{document}






























