\documentclass[a4paper]{article}

\usepackage[ngerman]{babel}
\usepackage[utf8]{inputenc}
\usepackage{amsthm}
\usepackage{amsmath}
\usepackage{amssymb}
\usepackage{tikz,tkz-euclide}
\usepackage{titlesec}
\usepackage{gensymb}
\usepackage{textcomp}
\usepackage[titles]{tocloft}
\usepackage{csquotes}
\usepackage[babel]{microtype}
\usepackage{MnSymbol}
\usepackage{stmaryrd}
\usepackage{mathtools}
\usepackage[
separate-uncertainty = true,
multi-part-units = repeat
]{siunitx}

\usetkzobj{all}
\usetikzlibrary{shapes.misc}

\MakeOuterQuote{"}

\newcommand*\circled[1]{%
  \tikz[baseline=(C.base)]\node[draw,circle,inner sep=0.75pt](C) {#1};\!
}

\renewcommand{\thesubsection}{\arabic{subsection}}
\titleformat{\section}{\normalfont\Large\bfseries}{Kapitel \arabic{section}: }{0em}{}
\titleformat{\subsection}{\normalfont\large\bfseries}{§\arabic{subsection} }{0em}{}
\titleformat{\subsubsection}{\normalfont\bfseries}{\arabic{subsection}.\arabic{subsubsection} }{0em}{}
\renewcommand{\cftsubsecpresnum}{§}
\newlength\mylength
\settowidth\mylength{\cftsubsecpresnum}
\settowidth\mylength{\cftsubsecaftersnum}
\addtolength\cftsubsecnumwidth{\mylength}
\renewcommand{\cftsecpresnum}{Kapitel }
\renewcommand{\cftsecaftersnum}{: }
\settowidth\mylength{\cftsecpresnum}
\addtolength\cftsecnumwidth{\mylength}

\newcommand{\ul}{\underline}
\renewcommand{\qed}{\begin{flushright}
\ul{\(q.e.d.\)}
\end{flushright}}
\let\origphi\phi
\let\phi\varphi

\title{Lineare Algebra II\\Übungsblatt 1}
\author{Nico Mexis}
\date{\today}

\begin{document}
\maketitle
\newpage

\section*{Aufgabe 1)}
\(\phi\) isomorph \(\Leftrightarrow\) \(\phi\) bijektiv \(\Leftrightarrow\) \(det(\phi)\neq 0\) \(\Leftrightarrow\) \(\chi_\phi(0)\neq 0\) \(\Leftrightarrow\) \(\chi_\phi(x)\) besitzt ein Monom vom Grad 0
\section*{Aufgabe 2)}
\(det\begin{pmatrix}
-x & 0 & \hdots & 0 & -a_0 \\
1 & -x & \hdots & 0 & -a_1 \\
0 & 1 & \hdots & 0 & -a_2 \\
\vdots & \vdots & \ddots & \vdots & \vdots \\
0 & 0 & \hdots & 1 & -a_{n-1}-x \\
\end{pmatrix}\)\\
\(=(-1)^na_0\cdot det\begin{pmatrix}
1 & -x & \hdots & 0 \\
0 & 1 & \hdots & 0 \\
\vdots & \vdots & \ddots & \vdots \\
0 & 0 & \hdots & 1 \\
\end{pmatrix}+(-1)^{n-1}a_1\cdot det\begin{pmatrix}
-x & 0 & \hdots & 0 \\
0 & 1 & \hdots & 0 \\
\vdots & \vdots & \ddots & \vdots \\
0 & 0 & \hdots & 1 \\
\end{pmatrix}\)\\
\(+(-1)^{n}a_2\cdot det\begin{pmatrix}
-x & 0 & 0 & \hdots & 0 \\
1 & -x & 0 & \hdots & 0 \\
0 & 0 & 1 & \hdots & 0 \\
\vdots & \vdots & \vdots & \ddots & \vdots \\
0 & 0 & 0 & \hdots & 1 \\
\end{pmatrix}+\dots+(-1)^{n}(a_{n-1}+x)\cdot det\begin{pmatrix}
-x & 0 & \hdots & 0 \\
1 & -x & \hdots & 0 \\
\vdots & \vdots & \ddots & \vdots \\
0 & 0 & \hdots & -x \\
\end{pmatrix}\)\\
\(=(-1)^na_0\cdot 1+(-1)^{n-1}a_1\cdot(-x)+(-1)^na_2\cdot x^2 + \dots + (-1)^n(a_{n-1}+x)x^{n-1}\)\\
\(=(-1)^na_0+(-1)^na_1x+\dots+(-1)^na_{n-1}x^{n-1}+(-1)^nx^n\)\\
\(=(-1)^n\cdot (x^n+a_{n-1}x^{n-1}+\dots+a_1x+a_0)\)
\qed
\section*{Aufgabe 3)}
\(\chi_A(x)\) besitzt Nullstelle 0 \(\Rightarrow\) \(rk(A)=3-1=2\)\\
\(\chi_B(x)\) besitzt keine Nullstelle \(\Rightarrow\) \(det(B)\neq 0\) \(\Rightarrow\) \(B\) invertierbar, \(rk(B)=3\) \(\Rightarrow\) \(Ker(B)=\{0\}\)\\
\(\chi_A(x)=x(-x^2+2x-1)=-x(x-1)^2\)\\
\(0<dim(Eig(A,0))\leq dim(Gen(A,x))=1\)
\end{document}






























