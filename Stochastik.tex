\documentclass[a4paper]{article}

\usepackage[ngerman]{babel}
\usepackage[utf8]{inputenc}
\usepackage{amsthm}
\usepackage{amsmath}
\usepackage{amssymb}
\usepackage{tikz,tkz-euclide}
\usepackage{titlesec}
\usepackage{gensymb}
\usepackage{textcomp}
\usepackage[titles]{tocloft}
\usepackage{csquotes}
\usepackage[babel]{microtype}
\usepackage{MnSymbol}
\usepackage{stmaryrd}
\usepackage{mathtools}
\usepackage{ulem}
\usepackage[shortlabels]{enumitem}
\usepackage{scalerel}
\usepackage{stackengine}
\usepackage{dsfont}
\usepackage[
  separate-uncertainty = true,
  multi-part-units = repeat
]{siunitx}

\usetkzobj{all}
\usetikzlibrary{shapes.misc}

\MakeOuterQuote{"}

\setcounter{secnumdepth}{4}

\renewcommand\hateq{\mathrel{\stackon[1.5pt]{=}{\stretchto{%
				\scalerel*[\widthof{=}]{\wedge}{\rule{1ex}{3ex}}}{0.5ex}}}}

\newcommand*\circled[1]{
  \tikz[baseline=(C.base)]\node[draw,circle,inner sep=0.75pt](C) {#1};\!
}

\newcommand*{\obot}{\perp\mkern-20.7mu\bigcirc}

\DeclarePairedDelimiter\abs{\lvert}{\rvert}
\DeclarePairedDelimiter\norm{\lVert}{\rVert}
\makeatletter
\let\oldabs\abs
\def\abs{\@ifstar{\oldabs}{\oldabs*}}
\let\oldnorm\norm
\def\norm{\@ifstar{\oldnorm}{\oldnorm*}}
\makeatother

\renewcommand{\thesubsection}{\arabic{subsection}}
\titleformat{\section}{\normalfont\Large\bfseries}{Kapitel \arabic{section}: }{0em}{}
\titleformat{\subsection}{\normalfont\large\bfseries}{§\arabic{subsection} }{0em}{}
\titleformat{\subsubsection}{\normalfont\bfseries}{\arabic{subsection}.\arabic{subsubsection} }{0em}{}
\renewcommand{\cftsubsecpresnum}{§}
\newlength\mylength
\settowidth\mylength{\cftsubsecpresnum}
\settowidth\mylength{\cftsubsecaftersnum}
\addtolength\cftsubsecnumwidth{\mylength}
\renewcommand{\cftsecpresnum}{Kapitel }
\renewcommand{\cftsecaftersnum}{: }
\settowidth\mylength{\cftsecpresnum}
\addtolength\cftsecnumwidth{\mylength}

\newcommand{\ul}{\underline}
\renewcommand{\proof}{\ul{Beweis:}\\}
\renewcommand{\qed}{\begin{flushright}
\ul{\(q.e.d.\)}
\end{flushright}}
\let\origphi\phi
\let\phi\varphi
\let\origepsilon\epsilon
\let\epsilon\varepsilon

\title{Stochastik}
\author{Nico Mexis}
\date{\today}

\begin{document}
\maketitle
\newpage

\tableofcontents
\newpage

\section{Wahrscheinlichkeitsräume}
Die Menge aller möglichen Ausgänge/Ergebnisse $\omega$ eines Zufallsexperiments heißt \ul{Ergebnisraum}/\ul{Stichprobenraum}/\ul{Grundmenge} $\Omega$\\
Eine Teilmenge $A\subseteq \Omega$, welcher eine Wahrscheinlichkeit zugeordnet werden soll, heißt \ul{Ereignis}.\\
"A tritt ein", falls $\omega\in A$\\
Für $\omega\in\Omega$ heißt $\{\omega\}\subseteq\Omega$ Elementarereignis.\\
Ereignisraum $\mathcal{A}$: Menge aller Ereignisse in $\Omega$, d.h. $\mathcal{A}\subseteq\mathcal{P}(\Omega)$.\\
Seien $A$, $B$ Ereignisse.\\
\begin{tabular}{ll}
$A\cup B$ & "oder"\\
$A\cap B$ & "und"\\
$A^c=\overline{A}=\Omega\backslash A$ & "nicht"
\end{tabular}\\
Sei $\Omega$ eine Menge. Ein Teilmengensystem $\mathcal{A}\subseteq\mathcal{P}(\Omega)$ heißt \ul{$\sigma$-Algebra} (in $\Omega$), falls
\begin{enumerate}[1)]
	\item $\Omega\in\mathcal{A}$
	\item $A\in\mathcal{A}\Rightarrow A^c\in\mathcal{A}$
	\item $A_1,A_2,\dots\in \mathcal{A}\Rightarrow \bigcup_{i=1}^\infty A_i\in\mathcal{A}$
\end{enumerate}
$(\Omega,\mathcal{A})$ mit $\mathcal{A}$ $\sigma$-Algebra heißt \ul{messbarer Raum}, Elemente $A\in\mathcal{A}$ heißen $\mathcal{A}$-messbare Teilmengen.\\
$\Omega$ beliebig $\Rightarrow$ $\mathcal{A}_1=\{\Omega,\varnothing\}$ ist die \ul{gröbste $\sigma$-Algebra}.\\
$\mathcal{A}_2=\mathcal{P}(\Omega)$ ist die \ul{feinste $\sigma$-Algebra}.\\
In einem höchstens abzählbaren Grundraum $\Omega$ betrachtet man in der Regel die feinste $\sigma$-Algebra $\mathcal{A}=\mathcal{P}(\Omega)$. Für überabzählbares $\Omega$ ist dies i.A. nicht sinnvoll (s. Kapitel 4).\\
Sei $\mathcal{A}$ eine $\sigma$-Algebra. Dann gilt:
\begin{enumerate}[1)]
	\item $\varnothing\in\mathcal{A}$
	\item $A,B\in\mathcal{A}\Rightarrow A\cup B\in\mathcal{A},A\cap B\in\mathcal{A},A\backslash B\in\mathcal{A}$
	\item $A_1,A_2,\dots\in\mathcal{A}\Rightarrow\bigcap_{i=1}^\infty A_i\in\mathcal{A}$
\end{enumerate}
Sei $E\subseteq\mathcal{P}(\Omega)$. Dann gibt es (bzgl. Inklusion) eine \ul{kleinste} $\sigma$-Algebra, welche $E$ enthält.\\
Mit anderen Worten: Es gibt eine sparsamste $\sigma$-Algebra, für die jedes $A\in E$ ein Ereignis ist.\\
Man nennt $\mathcal{A}=\bigcap_{k\in I}\mathcal{A}_k$, wobei $(A_k)_{k\in I}$ das System \ul{aller} $\sigma$-Algebren in $\Omega$ mit $\forall k\in I:E\subseteq A_k$ ist, die \ul{von $E$ erzeugte $\sigma$-Algebra}.\\
Notation: $\mathcal{A}=\sigma(E)$\\
Wir wollen jetzt jedem Ereignis $A\in\mathcal{A}$ eine Wahrscheinlichkeit $\mathbb{P}(A)$ zuordnen.\\
Notation: Mengen $A_1,A_2,\dots\subseteq\Omega$ heißen \ul{paarweise disjunkt} (p.d.), falls $\forall i,j\in\mathbb{N}:i\neq j,A_i\cap A_j=\varnothing$.\\
Sei $\mathcal{A}$ eine $\sigma$-Algebra in $\Omega$ (d.h. $(\Omega,\mathcal{A})$ messbarer Raum). Eine Abbildung $\mathbb{P}:\mathcal{A}\rightarrow\left[0,1\right]$ heißt \ul{Wahrscheinlichkeitsmaß}/\ul{Wahrscheinlichkeitsverteilung} auf $\mathcal{A}$, falls
\begin{enumerate}[1)]
	\item $\mathbb{P}(\Omega)=1$ ($\hateq 100\%$)
	\item $A_1,A_2,\dots\in\mathcal{A}$ p.d.\\
	$\Rightarrow\mathbb{P}(\bigcupdot_{i=1}^\infty A_i)=\sum_{i=1}^\infty\mathbb{P}(A_i)$ ($\sigma$-Additivität)
\end{enumerate}
Es folgen sofort:
\begin{enumerate}[1)]
	\item $\mathbb{P}(\varnothing)=0$
	\item \ul{Additivität}: Falls $A_1,\dots,A_n\in\mathcal{A}$ p.d.\\
	$\Rightarrow\mathbb{P}(\bigcupdot_{i=1}^n A_i)=\sum_{i=1}^n\mathbb{P}(A_i)$
\end{enumerate}
Sei $(\Omega,\mathcal{A})$ ein messbarer Raum und $\mathbb{P}:\mathcal{A}\rightarrow\left[0,1\right]$ ein Wahrscheinlichkeitsmaß auf $\mathcal{A}$. Dann heißt $(\Omega,\mathcal{A},\mathbb{P})$ ein \ul{Wahrscheinlichkeitsraum}.\\
Sei $\Omega\neq\varnothing$ endlich, $\mathcal{A}=\mathcal{P}(\Omega)$, $A\subseteq\Omega$\\
\ul{Gleichverteilung auf $\Omega$}: $\mathbb{P}(A)=\frac{\abs{A}}{\abs{\Omega}}$ (\ul{uniforme Verteilung})\\
\ul{Speziell für $\omega\in\Omega$}: $\mathbb{P}(\{\omega\})=\frac{1}{\abs{\Omega}}$\\
d.h. jeder Ausgang des Experiments ist gleich wahrscheinlich (Laplace-Annahme).\\
$\mathbb{P}$ ist Wahrheitsmaß auf $\mathcal{A}$.\\
$(\Omega,\mathcal{A},\mathbb{P})$ heißt \ul{Laplace-Raum}.\\
\ul{Beachte}: Eine Wahrscheinlichkeit von 0 heißt nicht, dass das Ereignis nicht eintreten kann!\\
\ul{Dirac-Wahrscheinlichkeit}:\\
$\delta_\omega(A):=\mathds{1}_A(\omega)=\begin{cases}
1, & \text{falls } \omega\in A\\
0, & \text{falls } \omega\notin A
\end{cases}$\\
$\delta_\omega:\mathcal{A}\rightarrow\left[0,1\right]$ ist Wahrscheinlichkeitsmaß\\
Seien $A,B\in\mathcal{A}$. Es gilt:
\begin{enumerate}[i)]
	\item $A\subseteq B$ $\Rightarrow$ $\mathbb{P}(B)=\mathbb{P}(A)+\mathbb{P}(B\backslash A)$
	\item $A\subseteq B$ $\Rightarrow$ $\mathbb{P}(A)\leq \mathbb{P}(B)$ (Monotonie)
	\item $\mathbb{P}(A^c)=1-\mathbb{P}(A)$
	\item $\mathbb{P}(A\cup B)=\mathbb{P}(A)+\mathbb{P}(B)-\mathbb{P}(A\cap B)$
\end{enumerate}
Seien $A_1,A_2,\dots\in\mathcal{A}$. Dann gilt:
\begin{enumerate}[i)]
	\item $\mathbb{P}(\bigcup_{i=1}^\infty A_i)\leq\sum_{i\geq 1}\mathbb{P}(A_i)\in\left[0,\infty\right]$ (\ul{$\sigma$-Subadditivität})
	\item Falls $A_1\subseteq A_2\subseteq A_3\subseteq\dots$, dann gilt: $\mathbb{P}(\bigcup_{i=1}^\infty A_i)=\lim_{n\rightarrow\infty}\mathbb{P}(A_n)$ (\ul{$\sigma$-Stetigkeit von unten})
	\item Falls $A_1\supseteq A_2\supseteq A_3\supseteq\dots$, dann gilt: $\mathbb{P}(\bigcap_{i=1}^\infty A_i)=\lim_{n\rightarrow\infty}\mathbb{P}(A_n)$ (\ul{$\sigma$-Stetigkeit von oben})
\end{enumerate}
Ein Ereignis $N\in\mathcal{A}$ mit $\mathbb{P}(N)=0$ heißt \ul{$\mathbb{P}$-Nullmenge}.
\begin{enumerate}[1)]
	\item Sei $N$ eine $\mathbb{P}$-Nullmenge. Dann gilt für $A\in\mathcal{A}$
	\begin{itemize}
		\item $\mathbb{P}(A\cup N)=\mathbb{P}(A)$
		\item $\mathbb{P}(A\backslash N)=\mathbb{P}(A)$
	\end{itemize}
	\item Sei $M\in\mathcal{A}$ mit $\mathbb{P}(M)=1$ $\Rightarrow$ $\mathbb{P}(A)=\mathbb{P}(A\cap M)$
\end{enumerate}
\ul{Siebformeln von Sylvester-Poincaré}:\\
Seien $A_1,\dots,A_n\in\mathcal{A}$
\begin{enumerate}[i)]
	\item $\mathbb{P}(A_1\cup A_2\cup\dots\cup A_n)=-\sum_{j=1}^n(-1)^j\sum_{F\subseteq\{1,\dots,n\},\abs{F}=j}\mathbb{P}(\bigcap_{k\in F}A_k)$
	\item $\mathbb{P}(A_1\cap A_2\cap\dots\cap A_n)=-\sum_{j=1}^n(-1)^j\sum_{F\subseteq\{1,\dots,n\},\abs{F}=j}\mathbb{P}(\bigcup_{k\in F}A_k)$
\end{enumerate}
Sei $\Omega$ endlich oder abzählbar, sei $f:\Omega\rightarrow\left[0,1\right]$ mit $\sum_{\omega\in\Omega}f(\omega)=1$.\\
Dann gibt es \ul{genau ein} Wahrscheinlichkeitsmaß auf $\mathcal{P}(\Omega)$ mit $\mathbb{P}(\{\omega\})=f(\omega)$ $\forall\omega\in\Omega$\\
Es gilt dann: $\mathbb{P}(A)=\sum_{\omega\in A}f(\omega)$ $\forall A\subseteq\Omega$, $f$ heißt \ul{Zähldichte}
\section{Bedingte Wahrscheinlichkeiten \& Stochastische Unabhängigkeit}
Seien $A,B\in\mathcal{A}$ mit $\mathbb{P}(B)>0$.\\
\ul{Bedingte Wahrscheinlichkeit von $A$ gegeben $B$}: $$\mathbb{P}(A\vert B):=\frac{\mathbb{P}(A\cap B)}{\mathbb{P}(B)}$$
\ul{Interpretation}: $\mathbb{P}(A\vert B)$ ist der relative Anteil von $A$ in $B$\\
Sei $(\Omega,\mathcal{A},\mathbb{P})$ Wahrscheinlichkeitsraum und $B\in\mathcal{A}$ mit $\mathbb{P}(B)>0$. Dann ist $(\Omega,\mathcal{A},\mathbb{P}(\cdot\vert B))$ ein Wahrscheinlichkeitsraum mit $\mathbb{P}(B\vert B)=1$.\\
\begin{enumerate}[1)]
	\item $\forall A\in\mathcal{A}$ mit $B\subseteq A$ gilt $\mathbb{P}(A\vert B)=\frac{\mathbb{P}(A\cap B)}{\mathbb{P}(B)}=\frac{\mathbb{P}(B)}{\mathbb{P}(B)}=1$
	\item $\mathbb{P}(A\cap B)=\mathbb{P}(B)\cdot\mathbb{P}(A\vert B)=\mathbb{P}(A)\mathbb{P}(B\vert A)$\\
	In den meisten Fällen sind \ul{bedingte} Wahrscheinlichkeiten bekannt und man will die Wahrscheinlichkeiten von Durchschnitten berechnen.
	\item Bei Berechnung von $\mathbb{P}(A\vert B)$ kann vorausgesetzt werden, dass das Ereignis $B$ eingetreten ist.
\end{enumerate}
\end{document}





























