\documentclass[a4paper]{article}

\usepackage[ngerman]{babel}
\usepackage[utf8]{inputenc}
\usepackage{amsthm}
\usepackage{amsmath}
\usepackage{amssymb}
\usepackage{tikz,tkz-euclide}
\usepackage{titlesec}
\usepackage{gensymb}
\usepackage{textcomp}
\usepackage[titles]{tocloft}
\usepackage{csquotes}
\usepackage[babel]{microtype}
\usepackage{MnSymbol}
\usepackage{stmaryrd}
\usepackage{mathtools}
\usepackage{ulem}
\usepackage[shortlabels]{enumitem}
\usepackage{scalerel}
\usepackage{stackengine}
\usepackage{dsfont}
\usepackage[
  separate-uncertainty = true,
  multi-part-units = repeat
]{siunitx}

\usetkzobj{all}
\usetikzlibrary{shapes.misc}

\MakeOuterQuote{"}

\setcounter{secnumdepth}{4}

\renewcommand\hateq{\mathrel{\stackon[1.5pt]{=}{\stretchto{%
				\scalerel*[\widthof{=}]{\wedge}{\rule{1ex}{3ex}}}{0.5ex}}}}

\newcommand*\circled[1]{
  \tikz[baseline=(C.base)]\node[draw,circle,inner sep=0.75pt](C) {#1};\!
}

\newcommand*{\obot}{\perp\mkern-20.7mu\bigcirc}

\DeclarePairedDelimiter\abs{\lvert}{\rvert}
\DeclarePairedDelimiter\norm{\lVert}{\rVert}
\makeatletter
\let\oldabs\abs
\def\abs{\@ifstar{\oldabs}{\oldabs*}}
\let\oldnorm\norm
\def\norm{\@ifstar{\oldnorm}{\oldnorm*}}
\makeatother

\renewcommand{\thesubsection}{\arabic{subsection}}
\titleformat{\section}{\normalfont\Large\bfseries}{Kapitel \arabic{section}: }{0em}{}
\titleformat{\subsection}{\normalfont\large\bfseries}{§\arabic{subsection} }{0em}{}
\titleformat{\subsubsection}{\normalfont\bfseries}{\arabic{subsection}.\arabic{subsubsection} }{0em}{}
\renewcommand{\cftsubsecpresnum}{§}
\newlength\mylength
\settowidth\mylength{\cftsubsecpresnum}
\settowidth\mylength{\cftsubsecaftersnum}
\addtolength\cftsubsecnumwidth{\mylength}
\renewcommand{\cftsecpresnum}{Kapitel }
\renewcommand{\cftsecaftersnum}{: }
\settowidth\mylength{\cftsecpresnum}
\addtolength\cftsecnumwidth{\mylength}

\newcommand{\ul}{\underline}
\renewcommand{\proof}{\ul{Beweis:}\\}
\renewcommand{\qed}{\begin{flushright}
\ul{\(q.e.d.\)}
\end{flushright}}
\let\origphi\phi
\let\phi\varphi
\let\origepsilon\epsilon
\let\epsilon\varepsilon

\title{Stochastik}
\author{Nico Mexis}
\date{\today}

\begin{document}
\maketitle
\newpage

\tableofcontents
\newpage

\section{Wahrscheinlichkeitsräume}
Die Menge aller möglichen Ausgänge/Ergebnisse $\omega$ eines Zufallsexperiments heißt \ul{Ergebnisraum}/\ul{Stichprobenraum}/\ul{Grundmenge} $\Omega$\\
Eine Teilmenge $A\subseteq \Omega$, welcher eine Wahrscheinlichkeit zugeordnet werden soll, heißt \ul{Ereignis}.\\
"A tritt ein", falls $\omega\in A$\\
Für $\omega\in\Omega$ heißt $\{\omega\}\subseteq\Omega$ Elementarereignis.\\
Ereignisraum $\mathcal{A}$: Menge aller Ereignisse in $\Omega$, d.h. $\mathcal{A}\subseteq\mathcal{P}(\Omega)$.\\
Seien $A$, $B$ Ereignisse.\\
\begin{tabular}{ll}
$A\cup B$ & "oder"\\
$A\cap B$ & "und"\\
$A^c=\overline{A}=\Omega\backslash A$ & "nicht"
\end{tabular}\\
Sei $\Omega$ eine Menge. Ein Teilmengensystem $\mathcal{A}\subseteq\mathcal{P}(\Omega)$ heißt \ul{$\sigma$-Algebra} (in $\Omega$), falls
\begin{enumerate}[1)]
	\item $\Omega\in\mathcal{A}$
	\item $A\in\mathcal{A}\Rightarrow A^c\in\mathcal{A}$
	\item $A_1,A_2,\dots\in \mathcal{A}\Rightarrow \bigcup_{i=1}^\infty A_i\in\mathcal{A}$
\end{enumerate}
$(\Omega,\mathcal{A})$ mit $\mathcal{A}$ $\sigma$-Algebra heißt \ul{messbarer Raum}, Elemente $A\in\mathcal{A}$ heißen $\mathcal{A}$-messbare Teilmengen.\\
$\Omega$ beliebig $\Rightarrow$ $\mathcal{A}_1=\{\Omega,\varnothing\}$ ist die \ul{gröbste $\sigma$-Algebra}.\\
$\mathcal{A}_2=\mathcal{P}(\Omega)$ ist die \ul{feinste $\sigma$-Algebra}.\\
In einem höchstens abzählbaren Grundraum $\Omega$ betrachtet man in der Regel die feinste $\sigma$-Algebra $\mathcal{A}=\mathcal{P}(\Omega)$. Für überabzählbares $\Omega$ ist dies i.A. nicht sinnvoll (s. Kapitel 4).\\
Sei $\mathcal{A}$ eine $\sigma$-Algebra. Dann gilt:
\begin{enumerate}[1)]
	\item $\varnothing\in\mathcal{A}$
	\item $A,B\in\mathcal{A}\Rightarrow A\cup B\in\mathcal{A},A\cap B\in\mathcal{A},A\backslash B\in\mathcal{A}$
	\item $A_1,A_2,\dots\in\mathcal{A}\Rightarrow\bigcap_{i=1}^\infty A_i\in\mathcal{A}$
\end{enumerate}
Sei $E\subseteq\mathcal{P}(\Omega)$. Dann gibt es (bzgl. Inklusion) eine \ul{kleinste} $\sigma$-Algebra, welche $E$ enthält.\\
Mit anderen Worten: Es gibt eine sparsamste $\sigma$-Algebra, für die jedes $A\in E$ ein Ereignis ist.\\
Man nennt $\mathcal{A}=\bigcap_{k\in I}\mathcal{A}_k$, wobei $(A_k)_{k\in I}$ das System \ul{aller} $\sigma$-Algebren in $\Omega$ mit $\forall k\in I:E\subseteq A_k$ ist, die \ul{von $E$ erzeugte $\sigma$-Algebra}.\\
Notation: $\mathcal{A}=\sigma(E)$\\
Wir wollen jetzt jedem Ereignis $A\in\mathcal{A}$ eine Wahrscheinlichkeit $\mathbb{P}(A)$ zuordnen.\\
Notation: Mengen $A_1,A_2,\dots\subseteq\Omega$ heißen \ul{paarweise disjunkt} (p.d.), falls $\forall i,j\in\mathbb{N}:i\neq j,A_i\cap A_j=\varnothing$.\\
Sei $\mathcal{A}$ eine $\sigma$-Algebra in $\Omega$ (d.h. $(\Omega,\mathcal{A})$ messbarer Raum). Eine Abbildung $\mathbb{P}:\mathcal{A}\rightarrow\left[0,1\right]$ heißt \ul{Wahrscheinlichkeitsmaß}/\ul{Wahrscheinlichkeitsverteilung} auf $\mathcal{A}$, falls
\begin{enumerate}[1)]
	\item $\mathbb{P}(\Omega)=1$ ($\hateq 100\%$)
	\item $A_1,A_2,\dots\in\mathcal{A}$ p.d.\\
	$\Rightarrow\mathbb{P}(\bigcupdot_{i=1}^\infty A_i)=\sum_{i=1}^\infty\mathbb{P}(A_i)$ ($\sigma$-Additivität)
\end{enumerate}
Es folgen sofort:
\begin{enumerate}[1)]
	\item $\mathbb{P}(\varnothing)=0$
	\item \ul{Additivität}: Falls $A_1,\dots,A_n\in\mathcal{A}$ p.d.\\
	$\Rightarrow\mathbb{P}(\bigcupdot_{i=1}^n A_i)=\sum_{i=1}^n\mathbb{P}(A_i)$
\end{enumerate}
Sei $(\Omega,\mathcal{A})$ ein messbarer Raum und $\mathbb{P}:\mathcal{A}\rightarrow\left[0,1\right]$ ein Wahrscheinlichkeitsmaß auf $\mathcal{A}$. Dann heißt $(\Omega,\mathcal{A},\mathbb{P})$ ein \ul{Wahrscheinlichkeitsraum}.\\
Sei $\Omega\neq\varnothing$ endlich, $\mathcal{A}=\mathcal{P}(\Omega)$, $A\subseteq\Omega$\\
\ul{Gleichverteilung auf $\Omega$}: $\mathbb{P}(A)=\frac{\abs{A}}{\abs{\Omega}}$ (\ul{uniforme Verteilung})\\
\ul{Speziell für $\omega\in\Omega$}: $\mathbb{P}(\{\omega\})=\frac{1}{\abs{\Omega}}$\\
d.h. jeder Ausgang des Experiments ist gleich wahrscheinlich (Laplace-Annahme).\\
$\mathbb{P}$ ist Wahrheitsmaß auf $\mathcal{A}$.\\
$(\Omega,\mathcal{A},\mathbb{P})$ heißt \ul{Laplace-Raum}.\\
\ul{Beachte}: Eine Wahrscheinlichkeit von 0 heißt nicht, dass das Ereignis nicht eintreten kann!\\
\ul{Dirac-Wahrscheinlichkeit}:\\
$\delta_\omega(A):=\mathds{1}_A(\omega)=\begin{cases}
1, & \text{falls } \omega\in A\\
0, & \text{falls } \omega\notin A
\end{cases}$\\
$\delta_\omega:\mathcal{A}\rightarrow\left[0,1\right]$ ist Wahrscheinlichkeitsmaß\\
Seien $A,B\in\mathcal{A}$. Es gilt:
\begin{enumerate}[i)]
	\item $A\subseteq B$ $\Rightarrow$ $\mathbb{P}(B)=\mathbb{P}(A)+\mathbb{P}(B\backslash A)$
	\item $A\subseteq B$ $\Rightarrow$ $\mathbb{P}(A)\leq \mathbb{P}(B)$ (Monotonie)
	\item $\mathbb{P}(A^c)=1-\mathbb{P}(A)$
	\item $\mathbb{P}(A\cup B)=\mathbb{P}(A)+\mathbb{P}(B)-\mathbb{P}(A\cap B)$
\end{enumerate}
Seien $A_1,A_2,\dots\in\mathcal{A}$. Dann gilt:
\begin{enumerate}[i)]
	\item $\mathbb{P}(\bigcup_{i=1}^\infty A_i)\leq\sum_{i\geq 1}\mathbb{P}(A_i)\in\left[0,\infty\right]$ (\ul{$\sigma$-Subadditivität})
	\item Falls $A_1\subseteq A_2\subseteq A_3\subseteq\dots$, dann gilt: $\mathbb{P}(\bigcup_{i=1}^\infty A_i)=\lim_{n\rightarrow\infty}\mathbb{P}(A_n)$ (\ul{$\sigma$-Stetigkeit von unten})
	\item Falls $A_1\supseteq A_2\supseteq A_3\supseteq\dots$, dann gilt: $\mathbb{P}(\bigcap_{i=1}^\infty A_i)=\lim_{n\rightarrow\infty}\mathbb{P}(A_n)$ (\ul{$\sigma$-Stetigkeit von oben})
\end{enumerate}
Ein Ereignis $N\in\mathcal{A}$ mit $\mathbb{P}(N)=0$ heißt \ul{$\mathbb{P}$-Nullmenge}.
\begin{enumerate}[1)]
	\item Sei $N$ eine $\mathbb{P}$-Nullmenge. Dann gilt für $A\in\mathcal{A}$
	\begin{itemize}
		\item $\mathbb{P}(A\cup N)=\mathbb{P}(A)$
		\item $\mathbb{P}(A\backslash N)=\mathbb{P}(A)$
	\end{itemize}
	\item Sei $M\in\mathcal{A}$ mit $\mathbb{P}(M)=1$ $\Rightarrow$ $\mathbb{P}(A)=\mathbb{P}(A\cap M)$
\end{enumerate}
\ul{Siebformeln von Sylvester-Poincaré}:\\
Seien $A_1,\dots,A_n\in\mathcal{A}$
\begin{enumerate}[i)]
	\item $\mathbb{P}(A_1\cup A_2\cup\dots\cup A_n)=-\sum_{j=1}^n(-1)^j\sum_{F\subseteq\{1,\dots,n\},\abs{F}=j}\mathbb{P}(\bigcap_{k\in F}A_k)$
	\item $\mathbb{P}(A_1\cap A_2\cap\dots\cap A_n)=-\sum_{j=1}^n(-1)^j\sum_{F\subseteq\{1,\dots,n\},\abs{F}=j}\mathbb{P}(\bigcup_{k\in F}A_k)$
\end{enumerate}
Sei $\Omega$ endlich oder abzählbar, sei $f:\Omega\rightarrow\left[0,1\right]$ mit $\sum_{\omega\in\Omega}f(\omega)=1$.\\
Dann gibt es \ul{genau ein} Wahrscheinlichkeitsmaß auf $\mathcal{P}(\Omega)$ mit $\mathbb{P}(\{\omega\})=f(\omega)$ $\forall\omega\in\Omega$\\
Es gilt dann: $\mathbb{P}(A)=\sum_{\omega\in A}f(\omega)$ $\forall A\subseteq\Omega$, $f$ heißt \ul{Zähldichte}
\section{Bedingte Wahrscheinlichkeiten \& Stochastische Unabhängigkeit}
Seien $A,B\in\mathcal{A}$ mit $\mathbb{P}(B)>0$.\\
\ul{Bedingte Wahrscheinlichkeit} von $A$ gegeben $B$: $$\mathbb{P}(A\vert B):=\frac{\mathbb{P}(A\cap B)}{\mathbb{P}(B)}$$
\ul{Interpretation}: $\mathbb{P}(A\vert B)$ ist der relative Anteil von $A$ in $B$\\
Sei $(\Omega,\mathcal{A},\mathbb{P})$ Wahrscheinlichkeitsraum und $B\in\mathcal{A}$ mit $\mathbb{P}(B)>0$. Dann ist $(\Omega,\mathcal{A},\mathbb{P}(\cdot\vert B))$ ein Wahrscheinlichkeitsraum mit $\mathbb{P}(B\vert B)=1$.
\begin{enumerate}[1)]
	\item $\forall A\in\mathcal{A}$ mit $B\subseteq A$ gilt $\mathbb{P}(A\vert B)=\frac{\mathbb{P}(A\cap B)}{\mathbb{P}(B)}=\frac{\mathbb{P}(B)}{\mathbb{P}(B)}=1$
	\item $\mathbb{P}(A\cap B)=\mathbb{P}(B)\cdot\mathbb{P}(A\vert B)=\mathbb{P}(A)\mathbb{P}(B\vert A)$\\
	In den meisten Fällen sind \ul{bedingte} Wahrscheinlichkeiten bekannt und man will die Wahrscheinlichkeiten von Durchschnitten berechnen.
	\item Bei Berechnung von $\mathbb{P}(A\vert B)$ kann vorausgesetzt werden, dass das Ereignis $B$ eingetreten ist.
\end{enumerate}
TODO VL
Sei $(\Omega,\mathcal{A},\mathbb{P})$ ein Wahrscheinlichkeitsraum.\\
Bei einem Zufallsexperiment ist oft nur ein gewisser Aspekt (Eigenschaft) von $\omega\in\Omega$ von Interesse.\\
Sei $M$ eine Menge, $U\subseteq M$ $\Rightarrow$ \ul{Indikatorfunktion von $U$ (auf $M$)}:
$$\mathds{1}_U:M\rightarrow\mathbb{R}:\mathds{1}_U(x)=\begin{cases}
1, & \text{falls }x\in U\\
0, & \text{falls }x\notin U
\end{cases}$$
Menge aller reellen Intervalle $L:=\{I\subseteq\mathbb{R}\ \vert\ I\text{ Intervall}\}$, also $(a,b),\left(a,b\right],\dots\in J$\\
Sei $(\Omega,\mathcal{A},\mathbb{P})$ ein Wahrscheinlichkeitsraum.\\
Eine Abbildung $X:\Omega\rightarrow\mathbb{R}$ heißt \ul{(reellwertige) Zufallsvariable} (ZV) auf $(\Omega,\mathcal{A},\mathbb{P})$, falls $\forall I\in J:\{\omega\in\Omega\ \vert\ X(\omega)\in I\}\in\mathcal{A}$\\
Eine $\mathbb{R}^d$-wertige Abbildung $X=(X_1,\dots,X_d):\Omega\rightarrow\mathbb{R}^d$ heißt \ul{$d$-dimensionaler Zufallsvektor} (auf $(\Omega,\mathcal{A},\mathbb{P})$), falls alle Komponenten $X_i:\Omega\rightarrow\mathbb{R}$ ZVen sind.\\
\begin{enumerate}[i)]
	\item Zufallsvariablen/Zufallsvektoren sind Abbildungen
	\item Für Zufallsvariablen sind die Wahrscheinlichkeiten $\mathbb{P}(\{\omega\in\Omega\ \vert\ X(\omega)\in I\})$ und $\mathbb{P}(\{\omega\in\Omega\ \vert\ X(\omega)\notin I\})$ wohldefiniert
	\item Falls $A=\mathcal{P}(\Omega)$ ist \ul{jede} Abbildung $\Omega\rightarrow\mathbb{R}$ eine ZV und für alle $B\subseteq\mathbb{R}$ ist $\mathbb{P}(\{\omega\in\Omega\ \vert\ X(\omega)\in B\})$ wohldefiniert
	\item Allgemeine Definition einer ZV: Sei $(\Omega',\mathcal{A}')$ messbarer Raum. Eine Abbildung $X:\Omega\rightarrow\Omega'$ ist eine ZV, falls $\forall A'\in \mathcal{A}':\{\omega\in\Omega\ \vert\ X(\omega)\in A'\}\in \mathcal{A}$
\end{enumerate}
Schreibweisen für Urbilder:\\
$\{X\in B\}:=\{\omega\in\Omega\ \vert\ X(\omega)\in B\}=X^{-1}(B)=:\left[X\in B\right]$ für $X:\Omega\rightarrow\mathbb{R}^d$, $B\subseteq\mathbb{R}^d$, $d\geq 1$\\
\ul{Fall $d=1$}: $x\in\mathbb{R}$\\
$\{X\leq x\}=\{\omega\in\Omega\ \vert\ X(\omega)\in B\}=X^{-1}(\left(-\infty,x\right])$\\
TODO Notation fertig\\
Sei $(\Omega,\mathcal{A})$ messbarer Raum, $\Omega\neq\varnothing$ und $X:\Omega\rightarrow\mathbb{R}$ Abbildung. Dann sind folgende Aussagen äquivalent:
\begin{enumerate}[(i)]
	\item $\forall x\in\mathbb{R}:\{X\leq x\}\in\mathcal{A}$
	\item $\forall I\in J:\{X\in I\}\in\mathcal{A}$ (d.h. $X$ ist ZV)
\end{enumerate}
Im Folgenden: $I=\{1,\dots,n\}$ oder $I=\mathbb{N}$\\
Seien $(\Omega_1,\mathcal{A}_1,\mathbb{P}_1)$ und $(\Omega_2,\mathcal{A}_2,\mathbb{P}_2)$ Wahrscheinlichkeitsräume, $X_i:\Omega_i\rightarrow\mathbb{R}$, $i=1,2$, ZVen\\
Dann heißen $X_1$ und $X_2$ \ul{identisch verteilt}, falls $\forall I\in J:\mathbb{P}_1(\{X_1\in I\})=\mathbb{P}_2(\{X_2\in I\})$\\
Eine Folge $(X_i)_{i\in I}$, $X_i$ ZV auf Wahrscheinlichkeitsraum $(\Omega_i,\mathcal{A}_i,\mathbb{P}_i)$, von ZVen heißt \ul{identisch verteilt}, falls für alle $i,j\in I$ die ZVen $X_i$ und $X_j$ identisch verteilt sind.\\
\ul{Schreibweise}: $X_1,X_2,\dots$ identisch verteilt\\
Seien $(\Omega_1,\mathcal{A}_1,\mathbb{P}_1)$, $(\Omega_2,\mathcal{A}_2,\mathbb{P}_2)$ W'räume, $X_i:\Omega_1\rightarrow\mathbb{R}$, $i=1,2$, ZVen $\Rightarrow$ Äquivalent:
\begin{enumerate}[(i)]
	\item $\forall x\in\mathbb{R}:\mathbb{P}_1(\{X_1\leq x\})=\mathbb{P}_2(\{X_2\leq x\})$
	\item $\forall I\in J:\mathbb{P}_1(\{X_1\in I\})=\mathbb{P}_2(\{X_2\in I\})$
\end{enumerate}
Sei $X:\Omega\rightarrow\mathbb{R}$ ZV auf $(\Omega,\mathcal{A},\mathbb{P})$. Die Abbildung $F_X:\mathbb{R}\rightarrow\left[0,1\right]:x\mapsto F_X(x):=\mathbb{P}(\{X\leq x\})$ heißt \ul{Verteilungsfunktion} von $X$.\\
Die ZVen $X_1$, $X_2$ identisch verteilt $\Leftrightarrow$ $F_{X_1}=F_{X_2}$\\
\ul{Verteilungsfunktion einer diskreten ZV}:\\
Sei $I=\{1,\dots,n\}$, $n\in\mathbb{N}$ oder $I=\mathbb{N}$, sei $(a_i)_{i\in I}$ Folge in $\mathbb{R}$, paarweise verschieden. Sei $\Omega=\{a_i\ \vert\ i\in I\}\subseteq\mathbb{R}$, $f:I\rightarrow\left[0,1\right]$ Abbildung mit $\sum_{i\in I}f(a_i)=1$, ZV: $X:\Omega\rightarrow\Omega$, $\omega\mapsto X(\omega):=\omega$ $\Rightarrow$ $F_X(x)=\sum_{i\in I,a_i\leq x}f(x)$, $\mathbb{P}(\{X=a_i\})=f(a_i)$\\
TODO Satz 3.19\\
Es gilt für alle $x\in\mathbb{R}$: $F_X$ stetig in $x$ $\Leftrightarrow$ $\mathbb{P}\left[X=x\right]=0$\\
TODO Def 3.22\\
Seien $X,y:\Omega_1\rightarrow\Omega_2$ ZVen. Dann sind $X$ und $X$ unabhängig, falls für alle $A,B\in\mathcal{A_2}$ gilt: $$\mathbb{P}\left[X\in A,Y\in B\right]=\mathbb{P}\left[X\in A\right]\cdot\mathbb{P}\left[Y\in B\right]$$\\
Sei $(X_i)_{i\in I}$ eine Folge von ZVen. Dann ist $(X_i)_{i\in I}$ genau dann unabhängig, wenn $$\mathbb{P}\left(\bigcap_{i\in\tilde I}\{X_i\leq x_i\}\right)=\prod_{i\in\tilde I}\mathbb{P}\left[X_i\leq x_i\right]$$ für alle endlichen Mengen $\varnothing\neq \tilde I\subseteq I$ und alle Folgen $(x_i)_{i\in I}$ in $\mathbb{R}$ gilt.
\section{Diskrete Modelle}
Ein Wahrscheinlichkeitsraum $(\Omega,\mathcal{A},\mathbb{P})$ heißt \ul{diskret}, falls $\Omega$ höchstens abzählbar ist und $\mathcal{A}=\mathcal{P}(\Omega)$.\\
TODO Kapitel 3 Folien\\
Laplace-Raum $(\Omega,\mathcal{A},\mathbb{P})$ mit $\Omega$ endlich, $\mathcal{A}=\mathcal{P}(\Omega)$ und $\mathbb{P}(A)=\frac{\abs{A}}{\abs{\Omega}}$ $\forall A\subset\Omega$\\
\ul{Binomialkoeffizient}: Sei $n\in\mathbb{N}_0$, $k\in\{0,\dots,n\}$.\\
$\binom{n}{k}:=\frac{n!}{k!(n-k)!}$ "$k$ aus $n$"\\
Es gilt für $k\neq 0$: $\binom{n}{k-1}+\binom{n}{k}=\binom{n+1}{k}$\\
Sei $k\in\mathbb{N}$ und $N,N_1,\dots,N_k$ endliche, nicht-leere Mengen. Dann gilt $$\abs{N_1\times\dots\times N_k}=\abs{N_1}\cdot\abs{N_2}\cdots\dots\cdot\abs{N_k}$$
Insbesondere: $\abs{N^k}=\abs{N}^k$ ($k$-Variation mit Wiederholung)\\
Sei $N$ eine nicht-leere, endliche Menge mit $\abs{N}=n\in\mathbb{N}$. Dann gilt für alle $k\in\{1,\dots,n\}$: $Q_k(N):=\abs{\{(w_1,\dots,w_k)\in N^k\ \vert\ w_1,\dots,w_k \text{ p.w. verschieden}\}}=\frac{n!}{(n-k)!}=n\cdot(n-1)\cdot\dots(n-k+1)$ ($k$-Variation ohne Wiederholung)\\
Insbesondere für $k=n$: Anzahl der Permutationen der Elemente von $N$ beträgt $n!$\\
Sei $N$ nicht-leere, endliche Menge mit $n=\abs{N}$. Für $k\in\{0,\dots,n\}$ gilt: $\abs{\{K\subseteq N\\ \vert\ \abs{K}=k\}}=\binom{n}{K}$\\
Es gilt für $n\in\mathbb{N}$ und $k\in\{1,\dots,n\}$: $$\abs{\{(w_1,\dots,w_k)\in\{1,\dots,n\}^k\ \vert\ 1\leq w_1< w_2<\dots<w_k\leq n\}}=\binom{n}{k}$$
Für $k,n\in\mathbb{N}$ gilt: $$\abs{\{(w_1,\dots,w_k)\in\mathbb{N}^k\ \vert\ 1\leq w_1\leq \dots\leq w_k\leq n\}}=\binom{n+k-1}{k}$$
Es gilt: $$\abs{\{(w_1,\dots,w_k)\in\mathbb{N}_0^k\ \vert\ w_1+\dots+w_k=n\}}=\binom{n+k-1}{n}$$
Seien $(\Omega_1,\mathcal{A}_1,\mathbb{P}_1),\dots,(\Omega_n,\mathcal{A}_N,\mathbb{P}_n)$ W'räume.\\
$\Omega = \Omega_1\times\Omega_2\times\dots\times\Omega_n$, $\mathcal{A}=\mathcal{P}(\Omega)$, $\forall\omega=\{\omega_1,\dots,\omega_n\}\in\Omega:\mathbb{P}(\{\omega\})=\mathbb{P}_1(\{\omega_1\})\cdot\dots\cdot\mathbb{P}_n(\{\omega_n\})$, bzw. $f(\omega)=f_1(\omega_1)\cdot\dots\cdots f_n(\omega_n)$ mit $f_i:\Omega\rightarrow\left[0,1\right]$ W'funktion zu $\mathbb{P}$.
\begin{enumerate}[i)]
	\item $f$ ist eine W'funktion auf $\Omega$
	\item Für das durch $f$ definierte W'maß $\mathbb{P}$ auf $\mathcal{A}$ gilt: $\forall A_1\subseteq\Omega_1,\dots,A_n\subseteq\Omega_n:\mathbb{P}(A_1\times\dots\times A_n)=\mathbb{P}_1(A_1)\cdot\times\cdot\mathbb{P}_n(A_n)$
	\item Es existiert genau ein W'maß auf $\mathcal{A}$ mit Eigenschaft ii)
\end{enumerate}
Das durch $f$ definierte W'maß $\mathbb{P}$ auf $\mathcal{A}$ heißt das \ul{Produkt der W'maße $\mathbb{P}_i$} (Produktmaß) und $(\Omega,\mathcal{A},\mathbb{P})$ heißt das \ul{Produkt der W'räume} $(\Omega_i,\mathcal{A}_i,\mathbb{P}_i)$ (\ul{Produktraum})\\
Seien $A_1\subseteq\Omega_1,\dots,A_n\subseteq\Omega_n$. Dann gilt: $\mathbb{P}(\bigcap_{i=1}^n\{X_i\in A_i\})=\prod_{i=1}^n\mathbb{P}(\{X_i\in A_i\})=\prod_{i=1}^n\mathbb{P}_i(A_i)$\\
Wir haben folgendes mit dieser Konstruktion erreicht:
\begin{enumerate}[1)]
	\item Produktraum-Modell beinhaltet die Modelle der Einzelexperimente durch Projektionen. Darstellung von $A_i\in\mathcal{A}_i$ durch $\{X_i\in A_i\}\in\mathcal{A}$ und $\mathbb{P}(\{X_i\in A_i\})=\mathbb{P}_i(A_i)$
	\item Die Ereignisse $\{X_1\in A_1\},\dots,\{X-n\in A_n\}$ sind unabhängig für alle $A_i\subseteq\Omega_i$, $i\in\{1,\dots,n\}$\\
	$\Rightarrow$ Falls $\Omega_1,\dots,\Omega_n\subseteq\mathbb{R}$: $X_1,\dots,X_n$ sind unabhängige ZVen!
	\item Falls zusätzlich $\mathbb{P}_1=\dots=\mathbb{P}_n$: $X_1,\dots,X_n$ i.i.d.
\end{enumerate}
Im Folgenden sind $,X_1,X_2,\dots$ ZVen auf einem W'raum $(\Omega,\mathcal{A},\mathbb{P})$\\
Eine ZV $X:\Omega\rightarrow\mathbb{R}$ heißt \ul{diskret}, falls es eine höchstens abzählbare Menge $D\subseteq\mathbb{R}$ gibt mit $\mathbb{P}(\{X\in D\})=1$.\\
ist $D\subseteq\mathbb{R}$ höchstens abzählbar, so gilt $\{X\in D\}=\bigcup_{\omega\in D}\{X=\omega\}\in\mathcal{A}\Rightarrow\mathbb{P}(\{X\in D\})$\\
$(\Omega,\mathcal{A},\mathbb{P})$ diskret\\
$\Rightarrow$ $X(\Omega)$ höchstens abzählbar\\
$\Rightarrow$ $X$ diskret (mit $D=X(\Omega)$)\\
TODO VL\\
Seien $X_1,\dots,X_n$ ZVen i.i.d. mit $X_i\sim B(1,p),p\in\left[0,1\right]$ und $X=\sum_{i=1}^n X_i$. Dann gilt für $k\in\{0,\dots,n\}$: $\mathbb{P}(\{X=k\})=\binom{n}{k}\cdot p^k(1-p)^{n-k}$\\
Eine Zufallsvariable $X:\Omega\rightarrow\{0,\dots,n\}$ heißt "binomialverteilt" mit Parameter $n\in\mathbb{N}$ und $p\in\left[0,1\right]$, falls für alle $k\in\{0,\dots,n\}$ gilt: $\mathbb{P}(\{X=k\})=\binom{n}{k}p^k(1-p)^{n-k}$\\
\ul{Notation}: $X\sim B(n,p)$\\
$D_X=\begin{cases}
\{0,\dots,n\} & \text{falls }p\in(0,1)\\
\{0\} & \text{falls }p=0\\
\{n\} & \text{falls }p=1
\end{cases}$\\
$\mathbb{P}(\{X\in\{0,n\}\})=\sum_{k=0}^n\mathbb{P}(\{X=k\})=\sum_{k=0}^n\binom{n}{k}p^k(1-p)^{n-k}=(p+(1-p))^n=1^n=1$\\
Eine Zufallsvariable $X$ heißt \ul{hypergeometrisch-verteilt} mit Parametern $N\in\mathbb{N}$, $N_0\in\{0,\dots,N\}$, $n\in\{1,\dots,N\}$, falls $\mathbb{P}(\{X=l\})=\frac{\binom{N_0}{l}\binom{N-N_0}{n-l}}{\binom{N}{n}}$ für $l\in\{\max\{0,n-(N-N_0)\},\dots,\min\{N_0,n\}\}$\\
\ul{Notation}: $X\sim H(N,N_0,n)$, $D_X=\{\max\{0,n-(N-N_0)\},\dots,\min\{N_0,n\}\}$\\
Sei $n\in\mathbb{N}, (N_0(N))_{N\in\mathbb{N}}$ Folge in $\mathbb{N}$ mit $N_0(N)\leq N$ und $\lim_{N\rightarrow\infty}\frac{N_0(N)}{N}=p\in(0,1)$\\
Sei $(X_N)_{N\geq n}$ Folge von Zufallsvariablen mit $X_N\sim H(N,N_0(N),n)$\\
Dann gilt $\forall k\in\{0,\dots,n\}:\lim_{N\rightarrow\infty}\mathbb{P}(\{X_N=k\})=\binom{n}{k}p^k(1-p)^{n-k}$\\
\ul{Interpretation}: Die Binomialverteilung ist gewissermaßen die Grenzverteilung von hypergeometrischen Verteilungen, falls obige Annahmen an $N_0(N)$ und $N$ zutreffen und $N\rightarrow\infty$\\
Binomialverteilung $\hateq$ Modell \ul{mit} Zurücklegen\\
Hypergeometrische Verteilung $\hateq$ Modell \ul{ohne} Zurücklegen\\
Bei großen $N$ ist es "fast egal", ob man mit/ohne Zurücklegen zieht.\\
Sei $n\in\mathbb{N}$ und $(N_0(N))_{N\in\mathbb{N}}$ Folge mit Werten in $\mathbb{N}$ mit $N_0(N)\leq N$ und $\lim_{N\rightarrow\infty}\frac{N_0(N)}{N}=p\in(0,1)$. Sei $(X_N)_{N\in\mathbb{N}}$ Folge von Zufallsvariablen mit $X_N\sim H(N,N_0(N),n)$, dann gilt für alle $k\in\{0,\dots,n\}$: $\lim_{N\rightarrow\infty}\mathbb{P}(\{X_N=k\})=\binom{n}{k}p^k(1-p)^{n-k}$\\
TODO Folien\\
$\sum_{k\geq 0}e^{-\lambda}\frac{\lambda^k}{k!}=e^{-\lambda}\sum_{k\geq 0}\frac{\lambda^k}{k!}=e^{-\lambda}e^{\lambda}=1$\\
\ul{$\lambda>0$}: $\Rightarrow$ d.h. $f(k):=e^{-\lambda}\frac{\lambda^k}{k!}$ für $k\in\mathbb{N}_0$ definiert eine Wahrscheinlichkeitsfunktion.\\
Eine Zufallsvariable $X$ heißt \ul{Poisson-verteilt} mit Parameter $\lambda>0$, falls $\forall k\in\mathbb{N}_0:\mathbb{P}(\{X=k\})=e^{-\lambda}\frac{\lambda^k}{k!}$\\
\ul{Notation}: $X\sim P(\lambda)$, $D_X=\mathbb{N}_0$\\
Die Poisson-Verteilung wird auch die "Verteilung von der seltenen Ereignissen" genannt.\\
TODO Folien\\
Eine Zufallsvariable $X$ heißt \ul{geometrisch verteilt} mit Parameter $p\in\left(0,1\right]$, falls $\forall k\in\mathbb{N}:\mathbb{P}(\{X=k\})=(1-p)^{k-1}\cdot p$\\
\ul{Notation}: $X\sim G(p)$\\
$D_X=\begin{cases}
\mathbb{N} & \text{falls }p\in (0,1)\\
\{1\} & \text{falls }p=1
\end{cases}$
\begin{enumerate}[1)]
	\item $\mathbb{P}(\{X=1\})=(1-p)^0\cdot p = p$, falls $X\sim G(p)$
	\item $X\sim G(p)$, bereits gesehen: $\sum_{k\geq0}\mathbb{P}(\{X=k\})=1$\\
	$G(p)$ ist tatsächlich Verteilung
	\item \ul{Anwendungen}:
	\begin{itemize}
		\item diskrete Wartezeiten: Warteschlangen an einem Schalter, zwischen 2 Anfragen in einem IT-System
		\item diskrete Lebensdauer von technischen Produkten
	\end{itemize}
\end{enumerate}
\ul{Gedächtnislosigkeit der geometrischen Verteilung}: Sei $X$ eine diskrete Zufallsvariable mit $D_X=\mathbb{N}$. Dann gilt: $X$ geometrisch verteilt $\Leftrightarrow$ $\forall k_1,k_2\in\mathbb{N}:\mathbb{P}(\{X>k_2\}\vert\{X>k_1\})=\mathbb{P}(\{X>k_2-k_1\})$\\
TODO? VL?\\
Eine Zufallsvariable heißt \ul{negativ-binomialverteilt} mit Parametern $p\in\left(0,1\right]$ und $r\in\mathbb{N}$, falls $\forall k\geq r:\mathbb{P}(\{X=k\})=\binom{k-1}{r-1}p^r(1-p)$.\\
Notation: $X~NB(r,p)$\\
$\sum_{k\geq r}\binom{k-1}{r-1}p^r(1-p)^{k-r}=1$, d.h. $f(k)=\binom{k-1}{r-1}p^r(1-p)^{k-r}$, $k\geq n$ ist eine Wahrscheinlichkeitsfunktion.\\
Ein $n$-dimensionaler Zufallsvektor $X=(X_1,\dots,X_n):\Omega\rightarrow\mathbb{R}^n$ heißt \ul{diskret}, falls es eine höchstens abzählbare Teilmenge $D\subset \mathbb{R}^n$ gibt mit $\mathbb{P}(\{X\in D\})=1$\\
\begin{enumerate}[1)]
	\item $D_X=\{x\in\mathbb{R}^n\ \vert\ \mathbb{P}(\{X=x\})>0\}$ ist die kleinste höchstens abzählbare Menge $D$ mit $\mathbb{P}(\{X\in D\})=1$, $X$ Zufallsvektor
	\item $X=(X_1,\dots,X_n)$ ist diskret genau dann, wenn alle $X_i$ diskret sind. $D_X\subseteq D_{X_1}\times\dots\times D_{X_n}$
\end{enumerate}
Sei $\varnothing\neq S\subseteq\mathbb{R}^n$ \ul{endlich} Ein $n$-dimensionaler Zufallsvektor $X:\Omega\rightarrow\mathbb{R}^n$ heißt \ul{gleichverteilt auf $S$}, falls $\forall \in S:\mathbb{P}(\{X=x\})=\frac{1}{\abs{S}}$ (Laplace-Modell für Zufallsvektoren).\\
Somit ist $X$ diskret ($D_X=S$)\\
Notation: $X~U(S)$
\section{Grundlagen allgemeiner Modelle}
\subsection{Die Borel'sche $\sigma$-Algebra in $\mathbb{R}^d$}

\end{document}





























