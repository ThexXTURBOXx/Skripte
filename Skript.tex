\documentclass[a4paper]{article}

\usepackage[ngerman]{babel}
\usepackage[utf8]{inputenc}
\usepackage{amsthm}
\usepackage{amsmath}
\usepackage{amssymb}
\usepackage{tikz,tkz-euclide}
\usepackage{titlesec}
\usepackage{textcomp}
\usepackage[titles]{tocloft}
\usepackage{csquotes}
\usepackage[babel]{microtype}
\usepackage{stmaryrd}
\usepackage[
  separate-uncertainty = true,
  multi-part-units = repeat
]{siunitx}

\usetkzobj{all}
\usetikzlibrary{shapes.misc}

\MakeOuterQuote{"}

\newcommand*\circled[1]{
  \tikz[baseline=(C.base)]\node[draw,circle,inner sep=0.75pt](C) {#1};\!
}

\renewcommand{\thesubsection}{\arabic{subsection}}
\titleformat{\section}{\normalfont\Large\bfseries}{Kapitel \arabic{section}: }{0em}{}
\titleformat{\subsection}{\normalfont\large\bfseries}{§\arabic{subsection} }{0em}{}
\titleformat{\subsubsection}{\normalfont\bfseries}{\arabic{subsection}.\arabic{subsubsection} }{0em}{}
\renewcommand{\cftsubsecpresnum}{§}
\newlength\mylength
\settowidth\mylength{\cftsubsecpresnum}
\settowidth\mylength{\cftsubsecaftersnum}
\addtolength\cftsubsecnumwidth{\mylength}
\renewcommand{\cftsecpresnum}{Kapitel }
\renewcommand{\cftsecaftersnum}{: }
\settowidth\mylength{\cftsecpresnum}
\addtolength\cftsecnumwidth{\mylength}

\newcommand{\ul}{\underline}
\newcommand{\proof}{\ul{Beweis:}\\}
\renewcommand{\qed}{\begin{flushright}
\ul{\(q.e.d.\)}
\end{flushright}}
\let\origphi\phi
\let\phi\varphi

\title{Lineare Algebra II: Skript}
\author{Nico Mexis}
\date{\today}

\begin{document}
\maketitle
\newpage

\tableofcontents
\newpage

\setcounter{section}{4}
\section{Endomorphismen}
\setcounter{subsection}{17}
\subsection{Eigenwerte (Buch: §4.1-4.2)}
Sei \(K\) ein Körper.\\
Sei \(V\) ein endlich dimensionaler \(K\)-Vektorraum.\\
Sei \(\phi: V \rightarrow V\) ein Endomorphismus.
\subsubsection{Definition}
\begin{itemize}
\item[\circled{a}] Ein Element \(\lambda \in K\) heißt ein \ul{Eigenwert} von \(\phi\), wenn es einen Vektor \(v \in V\ \backslash \{0\}\) gibt mit \(\phi(v)=\lambda*v\).
\item[\circled{b}] Ist \(\lambda \in K\) ein Eigenwert von \(\phi\), so heißt jeder Vektor \(v \in V \backslash \{0\}\) mit \(\phi(v) = \lambda*v\) ein Eigenvektor von \(\phi\) zum Eigenwert \(\lambda\).
\item[\circled{c}] Ist \(A \in Mat_n(K)\), so heißt ein \(\lambda \in K\) ein \ul{Eigenwert} von A, wenn es ein \(v \in K^n\backslash\{0\}\) gibt mit \(A*v=\lambda*v\).
\end{itemize}
\subsubsection{Beispiel}
\begin{itemize}
\item[\circled{a}] Das Element \(0 \in K\) ist ein Eigenwert von \(\phi\), wenn \(\phi\) nicht injektiv ist.
\item[\circled{b}] Das Element \(1 \in K\) ist ein Eigenwert von \(\phi\), wenn \(\phi\) einen \ul{Fixpunkt} \(v \neq 0\) hat (d.h. \(\phi(v)=v\)).
\end{itemize}
\subsubsection{Beispiel}
Sei \(K = \mathbb{R}, V= \mathbb{R}^2\).\\
Sei \(\phi:\mathbb{R}^2\rightarrow\mathbb{R}^2\) die Drehung um \(0=(0,0)\) um den Winkel \(\alpha \in [0,2\pi[\).\\
\begin{itemize}
\item[\circled{a}] Ist \(\alpha = 0\), so ist \(\phi = id_{\mathbb{R}^2}\) und \(\lambda = 1\) ist der einzige Eigenwert von \(\phi\).
\item[\circled{b}] Ist \(\alpha = \pi\), so ist \(\phi = -id_{\mathbb{R}^2}\) und \(\lambda = -1\) ist der einzige Eigenwert von \(\phi\).
\item[\circled{c}] Ist \(\alpha \notin \{0,\pi\}\), so besitzt \(\phi\) keine Eigenwerte.
\end{itemize}
\subsubsection{Beispiel}
Sei \(K = \mathbb{R}, V= \mathbb{R}^2\) und \(\sigma :\mathbb{R}^2 \rightarrow\mathbb{R}^2\) die Spiegelung an der Geraden \(G\) durch \(0\), die mit der x-Achse einen Winkel \(\frac{\alpha}{2}\) einschließt.\newpage
\ul{Skizze:}\\
\begin{tikzpicture}
  \tkzInit[ymin=0,ymax=1.11,xstep=0.1,ystep=0.1,xmin=0,xmax=1.11]
  \tkzGrid
  \tkzLabelX
  \tkzLabelY
  \tkzDrawX[below=12 pt,label={x}]
  \tkzDrawY[left=12 pt,label={y}]
  \tkzDefPoint(1,0){V}
  \tkzDefPoint(0,0){S}
  \tkzDefPoint(0.9,0.9){P}
  \tkzDefPoint(1, 0.75){T}
  \tkzDefPoint(0.27, 0.96){U}
  \tkzDefPoint(0.64, 0.48){E}
  \tkzSetUpPoint[size = 10]
  \tkzDrawPoints[fill=black](V,U)
  \tkzLabelPoint[above](U){\((cos(\alpha),sin(\alpha))\)}
  \tkzLabelPoint[above right](V){\((1,0)\)}
  \tkzDrawSegment[ultra thick,black](S,T)
  \tkzDrawSegment[thick,black](U,V)
  \tkzDrawSegment[thick,black](S,U)
  \tkzMarkAngle(T,E,U)
  \tkzLabelAngle[pos=0.55](T,E,U){\(\bullet\)}
  \tkzMarkAngle(T,S,U)
  \tkzLabelAngle[pos=0.75](T,S,U){\(\frac{\alpha}{2}\)}
  \tkzMarkAngle(V,S,T)
  \tkzLabelAngle[pos=0.75](V,S,T){\(\frac{\alpha}{2}\)}
  \tkzLabelPoint[above](P){\(\begin{pmatrix}
cos(\alpha) & sin(\alpha) \\
sin(\alpha) & cos(\alpha)
\end{pmatrix}\)}
\end{tikzpicture}\\\\
Ein Eigenwert ist \(\lambda=1\) und die Menge der Eigenvektoren zum Eigenwert \(\lambda=1\) ist \(G\backslash\{0\}\).\\
Ein weiterer Eigenwert ist \(\lambda=-1\) und die Menge der Eigenvektoren zum Eigenwert \(\lambda=-1\) ist \(H\backslash\{0\}\), wobei \(H\) die zu \(G\) senkrechte Gerade durch 0 ist.
\subsubsection{Definition}
Sei \(\lambda \in K\) ein Eigenwert von \(\phi\). Dann heißt \(Eig(\phi,\lambda)=\{v \in V | \phi (v)=\lambda *v\}\) der \ul{Eigenraum} zum Eigenwert \(\phi\).
\subsubsection{Satz}
Sei \(\phi:V \rightarrow V\) ein Endomorphismus.
\begin{itemize}
\item[\circled{a}] Ist \(\lambda \in K\) ein Eigenwert von \(\phi\), so ist \(Eig(\phi,\lambda)\) ein Untervektorraum von V.
\item[\circled{b}] Sind \(\lambda,\mu \in K\) zwei verschiedene Eigenwerte von \(\phi\), so gilt: \(Eig(\phi,\mu)=\{0\}\).
\end{itemize}
\newpage
\ul{Beweis:}
\begin{itemize}
\item["\circled{a}"] Wegen \(\phi(0)=\lambda*0=0\) gilt \(\phi(v-w)=\phi(v)-\phi(w)=\lambda*v-\lambda*w=\lambda*(v-w)\), also \(v-w \in Eig(\phi,\lambda)\). Nach dem Untervektorraumkriterium folgt die Behauptung.
\item["\circled{b}"] Sei \(v \in Eig(\phi,\lambda)\cap Eig(\phi,\mu)\). Dann gilt \(\lambda*v=\phi(v)=\mu*v\), also \(\underbrace{(\lambda-\mu)}_{\neq 0}*v=0\). Dies liefert \(v=0\).
\end{itemize}
\qed
\subsubsection{Beispiel}
\begin{itemize}
\item[\circled{a}] Sei \(K = \mathbb{R}\), \(V = \mathbb{R}^2\) und \(\phi:\mathbb{R}^2\rightarrow \mathbb{R}^2\) die Spiegelung an der Winkelhalbierenden. Dann gilt \(M_{\xi}^{\xi}(\phi)=\begin{pmatrix}
0 & 1 \\
1 & 0
\end{pmatrix}\) und die Eigenwerte von \(\phi\) sind \(\lambda = \pm 1\)
\begin{itemize}
\item[\circled{1}] \(Eig(\phi,1)=\mathbb{R}*(1,1)\)
\item[\circled{2}] \(Eig(\phi,-1)=\mathbb{R}*(1,-1)\)
\end{itemize}
\item[\circled{b}] Sei \(K = \mathbb{R}\), \(V = \mathbb{R}^3\) und \(\phi:\mathbb{R}^3\rightarrow \mathbb{R}^3\) die Drehung um die z-Achse um einen Winkel \(\alpha \in ]0,\pi[\). Dann gilt \(M_{\xi}^{\xi}(\phi)=\begin{pmatrix}
cos(\alpha) & -sin(\alpha) & 0 \\
sin(\alpha) & cos(\alpha) & 0 \\
0 & 0 & 1
\end{pmatrix}\) und \(\lambda = 1\) ist der einzige Eigenwert mit \(Eig(\phi,1)=\mathbb{R}*\underbrace{(0,0,1)}_{z-Achse}\).
\end{itemize}
\subsubsection{Definition}
\begin{itemize}
\item[\circled{a}] Der Endomorphismus \(\phi:V\rightarrow V\) heißt \ul{diagonalisierbar}, wenn es eine Basis \(B\neq (v_1,\dots,v_d)\) von \(V\) gibt, die aus Eigenvektoren besteht.\\
Sei \(\phi(v_i)=a_iv_i\) mit \(a_i \in K\) für \(i=1,..,d\). Dann gilt also \(M^B_B(\phi)=\begin{pmatrix}
a_1 & 0 & \hdots & 0 \\
0 & a_2 & \hdots & 0 \\
\vdots & \vdots & \ddots & \vdots \\
0 & 0 & \hdots & a_n
\end{pmatrix}\) Diagonalmatrix.
\item[\circled{b}] Eine Matrix \(A \in Mat_n(K)\) heißt \ul{diagonalisierbar}, wenn der zugehörige Endomorphismus \(\phi_A:K^n\rightarrow K^n\) diagonalisierbar ist, d.h. wenn es ein \(T \in GL_n(K)\) gibt, sodass \(TAT^{-1}\) eine Diagonalmatrix ist.
\end{itemize}
\subsubsection{Satz (Charakterisierung der Diagonalisierbarkeit)}
Für einen Endomorphismus \(\phi:V\rightarrow V\) sind die folgenden Bedingungen äquivalent:
\begin{itemize}
\item[\circled{a}] \(\phi\) ist diagonalisierbar.
\item[\circled{b}] Es gibt eine Basis \(B\) von \(V\), sodass \(M_B^B(\phi)\) eine Diagonalmatrix ist.
\item[\circled{c}] Für jede Basis \(B\) von \(V\) ist \(M_B^B(\phi)\) eine diagonalisierbare Matrix.
\item[\circled{d}] Es gibt eine Basis \(B\) von \(V\), die aus Eigenvektoren von \(\phi\) besteht.
\end{itemize}
\ul{Beweis:}
\begin{itemize}
\item["\circled{a} \textrightarrow \space\circled{b}"] Def. 18.8a
\item["\circled{b} \textrightarrow \space\circled{c}"] Sei \(B\) eine Basis von \(V\), sodass \(M_B^B(\phi)\) eine Diagonalmatrix ist und sei \(C\) eine weitere Basis von \(V\). Dann gilt: \(M_B^B(\phi)=T_C^B*M_C^C(\phi)*(T_C^B)^{-1}\) mit \(T_C^B\in GL_n(K)\). Also ist \(M_C^C(\phi)\) diagonalisierbar.
\item["\circled{c} \textrightarrow \space\circled{d}"] Nach Def. 18.8b gibt es \(T \in GL_n(K)\) mit \(T*M_C^C(\phi)*T^{-1}\) Diagonalmatrix. Bzgl. der transponierten Basis \(B=T*C\) ist also \(M_B^B(\phi)\) eine Diagonalmatrix, d.h. \(B\) besteht aus Eigenvektoren von \(\phi\).
\item["\circled{d} \textrightarrow \space\circled{a}"] Nach Def.
\end{itemize}
\qed
\subsubsection{Beispiel}
Eigenwerte und Eigenvektoren spielen eine große Rolle beim Lösen von \ul{Differentialgleichungen}.\\
Suche z.B. im \(\mathcal{C}^1(\mathbb{R})\) die Lösungen von\\
\(f_1'(t)=f_1(t)-f_2(t)\)\\
\(f_2'(t)=2f_1(t)+4f_2(t)\).\\
Man macht folgenden Lösungsansatz: \(f_1(t)=c_1*e^{\lambda t}\) und \(f_2(t)=c_2*e^{\lambda t}\)\\
\(\Rightarrow \begin{pmatrix}
f_1'(t) \\
f_2'(t)
\end{pmatrix} = \begin{pmatrix}
\lambda c_1e^{\lambda t} \\
\lambda c_2e^{\lambda t}
\end{pmatrix} = \lambda * \begin{pmatrix}
f_1(t) \\
f_2(t)
\end{pmatrix}\).\\
Das Differentialgleichungssystem lautet also:\\
\(\begin{pmatrix}
f_1'(t) \\
f_2'(t)
\end{pmatrix} = \lambda * \begin{pmatrix}
f_1(t) \\
f_2(t)
\end{pmatrix} = \begin{pmatrix}
1 & -1 \\
2 & 4
\end{pmatrix} * \begin{pmatrix}
f_1(t) \\
f_2(t)
\end{pmatrix}\).\\
Im \(\mathbb{R}\)-Vektorraum \(\mathcal{C}^1(\mathbb{R})\) ist \(\begin{pmatrix}
f_1'(t) \\
f_2'(t)
\end{pmatrix}\) ein Eigenvektor der Matrix \(\begin{pmatrix}
1 & -1 \\
2 & 4
\end{pmatrix}\) zum Eigenwert \(\lambda\).
\subsubsection{Satz}
Ist \(\lambda \in K\) ein Eigenwert von \(\phi\), so gilt: \(Eig(\phi, \lambda)=Ker(\phi-\lambda*id_V)\).\\
\ul{Beweis:}\\
Es gilt \(v \in Eig(\phi, \lambda) \Leftrightarrow \phi(v)=\lambda *v \Leftrightarrow(\phi-\lambda*id_V)(v)=0 \Leftrightarrow v\in Ker(\phi-v*id_V)\).
\qed
\subsubsection{Definition}
Sei \(x\) eine Unbestimmte und \(\mathcal{A} = (a_{ij}) \in Mat_n(K)\). Dann gilt: \(\mathcal{A}-x*\mathcal{I}_n=\begin{pmatrix}
a_{11}-x & a_{12} & \hdots & a_{1n} \\
a_{21} & a_{22}-x & \hdots & a_{2n} \\
\vdots & \vdots & \ddots & \vdots \\
a_{n1} & a_{n2} & \hdots & a_{nn}-x
\end{pmatrix} \in Mat_n(K[x])\).\\
Dann heißt \(det(\mathcal{A}-x*\mathcal{I}_n)\in K[x]\) das \ul{charakteristische Polynom} von \(\mathcal{A}\) und wird mit \(\underset{"chi"}{\chi_A}(x)\) bezeichnet.
\subsubsection{Beispiel}
Sei \(\mathcal{A}=\begin{pmatrix}
0 & 1 \\
1 & 0
\end{pmatrix} \in Mat_2(\mathbb{Q})\). Dann gilt \(\chi_A(x)=det\begin{pmatrix}
-x & 1 \\
1 & -x
\end{pmatrix}=x^2-1\).
\subsubsection{Satz}
Seien \(B,C\) zwei Basen von \(V\). Dann gilt:\\
\(\chi_{M_B^B(\phi)}(x)=\chi_{M_C^C(\phi)}(x)\)\\
\ul{Beweis:}\\
Nach der Transformationsformel gilt:\\
\(M_C^C(\phi)=T_C^B*M_B^B(\phi)*T_B^C\)\\
Es folgt: \(\chi_{M_C^C(\phi)}(x)=det(T_C^BM_B^B(\phi)T_B^C-\underbrace{T_C^Bx\mathcal{I}_nT_B^C}_{=x\mathcal{I}_n})\\=det(T_C^B(M_B^B(\phi)-x\mathcal{I}_n)T_B^C)\\=det(T_C^B)*det(M_B^B(\phi)-x*\mathcal{I}_n)*det(T_C^B)^{-1}=\chi_{M_B^B(\phi)}(x)\).
\qed
\ul{Folgerung:} Das Polynom \(\chi_{M_B^B(\phi)}(x)\) hängt nur von \(\phi\), aber nicht von der Wahl der Basis \(B\) ab. Es heißt das \ul{charakteristische Polynom} von \(\phi\) und wird mit \(\chi_\phi(x)\) bezeichnet.
\subsubsection{Satz}
Ein Element \(\lambda \in K\) ist genau dann ein Eigenwert von \(\phi\), wenn \(\chi_{\phi}(\lambda)=0\) gilt.\\
\ul{Beweis:}\\
Genau dann ist \(\lambda \in K\) ein Eigenwert von \(\phi\), wenn \(Ker(\phi-\lambda id_V)\neq \{0\}\) gilt. Dies ist genau dann der Fall, wenn \(\phi-\lambda id_V\) nicht bijektiv ist. Letztere Bedingung ist äquivalent zu \(det(\phi-\lambda id_V)=0\), also zu \(\chi_\phi(\lambda)=0\).
\qed
\subsubsection{Beispiel}
\begin{itemize}
\item[\circled{a}] Sei \(\phi:\mathbb{R}^2\rightarrow\mathbb{R}^2\) die Spiegelung an der Winkelhalbierenden \(G=\mathbb{R}*\begin{pmatrix}
1 \\
1
\end{pmatrix}\). Dann gilt: \(M_\xi^\xi(\phi)=\begin{pmatrix}
0 & 1 \\
1 & 0
\end{pmatrix}\) und  \(\chi_\phi(x)=x^2-1=(x-1)(x+1)\). Also besitzt \(\phi\) zwei Eigenwerte, nämlich \(\lambda_1=1\) und \(\lambda_2=-1\).\\
Für \(\lambda_1=1\) gilt \(Eig(\phi,\lambda_1)=Ker(\phi-\lambda_1 id_{\mathbb{R}^2})=Ker\begin{pmatrix}
-1 & 1 \\
1 & -1
\end{pmatrix} = \\<\begin{pmatrix}
1 \\
1
\end{pmatrix}> = G\).\\
Für \(\lambda_2=-1\) gilt \(Eig(\phi,\lambda_2)=Ker(\phi-\lambda_2 id_{\mathbb{R}^2})=Ker\begin{pmatrix}
1 & 1 \\
1 & 1
\end{pmatrix} = \\<\begin{pmatrix}
-1 \\
1
\end{pmatrix}> = G^\bot\).
\item[\circled{b}] Sei \(\psi:\mathbb{R}^2\rightarrow\mathbb{R}^2\) die Drehung um \(0\) um den Winkel \(\alpha \in ]0,\pi[\). Dann gilt \(M_\xi^\xi(\psi)=\begin{pmatrix}
cos(\alpha) & -sin(\alpha) \\
sin(\alpha) & cos(\alpha)
\end{pmatrix}\).\\
\(\chi_\psi(x)=det\begin{pmatrix}
cos(\alpha)-x & -sin(\alpha) \\
sin(\alpha) & cos(\alpha)-x
\end{pmatrix}=(cos(\alpha)-x)^2+sin^2(\alpha)=x^2-2cos(\alpha)x+sin^2(\alpha)+cos^2(\alpha)=x^2-2cos(\alpha)x+1\).\\
Dieses quadratische Polynom hat die Diskriminante \(\triangle=4cos^2(\alpha)-4=-4sin^2(\alpha)<0\). Somit besitzt \(\psi\) keine Eigenwerte.
\end{itemize}
\subsubsection{Bemerkung}
Ein Polynom \(f \in K[x]\backslash K\) braucht keine Nullstelle in \(K\) zu besitzen (z.B. \(x^2+1 \in \mathbb{R}[x])\).\\
Der \ul{Fundamentalsatz der Algebra} besagt, dass jedes Polynom \(f \in \mathbb{C} [x] \backslash \mathbb{C}\) eine Nullstelle besitzt und somit in ein Produkt von Linearfaktoren zerfällt.
\subsubsection{Satz (Eigenschaften des charakteristischen Polynoms)}
Sei \(\phi:V\rightarrow V\ K\)-linear und \(n=dim_K(V)\). Dann gilt:
\begin{itemize}
\item[\circled{a}] \(deg(\chi_\phi(x))=n\).
\item[\circled{b}] Der Gradkoeffizient von \(\chi_\phi(x)\) ist \((-1)^n\), also \(\chi_\phi(x)=(-1)^nx^n+(Monome\ kleineren\ Grades)\)
\item[\circled{c}] Der konstante Term von \(\chi_\phi(x)\) ist \(det(\phi)=\chi_\phi(0)\).
\end{itemize}
\ul{Beweis:}\\
\begin{itemize}
\item["\circled{a},\circled{b}"] Sei \(\mathcal{A}=(a_{ij})\in Mat_n(K)\) eine Darstellungsmatrix von \(\phi\). Dann gilt \(\chi_\phi(x)=det\begin{pmatrix}
a_{11}-x & & * \\
 & \ddots & \\
 * & & a_{nn}-x
\end{pmatrix}=(a_{11}-x)(a_{22}-x)*\dots*(a_{nn}-x)+(Monome\ niedriegeren\ Grades\ als\ n)=(-x)^n+(Monome\ niedriegeren\ Grades\ als\ n)\)
\item["\circled{c}"] \(\chi_\phi(0)=det(\phi-0*id_V)=det(\phi)\).
\end{itemize}
\qed
\subsubsection{Bemerkung (Der zweite Koeffizient von \(\chi_\phi(x)\))}
Schreibe \(\chi_\phi(x)=(-1)^nx^n+c_{n-1}x^{n-1}+\dots+c_1x+c_0\). Dann heißt \(Spur(\phi)=\underset{"Trace"}{Tr}(\phi)=(-1)^{n-1}c_{n-1}\) die \ul{Spur} von \(\phi\).\\
Sie hängt nicht ab von der Wahl einer Darstellungsmatrix \(\mathcal{A}=(a_{ij})\). Es gilt \(Spur(\mathcal{A})=a_{11}+a_{22}+\dots+a_{nn}\).\\
\ul{Beweis:}\\
Um einen Summanden \(c*x^{n-1}\) in \(det\begin{pmatrix}
a_{11}-x & & & * \\
 & a_{22}-x & & \\
 & & \ddots & \\
* & & & a_{nn}-x
\end{pmatrix}\) zu erhalten, muss man genau \(n-1\)mal den Faktor \(a_{ii}-x\) wählen. Also suchen wir den Koeffizienten von \(x^{n-1}\) in \((a_{11}-x)*\dots*(a_{nn}-x)\). Beim Ausmultiplizieren muss man \((n-1)\)mal \(-x\) wählen und einmal \(a_{ii}\). Also gilt \(c=(-1)^{n-1}(a_{11}+a_{22}+\dots+a_{nn})=(-1)^{n-1} Spur(\phi)\).
\qed
HIER FEHLT VIEL
\setcounter{subsection}{21}
\subsection{Verallgemeinerte Eigenräume}
Sei \(K\) Körper, \(V\) endlich-dimensionaler \(K\)-Vektorraum, \(\phi \in End_K(V)\)\\
\(\chi_\phi(v)=det(\phi-x\cdot id_V)\) charakteristisches Polynom\\
\(\mu_\phi(x)=\) normiertes Polynom kleinsten Grades mit \(\mu_\phi (\phi) = 0 \hat{=}\) Minimalpolynom von \(\phi\).
\subsubsection{Definition (Großer Kern und kleines Bild)}
(a) Die Kette \(Ker(\phi) \subseteq Ker(\phi^2) c_ Ker(\phi^3) c_ \dots\) von \(K\)-Untervektorräumen von \(V\) wird stationär. Der Untervektorraum \(BigKer(\phi)=Ker(\phi^i)\) mit \(i >> 0\) heißt der \(\ul{große Kern}\) von \(\phi\).
(b) Die Kette \(Im(\phi) _c Im(\phi^2) _c \dots\) wird stationär. Der Untervektorraum \(SmIm(\phi) = Im(\phi')\) mit \(i >> 0\) heißt das \ul{kleine Bild} von \(\phi\).
\subsubsection{Satz (Das Lemma von Fitting)}
(a) Gilt \(Ker(\phi^m)=Ker(\phi^{m+1})\) für ein \(m \geq 1\), so gilt \(Ker(\phi^m)=Ker(\phi^i)\) für alle \(i \geq m\) und somit \(BigKer(\phi)=Ker(\phi^m)\).
(b) Gilt \(Im(\phi^{m'})=Im(\phi^{m'+1})\) für ein \(m' \geq 1\), so folgt \(Im(\phi^{m'})=Im(\phi^i)\) für alle \(i \geq m'\) und somit \(SmIm(\phi)=Im(\phi^{m'})\).
(c) Sind \(m,m'\) in (a) und (b) minimal, so folgt \(m=m'\).
(d) Die Ketten \({0} c_ Ker(\phi) c+ Ker(\phi^2) c+ \dots c+ Ker(\phi^m)=BigKer(\phi)\) und \(V _c Im(\phi) +c Im(\phi^2) +c \dots +c Im(\phi^{m'}) = SmIm(\phi)\) enthalten strikte Inklusionen.
(e) \fbox{\(V=BigKer(\phi) o+ SmIm(\phi)\)}
\ul{Beweis:}\\
"a" Es genügt \(Ker(\phi^{m+1}) =Ker(\phi^{m+2})\) zu beweisen. Zeige "\(_c\)".Sei \(v \in Ker(\phi^{m+2})\), also \(\phi^{m+1}(v)=0\), so folgt \(\phi(v)\in Ker(\phi^{m+1})=Ker(\phi^m)\), also \(\phi^{m+1}(v)= \phi^m(\phi(v))=0\) und somit \(v \in Ker(\phi^{m+1})\).
"b" Zu zeigen ist nur \(Im(\phi^{m+1}) c_ Im(\phi^{m+2})\). Dazu sei \(v\in Im(\phi^{m+1})\), also \(v=\phi^{m+1}(w)\) mit \(w\in V\). Dann folgt \(v=\phi(\overset{~}{v})\) mit \(\overset{~}{v} = \phi^m(w)\in I(\phi^m)=Im(\phi^{m+1})\). Somit gilt \(v~ = \phi^{m+1}(w~)\) mit \(w~ \in V\) und \(v=\phi(v~)=\phi^{m+1}(w~)\in Im(\phi^{m+1})\).
"c" Es gilt \(dim_K(V)=dim_K(Ker(\phi^m))+dim_K(Im(\phi^m))\) und \(dim_K(V)=dim_K(Ker(\phi^{m+1})+dim_K(Im(\phi^{m+1})\).
Aus \(Ker(\phi^m)=Ker(\phi^{m+1})\) folgt damit \(dim_K(Im(\phi^m))=dim_K(\phi^{m+1})\) und daher \(Im(\phi^m)=Im(\phi1{m+1})\).
"d" folgt aus (a) und (b).
"e" Nach (c) gilt \(dim_K(V)=dim_K(Ker(\phi^m))+dim_K(Im(\phi^m))\) für das minimale \(m\). (Insbesondere gilt: \(BigKer(\phi)=Ker(\phi^m)\) und \(SmIm(\phi)=Im(\phi^m)\) für dieses \(m\)).\\
Zu zeigen ist also \(Ker(\phi^m) \cap Im(\phi^m)={0}\).\\
Sei \(v = \phi^m(w)\in Ker(\phi^m)\) für ein \(w\in V\). Dann gilt \(\phi^m(v)=\phi^{2m}(w)=0\), also \(w\in Ker(\phi^{2m})=Ker(\phi^m)\) und somit \(v=\phi^m(w)=0\).
\qed
\ul{Wdh.:}\\
Sei \(\lambda\in K\) ein Eigenwert von \(\phi\). Dann ist \(x-\lambda\) ein Eigenfaktor und \(Eig(\phi, \lambda)=Ker(\phi-\lambda\cdot id_V)={v\in V | \phi(v)=\lambda v}\).
\subsubsection{Definition}
Sei \(\phi\in End_K(V)\) und \(\mu_\phi(x)^m_1\dotsp_s(x)^{m_s}\) mit den Eigenfaktoren \(p_1(x),\dots,p_s(x)\) und mit \(m_i\geq 1\).
(a) der \(K\)-Untervektorraum \(Eig(\phi, p_i(x)) = Ker(p_i(\phi))\) heißt der \ul{Eigenraum} von \(\phi\) bzgl. des Eigenfaktors \(p_i(x)\):
(b) Der \(K\)-Untervektorraum \(Gen(\phi, p_i(x))=BigKer(p_i(\phi))\) heißt der \ul{verallgemeinerte  Eigenraum} (oder \ul{Hauptraum}) von \(\phi\) bzgl. \(p_i(x)\).
\subsubsection{Bemerkung}
Ist \(p_i(x)=x-\lambda\) mit \(\lambda\in K\) Eigenwert, so gilt:
(a) \(Eig(\phi,p_i(x))=Ker(\phi-id_V)=Eig(\phi,\lambda)\).
(b) \(Gen(\phi,p_i(x))=BigKer(\phi-\lambda\cdot id_V)=Ker(\phi-\lambda\cdot id_V)^m=\underset{Hau(\phi, \lambda)}{Gen(\phi, \lambda)}\). TODO PFEIL m>>0 BEI m
\subsubsection{Lemma}
Sei \(p(x)\in K[x]\backslash\{0\}\) ein Vielfaches von \(\mu_\phi(x)\). Es gelte \(p(x)=q_1(x)q_2(x)\) mit \(ggT(q_1(x),q_2(x))=1\). Ferner sei \(W_1=Ker(q_1(\phi))\) und \(W_2=Ker(q_2(\phi))\). Dann gilt:
(a) Für jedes \(f\in K[x]\) sind \(W_1\) und \(W_2 f(\phi)\)-invariant.
(b) Für \(i \neq j\) ist \(q_i(\phi)\Big|_{W_j}: W_j \rightarrow W_j\) ein Isomorphismus. TODO Einschränkung
(c) \(V=W_1 \oplus W_2\)
(d) Ist \(p(x)=\mu_\phi(x)\), so gilt \(q_i(x)=\mu_{\phi_{|_{W_i}}}(x)\) für \(i=1,2\). TODO Einschränkung
\ul{Beweis:}
"a" Da \(f(\phi),q_1(\phi),q_2(\phi)\) kommentieren, gilt für \(v\in W_i=Ker(q_i(\phi))\) dass \(q_i(\phi)\underbrace{f(\phi)(v)}_{z.z.\in W_i}=f(\phi)q_i(\phi)(v)=f(\phi)(0)=0\) und somit \(f(\phi)(v)\in Ker(q_i(\phi))=W_i\).\\
"b" Wegen \(ggT(q_1(x),q_2(x))=1\) gibt es \(f_1(x),f_2(x)\in K[x]\) mit \(f_1(x)\cdot q_1(x)+f_2(x)\cdot q_2(x)=1\). Für \(v\in V\) folgt damit \(f_1(\phi)q_1(\phi)(v)+f_2(\phi)q_2(\phi)(v)=v\) (*). TODO EINRAHMEN\\
Hieraus folgt \(W_1 \cap W_2 = Ker(q_1(\phi)) \cap Ker(q_2(\phi))={0}\) nach (*) und daher ist \(q_i(\phi)|W_j:W_j\rightarrow W_j\) injektiv, also bijektiv. TODO Einschränkung
"c" Es ist nur noch zu zeigen, dass \(V=W_1+W_2\) gilt. Sei \(v\in V\). Schreibe \(v=w_1+w_2\) mit \(w_i=f_i(\phi)q_i(\phi)(v)\) nach (*).\\
Dann folgt für \(j \neq i q_j(\phi)(w_i)=q_j(\phi)f_i(\phi)q_i(\phi)(v)=f_i(\phi)\underbrace{p(\phi)}_{=0}(v)=0\) und somit \(w_i\in Ker(q_j(\phi))=W_j\). Dies zeigt \(v\in W_1+W_2\).
"d" Sei \(q_i'(x)\) das Minimalpolynom von \(\phi\Big|_{W_i}\).
Nach Def. von \(W_j\) gilt \(q_i'(x)\Big|_{q_i(x)}\), denn \(q_i(\phi\Big|_{W_i})=q_i(\phi)\Big|_{W_i}=0\).
Also sind auch \(q_1'(x)\) teilerfremd und \(q_1'(x)q_2'(x)=\mu_\phi(x)=q_1(x)q_2(x)\), also \(q_1'(x)=q_1(x)\) und \(q_2'(x)=q_2(x)\). TODO Einschränkung
\qed
\subsubsection{Definition}
Ein Endomorphismus \(\phi\in End_K(V)\) heißt \ul{nilpotent}, wen es ein \(m\geq 1\) gibt mit \(\phi^m=0\). In diesem Fall heißt das kleinste solche m der \ul{Nilpotenzindex} von \(\phi\) und wird mit \(nix(\phi)\).
\subsubsection{Beispiel}
Sei \(\phi\in End_\mathbb{Q}(\mathbb{Q}^3)\) mit \(M_\xi^\xi(\phi)= TODO MATRIX\). Dann gilt \(M_\xi^\xi(\phi^2)=TODO MATRIX\) und \(\phi^3=0\), also \(nix(\phi)=3\).
\subsubsection{Lemma}
Seien \(U', U''\) zwei \(\phi\)-invariante \(K\)-Untervektorräume von \(V\) mit \(V=U' +o U''\).
Sei \(\phi'=\mu_{\phi \Big|_{U'}}\) und \(\phi''=\mu_{\phi \Big|_{U''}}\). Dann gilt \(\mu_\phi(x)=kgV(\mu_{\phi \Big|_{U'}}(x),\mu_{\phi \Big|_{U''}}(x))\).\\
\ul{Beweis:}\\
Es gilt \(\mu_\phi(\phi)=0\). Also folgt \(\mu_\phi(\phi\Big|_{U'})=0\) und \(\mu_\phi(\phi \Big|_{U''})=0\). Also ist \(\mu_\phi\) ein gemeinsames Vielfaches von \(\mu_{\phi'}\) und \(\mu_{\phi''}\). Umgekehrt sei \(f(x)=kgV(\mu_{\phi'},\mu_{\phi''}(x))\). Zu zeigen: \(f(\phi)=0\). Betrachte \(f(\phi)\backslash U'=f(\phi \Big|_{U'})=0\) da \(f\) ein Vielfaches von \(\mu_{\phi'}\) ist. Genauso gilt \(f(\phi) \Big|_{U''}=f(\phi|U'')=0\). Wegen \(V=U' o+ U''\) folgt \(f(\phi)=0\), also \(\mu_\phi \Big|_{f}\).
\qed
\subsubsection{Theorem (Die verallgemeinerte Eigenraumzerlegung / Die Hauptraumzerlegung / Die Primärzerlegung von \(\phi\))}
Sei \(\phi\in End_K(V)\) und \(\mu_\phi(x)=p_1(x)^{m_1}\dotsp_s(x)^{m_s}\) die Zerlegung in Eigenfaktoren.
(a) \(V=Gen(\phi,p_1(x)) o+ \dots o+ Gen(\phi, p_s(x))\).
(b) Die Einschränkung von \(p_i(\phi)\) auf \(\underbrace{Gen(\phi,p_i(x))}_{G_i}\) ist nilpotent mit \(nix(p_i(\phi)\Big|_{G_i})=m_i\).
(c) Die Einschränkung von \(p_i(\phi)\) auf \(G_j\) mit \(j \neq i\) sind Isomorphismen.
(d) Es gibt \(G_i=Gen(\phi,p_i(x))=Ker(p_i(\phi)^{m_i})=BigKer(p_i(\phi))\)
\ul{Beweis:}\\
Wir schließen mit vollständiger Induktion nach \(s\).\\
\ul{\(s=1\)}: Sei \(\mu_\phi(x)=p_1(x)^{m_1}\). Dies bedeutet \(p_1(\phi)^{m_1}=0\), d.h. \(p_1(\phi)\) ist nilpotent und \(nix(p_1(\phi))=m_1\) nach Def. von \(\mu_\phi(x)\). Offenbar folgt \(V=Ker(p_1(\phi)^{m_1})=BigKer/p_1(\phi))\).\\
\ul{\(s > 1\)}: Verwende das Lemma mit \(q_1(x)=p_1(x)^{m_1}\dotsp_{s-1}(x)^{m_{s-1}}\) und \(q_2(x)=p_s(x)^{m_s}\). Nach Teil (c) des Lemmas folgt \(V=Ker(q_1(\phi)) o+ Ker(q_2(\phi))\underset{I.V.}{=}(Ker(p_1(\phi))^{m_1} o+ \dots o+ Ker(p_{s-1}(\phi)^{m_{s-1}}) o+ Ker(p_s(\phi)^{m_s})\). Nach Teil (d) des Lemmas ist jeweils \(p_i(x)^{m_i}\) das Minimalpolynom von \(p_i|G_i\), Dies beweist (b). Teil (c) folgt aus Teil (b) des Lemmas. Teil (d) folgt aus (b).
TODO UNDERSET
\qed
\subsubsection{Beispiel}
Sei K = \mathbb{Q} und \phi \in End_\mathbb{Q}(\mathbb{Q}^3) mit Matrix M_\chi^\chi(\phi)=\begin{pmatrix}
e4 & e5 & e6 & e7 & e3 & e7 & e8 & e3
\end{pmatrix}.\\
Dann \mu_\phi(x)=x^6-x^2=x^2(x-1)(x+1)(x^2+1)\\
\xi_\phi(x)=x^4(x-1)(x+1)(x^2+1)\\
Nun berechnen wir: Gen(\phi, x+1)=Ker(\phi+id_V)=<(0,0,1,0,0,-1,1,-1)>\\
Gen(\phi, x-1)=Ker(\phi-id_V)=<(0,0,-1,0,0,-1,-1,-1)>\\
Gen(\phi, x^2+1)=Ker(\phi^2+id_V)=<(0,0,1,0,0,0,-1,0)<(0,0,0,0,0,1,0,-1)>\\
Gen(\phi, x)=BigKer(\phi)=Ker(\phi^2)=<(0,0,0,1,0,-1,0,0),(0,0,0,0,1,0,0,-1),(1,0,-1,0,0,0,0,0),(0,1,0,0,0,0,-1,0)>\\
Insgesamt folgt \(V=Ker(\phi+id_V)\oplus Ker(\phi-id_V) \oplus Ker(\phi^2+id_V)\oplus Ker(\phi^2)\).\\
Beachte: \underbrace{Ker(\phi)}_{=<e_4-e_6,e_5-e_8>} c+ Ker(\phi^2).
\subsection{Nilpotente Endomorphismen (Buch: §4.5-4.6}
Sei K ein Körper, V ein endlich-dimensionaler K-Vektorraum und \phi\in End_K(V).
\phi nilpotent \Leftrightarrow es gibt ein m\geq 1 mit \phi^m=0. Das minimale solche m heißt der Nilpotenzindex nix(\phi).\\
Im Folgenden sei \phi nilpotent und m=nix(\phi).
\subsubsection{Definition}
(a) Die Kette {0} c+ Ker(\phi) c+ Ker(\phi^2) c+ \dots c+ Ker(\phi^m)=V heißt die \ul{kanonische Filtrierung} von V bzgl. \phi.
(b) Für i=1,\dots,m sei \delta_i=dim_K Ker(\phi^i)-dim_K(Ker(\phi^{i-1}).
\subsubsection{Lemma}
Für i \geq 1 gilt: \phi(Ker(\phi^i))\subseteq Ker(\phi^{i-1})\\
\proof
Dies folgt aus v\in Ker(\phi^i)\Rightarrow \phi^{i-1}(\phi(v))=0\Rightarrow \phi(v)\in Ker(\phi^{i-1}).
\qed
\subsubsection{Bemerkung}
Sei \phi nilpotent mit m=nix(\phi)=2.\\
Dann gilt: \{0\}c+ Ker(\phi) c+ Ker(\phi^2) =V. Wähle ein Komplement U_2 und Ker(\phi) in V und erhalte: V = Ker(\phi)\oplus U_2.
Nach dem Lemma gilt: \phi(Ker(\phi^1))\subseteq Ker(\phi^0)=\{0\} und \phi(U_2)=\phi(V)=\phi(Ker(\phi^2))\subseteq Ker(\phi).\\
Wähle nun ein Komplement U_1 von \phi(U_2) in Ker(\phi) und erhalte: v=Ker(\phi)\oplus U_2=U_1\oplus U_2\oplus \phi(U_2).\\
Wegen U_2\cap Ker(\phi)=\{0\} gilt dabei \phi EINGESCHR_{U_2} ist injektiv und somit dim_K(\phi(U-_2))=dim_K(U_2).\\
Also folgt: \delta_1=dim_K(Ker(\phi))=dim_K(U_1)+dim_K(\phi(U_2))\\
\delta_2=dim_K(V)-dim_K(Ker(\phi))=dim_K(U_2).
\subsubsection{Satz (Die Jordan-Zerlegung nilpotenter Endomorphismen)}
Sei \phi\in End_K(V) nilpotent und s=nix(\phi). Für i=1,\dots,s sei \delta_i=dim_K(Ker(\phi^i))-dim_K(Ker(\phi^{i-1})).\\
Konstruiere Untervektorräume U_1,\dots U_s von V rekursiv wie folgt:\\
(1) Sei U_s ein Komplement von Ker(\phi^{s-1}) in V.
(2) Absteigend für i=s-1, s-2,\dots, 1 sei U_i ein Komplement von Ker(\phi^{i-1})+\phi^{s-i}(U_s)+\phi^{s-i-1}(U_{s-1})+\dots+\phi^1(U_{i+1}) in Ker(\phi^i).\\
Dann gilt: (a) V=U_1\\
\oplus U_2\oplus \phi(U_2)\\
\oplus U_3\oplus \phi(U_3)\oplus \phi^2(U_3)\\
\vdots\\
\oplus U_s\oplus \phi(U_s)\oplus\dots\oplus\phi^{s-1}(U_s)\\
Diese Darstellung heißt die \ul{Jordan-Zerlegung} von V bzgl. \phi.
(b) Für i=1,\dots,s gilt: \delta_i=\sum_{j=i}^{s} dim_K(U_j) und \delta_1\geq\delta_2\geq\dots\geq\delta_s>0.\\
\ul{Beweis:} Vgl. Buch S.265-267.
\subsubsection{Korollar}
(a) Gilt \delta_1=\dots=\delta_s=1, so folgt U_1=\dots=U_{s-1}=\{0\} und V=U_s\oplus \phi(U_s)\oplus\phi^2(U_s)\oplus\dots\oplus\phi^{s-1}(U_s).\\
Dabei gilt: Ker(\phi^i)=\phi^{s-i}(U_s)\oplus\dots\oplus\phi^{s-1}(U_s).
(b) Gilt \delta_1=1, so folgt \delta_2=\dots=\delta_s=1. In diesem Fall liefert jeder Vektor v\in V\backslash Ker(\phi^{s-1}) ein Basistupel B=(v,\phi(v),\phi^2(v),\dots,\phi^{s-1}(v)).\\
In dieser Basis gilt M_B^B(\phi)=\begin{pmatrix}
0 & 0 & & 0
1 & 0 & \Huge 0 & \vdots
0 & 1 & & \vdots
\vdots & \Vdots & \ddots & 0
0 & 0 & \hdots & 1 & 0
\end{pmatrix}. TODO MATRIX SCHAUEN OB RICHTIG
Eine solche Matrix heißt ein \ul{Jordan-Kästchen}.\\
In der Basis B'=(\phi^{s-1}(v),\phi^{s-2}(v),\dots,v) gilt M_{B'}^{B'}(\phi)=TODO MATRIX (Dies ist das üblichere Jordan-Kästchen).\\
\proof
"(a)" Nach Teil (b) des Satzes folgt U_1=\dots=U_{s-1}=\{0\}. Damit vereinfacht sich die Jordan-Zerlegung wie angegeben.
"(b)" folgt aus Teil (b) des Satzes und (a).
\qed
\subsubsection{Bemerkung}
Sei \delta_s\geq 1 und \{v_1,\dots,v_\delta\} eine Basis von U_s. Dann ist B=\{v_1,\phi(v_1),\dotsm\phi^{s-1}(v_1)\}\caprev\dots\caprev\{v_\delta,\phi(v_\delta),\dotsm\phi^{s-1}(v_\delta)\} eine Basis von W=U_s\oplus\phi(U_s)\oplus\dots\oplus\phi^{s-1}(U_s). Die Matrix von \phi|W bzgl. B ist blockdiagonal mit \delta_s Jordan-Kästchen auf der Hauptdiagonalen M_B^B(\phi|W)=TODO MATRIX  TODO CAPREV u.\\
\subsubsection{Satz (Die Jordansche Normalform)}
Sei \phi\in End_K(V) mit linearen Eigenfaktoren, also mit \chi_\phi(x)=(x-\lambda_1)^{a_1}\cdot\dots\cdot(x-\lambda_s)^{a_s}. Dann gibt es eine Basis B in V, so dass gilt: M_B^B(\phi)=TODO MATRIX mit \lambda_i\cdot I_{a_i}+N_i=TODO MATRIX, wobei die Größe der \ul{Jordan-Blöcke} TODO BLOCK durch die Jordan-Zerlegung von (\phi-\lambda_iid_V)|G_i mit G_i=Gen(\phi,\lambda_i) gegeben ist.
\subsection{Diagonalisierbarkeit und Triagonalisierbarkeit (Buch: §.4.3-4.4)}
Sei K ein Körper, V ein endlich-dimensionaler K-Vektorraum, \phi\in End_K(V).\\
\phi heißt diagonalisierbar \Leftrightarrow es gibt eine Basis von V bestehend aus Eigenvektoren von \phi \Leftrightarrow es gibt eine Basis B von V, so dass M_B^B(\phi) eine Diagonalmatrix ist.
\subsubsection{Satz (Zweite Charakterisierung diagonalisierbarer Endomorphismen)}
Die folgenden Bedingungen sind äquivalent:
(a) \phi ist diagonalisierbar
(b) Alle Eigenfaktoren von \phi sind linear und Eig(\phi,\lambda)=Gen(\phi,\lambda) für alle Eigenwerte \lambda von \phi
(c) \mu_\phi(x) zerfällt in ein Produkt paarweise verschiedener Faktoren x-\lambda_i.\\
\proof
"(a)\Rightarrow(b)" Sei B eine Basis von V, so dass M_B^B(\phi) eine Diagonalmatrix ist. Dann ist \chi_\phi(x)=(x-\lambda_1)^{a_1}\cdot\dots\cdot(x-\lambda_s)^{a_s} mit \lambda_i\neq\lambda_j und a_i\geq 1 und M_B^B(\phi)=TODO MATRIX und B=B_1\caprevdisj\dots\caprevdisj B_s und Eig(\phi,\lambda_i)=<B_i>_K=Gen(\phi,\lambda_i)
"(b)\Rightarrow(c)" \ul{z.z.} m_i=1 für 1,\dots,s (vgl. Übungsblatt 3, Aufgabe 7).
"(c)\Rightarrow(a)" Schreibe \mu_\phi(x)=(x-\lambda_1)\cdot\dots\cdot(x-\lambda_s) mit \lambda_i\neq\lambda_j für i\neq j.\\
Nach Übungsblatt 3, Aufgabe 7 gilt Eig(\phi,\lambda_i)=Gen(\phi,\lambda_i) für i=1,\dots,s.\\
Dann folgt V=Gen(\phi,\lambda_1)\oplus\dots\oplus Gen(\phi,\lambda_s)=Eig(\phi,\lambda_1)\oplus\dots\oplus Eig(\phi,\lambda_s). Somit hat V eine Basis bestehend aus Eigenvektoren von \phi.
\qed
\subsubsection{Definition}
Der Endomorphismus \phi heißt \ul{triagonalisierbar}, wenn es eine Basis von V gibt, so dass M_B^B(\phi) eine obere Dreiecksmatrix ist.
\subsubsection{Satz (Charakterisierung triagonalisierbarer Endomorphismen)}
Genau dann ist \phi triagonalisierbar, wenn alle Eigenfaktoren von \phi linear sind.\\
\proof
"\Rightarrow" Sei B eine Basis von V bzgl. der M_B^B(\phi) eine obere Dreiecksmatrix ist. Schreibe M_B^B(\phi)=TODO MATRIX und erhalte \chi_\phi(x)=det(M_B^B(\phi))-x\cdot I_n)=(-1)^n(x-\lambda_1)\cdot\dots\cdot (x-\lambda_n).
"\Leftarrow" Dies folgt aus der Jordan-Zerlegung.
\qed
\end{document}



























