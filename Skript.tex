\documentclass[a4paper]{article}

\usepackage[ngerman]{babel}
\usepackage[utf8]{inputenc}
\usepackage{amsthm}
\usepackage{amsmath}
\usepackage{amssymb}
\usepackage{tikz,tkz-euclide}
\usepackage{titlesec}
\usepackage{textcomp}
\usepackage[titles]{tocloft}
\usepackage{csquotes}
\usepackage[
  separate-uncertainty = true,
  multi-part-units = repeat
]{siunitx}

\usetkzobj{all}
\usetikzlibrary{shapes.misc}

\MakeOuterQuote{"}

\newcommand*\circled[1]{%
  \tikz[baseline=(C.base)]\node[draw,circle,inner sep=0.75pt](C) {#1};\!
}

\renewcommand{\thesubsection}{\arabic{subsection}}
\titleformat{\section}{\normalfont\Large\bfseries}{Kapitel \arabic{section}: }{0em}{}
\titleformat{\subsection}{\normalfont\large\bfseries}{§\arabic{subsection} }{0em}{}
\titleformat{\subsubsection}{\normalfont\bfseries}{\arabic{subsection}.\arabic{subsubsection} }{0em}{}
\renewcommand{\cftsubsecpresnum}{§}
\newlength\mylength
\settowidth\mylength{\cftsubsecpresnum}
\settowidth\mylength{\cftsubsecaftersnum}
\addtolength\cftsubsecnumwidth{\mylength}
\renewcommand{\cftsecpresnum}{Kapitel }
\renewcommand{\cftsecaftersnum}{: }
\settowidth\mylength{\cftsecpresnum}
\addtolength\cftsecnumwidth{\mylength}

\newcommand{\ul}{\underline}
\renewcommand{\qed}{\begin{flushright}
\ul{\(q.e.d.\)}
\end{flushright}}
\let\origphi\phi
\let\phi\varphi

\title{Lineare Algebra II: Skript}
\author{Nico Mexis}
\date{\today}

\begin{document}
\maketitle
\newpage

\tableofcontents
\newpage

\setcounter{section}{4}
\section{Endomorphismen}
\setcounter{subsection}{17}
\subsection{Eigenwerte (Buch: §4.1-4.2)}
Sei \(K\) ein Körper.\\
Sei \(V\) ein endlich dimensionaler \(K\)-Vektorraum.\\
Sei \(\phi: V \rightarrow V\) ein Endomorphismus.
\subsubsection{Definition}
\begin{itemize}
\item[\circled{a}] Ein Element \(\lambda \in K\) heißt ein \ul{Eigenwert} von \(\phi\), wenn es einen Vektor \(v \in V\ \backslash \{0\}\) gibt mit \(\phi(v)=\lambda*v\).
\item[\circled{b}] Ist \(\lambda \in K\) ein Eigenwert von \(\phi\), so heißt jeder Vektor \(v \in V \backslash \{0\}\) mit \(\phi(v) = \lambda*v\) ein Eigenvektor von \(\phi\) zum Eigenwert \(\lambda\).
\item[\circled{c}] Ist \(A \in Mat_n(K)\), so heißt ein \(\lambda \in K\) ein \ul{Eigenwert} von A, wenn es ein \(v \in K^n\backslash\{0\}\) gibt mit \(A*v=\lambda*v\).
\end{itemize}
\subsubsection{Beispiel}
\begin{itemize}
\item[\circled{a}] Das Element \(0 \in K\) ist ein Eigenwert von \(\phi\), wenn \(\phi\) nicht injektiv ist.
\item[\circled{b}] Das Element \(1 \in K\) ist ein Eigenwert von \(\phi\), wenn \(\phi\) einen \ul{Fixpunkt} \(v \neq 0\) hat (d.h. \(\phi(v)=v\)).
\end{itemize}
\subsubsection{Beispiel}
Sei \(K = \mathbb{R}, V= \mathbb{R}^2\).\\
Sei \(\phi:\mathbb{R}^2\rightarrow\mathbb{R}^2\) die Drehung um \(0=(0,0)\) um den Winkel \(\alpha \in [0,2\pi[\).\\
\begin{itemize}
\item[\circled{a}] Ist \(\alpha = 0\), so ist \(\phi = id_{\mathbb{R}^2}\) und \(\lambda = 1\) ist der einzige Eigenwert von \(\phi\).
\item[\circled{b}] Ist \(\alpha = \pi\), so ist \(\phi = -id_{\mathbb{R}^2}\) und \(\lambda = -1\) ist der einzige Eigenwert von \(\phi\).
\item[\circled{c}] Ist \(\alpha \notin \{0,\pi\}\), so besitzt \(\phi\) keine Eigenwerte.
\end{itemize}
\subsubsection{Beispiel}
Sei \(K = \mathbb{R}, V= \mathbb{R}^2\) und \(\sigma :\mathbb{R}^2 \rightarrow\mathbb{R}^2\) die Spiegelung an der Geraden \(G\) durch \(0\), die mit der x-Achse einen Winkel \(\frac{\alpha}{2}\) einschließt.\newpage
\ul{Skizze:}\\
\begin{tikzpicture}
  \tkzInit[ymin=0,ymax=1.11,xstep=0.1,ystep=0.1,xmin=0,xmax=1.11]
  \tkzGrid
  \tkzLabelX
  \tkzLabelY
  \tkzDrawX[below=12 pt,label={x}]
  \tkzDrawY[left=12 pt,label={y}]
  \tkzDefPoint(1,0){V}
  \tkzDefPoint(0,0){S}
  \tkzDefPoint(0.9,0.9){P}
  \tkzDefPoint(1, 0.75){T}
  \tkzDefPoint(0.27, 0.96){U}
  \tkzDefPoint(0.64, 0.48){E}
  \tkzSetUpPoint[size = 10]
  \tkzDrawPoints[fill=black](V,U)
  \tkzLabelPoint[above](U){\((cos(\alpha),sin(\alpha))\)}
  \tkzLabelPoint[above right](V){\((1,0)\)}
  \tkzDrawSegment[ultra thick,black](S,T)
  \tkzDrawSegment[thick,black](U,V)
  \tkzDrawSegment[thick,black](S,U)
  \tkzMarkAngle(T,E,U)
  \tkzLabelAngle[pos=0.55](T,E,U){\(\bullet\)}
  \tkzMarkAngle(T,S,U)
  \tkzLabelAngle[pos=0.75](T,S,U){\(\frac{\alpha}{2}\)}
  \tkzMarkAngle(V,S,T)
  \tkzLabelAngle[pos=0.75](V,S,T){\(\frac{\alpha}{2}\)}
  \tkzLabelPoint[above](P){\(\begin{pmatrix}
cos(\alpha) & sin(\alpha) \\
sin(\alpha) & cos(\alpha)
\end{pmatrix}\)}
\end{tikzpicture}\\\\
Ein Eigenwert ist \(\lambda=1\) und die Menge der Eigenvektoren zum Eigenwert \(\lambda=1\) ist \(G\backslash\{0\}\).\\
Ein weiterer Eigenwert ist \(\lambda=-1\) und die Menge der Eigenvektoren zum Eigenwert \(\lambda=-1\) ist \(H\backslash\{0\}\), wobei \(H\) die zu \(G\) senkrechte Gerade durch 0 ist.
\subsubsection{Definition}
Sei \(\lambda \in K\) ein Eigenwert von \(\phi\). Dann heißt \(Eig(\phi,\lambda)=\{v \in V | \phi (v)=\lambda *v\}\) der \ul{Eigenraum} zum Eigenwert \(\phi\).
\subsubsection{Satz}
Sei \(\phi:V \rightarrow V\) ein Endomorphismus.
\begin{itemize}
\item[\circled{a}] Ist \(\lambda \in K\) ein Eigenwert von \(\phi\), so ist \(Eig(\phi,\lambda)\) ein Untervektorraum von V.
\item[\circled{b}] Sind \(\lambda,\mu \in K\) zwei verschiedene Eigenwerte von \(\phi\), so gilt: \(Eig(\phi,\mu)=\{0\}\).
\end{itemize}
\newpage
\ul{Beweis:}
\begin{itemize}
\item["\circled{a}"] Wegen \(\phi(0)=\lambda*0=0\) gilt \(\phi(v-w)=\phi(v)-\phi(w)=\lambda*v-\lambda*w=\lambda*(v-w)\), also \(v-w \in Eig(\phi,\lambda)\). Nach dem Untervektorraumkriterium folgt die Behauptung.
\item["\circled{b}"] Sei \(v \in Eig(\phi,\lambda)\cap Eig(\phi,\mu)\). Dann gilt \(\lambda*v=\phi(v)=\mu*v\), also \(\underbrace{(\lambda-\mu)}_{\neq 0}*v=0\). Dies liefert \(v=0\).
\end{itemize}
\qed
\subsubsection{Beispiel}
\begin{itemize}
\item[\circled{a}] Sei \(K = \mathbb{R}\), \(V = \mathbb{R}^2\) und \(\phi:\mathbb{R}^2\rightarrow \mathbb{R}^2\) die Spiegelung an der Winkelhalbierenden. Dann gilt \(M_{\xi}^{\xi}(\phi)=\begin{pmatrix}
0 & 1 \\
1 & 0
\end{pmatrix}\) und die Eigenwerte von \(\phi\) sind \(\lambda = \pm 1\)
\begin{itemize}
\item[\circled{1}] \(Eig(\phi,1)=\mathbb{R}*(1,1)\)
\item[\circled{2}] \(Eig(\phi,-1)=\mathbb{R}*(1,-1)\)
\end{itemize}
\item[\circled{b}] Sei \(K = \mathbb{R}\), \(V = \mathbb{R}^3\) und \(\phi:\mathbb{R}^3\rightarrow \mathbb{R}^3\) die Drehung um die z-Achse um einen Winkel \(\alpha \in ]0,\pi[\). Dann gilt \(M_{\xi}^{\xi}(\phi)=\begin{pmatrix}
cos(\alpha) & -sin(\alpha) & 0 \\
sin(\alpha) & cos(\alpha) & 0 \\
0 & 0 & 1
\end{pmatrix}\) und \(\lambda = 1\) ist der einzige Eigenwert mit \(Eig(\phi,1)=\mathbb{R}*\underbrace{(0,0,1)}_{z-Achse}\).
\end{itemize}
\subsubsection{Definition}
\begin{itemize}
\item[\circled{a}] Der Endomorphismus \(\phi:V\rightarrow V\) heißt \ul{diagonalisierbar}, wenn es eine Basis \(B\neq (v_1,...,v_d)\) von \(V\) gibt, die aus Eigenvektoren besteht.\\
Sei \(\phi(v_i)=a_iv_i\) mit \(a_i \in K\) für \(i=1,..,d\). Dann gilt also \(M^B_B(\phi)=\begin{pmatrix}
a_1 & 0 & \hdots & 0 \\
0 & a_2 & \hdots & 0 \\
\vdots & \vdots & \ddots & \vdots \\
0 & 0 & \hdots & a_n
\end{pmatrix}\) Diagonalmatrix.
\item[\circled{b}] Eine Matrix \(A \in Mat_n(K)\) heißt \ul{diagonalisierbar}, wenn der zugehörige Endomorphismus \(\phi_A:K^n\rightarrow K^n\) diagonalisierbar ist, d.h. wenn es ein \(T \in GL_n(K)\) gibt, sodass \(TAT^{-1}\) eine Diagonalmatrix ist.
\end{itemize}
\subsubsection{Satz (Charakterisierung der Diagonalisierbarkeit)}
Für einen Endomorphismus \(\phi:V\rightarrow V\) sind die folgenden Bedingungen äquivalent:
\begin{itemize}
\item[\circled{a}] \(\phi\) ist diagonalisierbar.
\item[\circled{b}] Es gibt eine Basis \(B\) von \(V\), sodass \(M_B^B(\phi)\) eine Diagonalmatrix ist.
\item[\circled{c}] Für jede Basis \(B\) von \(V\) ist \(M_B^B(\phi)\) eine diagonalisierbare Matrix.
\item[\circled{d}] Es gibt eine Basis \(B\) von \(V\), die aus Eigenvektoren von \(\phi\) besteht.
\end{itemize}
\ul{Beweis:}
\begin{itemize}
\item["\circled{a} \textrightarrow \space\circled{b}"] Def. 18.8a
\item["\circled{b} \textrightarrow \space\circled{c}"] Sei \(B\) eine Basis von \(V\), sodass \(M_B^B(\phi)\) eine Diagonalmatrix ist und sei \(C\) eine weitere Basis von \(V\). Dann gilt: \(M_B^B(\phi)=T_C^B*M_C^C(\phi)*(T_C^B)^{-1}\) mit \(T_C^B\in GL_n(K)\). Also ist \(M_C^C(\phi)\) diagonalisierbar.
\item["\circled{c} \textrightarrow \space\circled{d}"] Nach Def. 18.8b gibt es \(T \in GL_n(K)\) mit \(T*M_C^C(\phi)*T^{-1}\) Diagonalmatrix. Bzgl. der transponierten Basis \(B=T*C\) ist also \(M_B^B(\phi)\) eine Diagonalmatrix, d.h. \(B\) besteht aus Eigenvektoren von \(\phi\).
\item["\circled{d} \textrightarrow \space\circled{a}"] Nach Def.
\end{itemize}
\qed
\subsubsection{Beispiel}
Eigenwerte und Eigenvektoren spielen eine große Rolle beim Lösen von \ul{Differentialgleichungen}.\\
Suche z.B. im \(\mathcal{C}^1(\mathbb{R})\) die Lösungen von\\
\(f_1'(t)=f_1(t)-f_2(t)\)\\
\(f_2'(t)=2f_1(t)+4f_2(t)\).\\
Man macht folgenden Lösungsansatz: \(f_1(t)=c_1*e^{\lambda t}\) und \(f_2(t)=c_2*e^{\lambda t}\)\\
\(\Rightarrow \begin{pmatrix}
f_1'(t) \\
f_2'(t)
\end{pmatrix} = \begin{pmatrix}
\lambda c_1e^{\lambda t} \\
\lambda c_2e^{\lambda t}
\end{pmatrix} = \lambda * \begin{pmatrix}
f_1(t) \\
f_2(t)
\end{pmatrix}\).\\
Das Differentialgleichungssystem lautet also:\\
\(\begin{pmatrix}
f_1'(t) \\
f_2'(t)
\end{pmatrix} = \lambda * \begin{pmatrix}
f_1(t) \\
f_2(t)
\end{pmatrix} = \begin{pmatrix}
1 & -1 \\
2 & 4
\end{pmatrix} * \begin{pmatrix}
f_1(t) \\
f_2(t)
\end{pmatrix}\).\\
Im \(\mathbb{R}\)-Vektorraum \(\mathcal{C}^1(\mathbb{R})\) ist \(\begin{pmatrix}
f_1'(t) \\
f_2'(t)
\end{pmatrix}\) ein Eigenvektor der Matrix \(\begin{pmatrix}
1 & -1 \\
2 & 4
\end{pmatrix}\) zum Eigenwert \(\lambda\).
\subsubsection{Satz}
Ist \(\lambda \in K\) ein Eigenwert von \(\phi\), so gilt: \(Eig(\phi, \lambda)=Ker(\phi-\lambda*id_V)\).\\
\ul{Beweis:}\\
Es gilt \(v \in Eig(\phi, \lambda) \Leftrightarrow \phi(v)=\lambda *v \Leftrightarrow(\phi-\lambda*id_V)(v)=0 \Leftrightarrow v\in Ker(\phi-v*id_V)\).
\qed
\subsubsection{Definition}
Sei \(x\) eine Unbestimmte und \(\mathcal{A} = (a_{ij}) \in Mat_n(K)\). Dann gilt: \(\mathcal{A}-x*\mathcal{I}_n=\begin{pmatrix}
a_{11}-x & a_{12} & \hdots & a_{1n} \\
a_{21} & a_{22}-x & \hdots & a_{2n} \\
\vdots & \vdots & \ddots & \vdots \\
a_{n1} & a_{n2} & \hdots & a_{nn}-x
\end{pmatrix} \in Mat_n(K[x])\).\\
Dann heißt \(det(\mathcal{A}-x*\mathcal{I}_n)\in K[x]\) das \ul{charakteristische Polynom} von \(\mathcal{A}\) und wird mit \(\underset{"chi"}{\chi_A}(x)\) bezeichnet.
\subsubsection{Beispiel}
Sei \(\mathcal{A}=\begin{pmatrix}
0 & 1 \\
1 & 0
\end{pmatrix} \in Mat_2(\mathbb{Q})\). Dann gilt \(\chi_A(x)=det\begin{pmatrix}
-x & 1 \\
1 & -x
\end{pmatrix}=x^2-1\).
\subsubsection{Satz}
Seien \(B,C\) zwei Basen von \(V\). Dann gilt:\\
\(\chi_{M_B^B(\phi)}(x)=\chi_{M_C^C(\phi)}(x)\)\\
\ul{Beweis:}\\
Nach der Transformationsformel gilt:\\
\(M_C^C(\phi)=T_C^B*M_B^B(\phi)*T_B^C\)\\
Es folgt: \(\chi_{M_C^C(\phi)}(x)=det(T_C^BM_B^B(\phi)T_B^C-\underbrace{T_C^Bx\mathcal{I}_nT_B^C}_{=x\mathcal{I}_n})\\=det(T_C^B(M_B^B(\phi)-x\mathcal{I}_n)T_B^C)\\=det(T_C^B)*det(M_B^B(\phi)-x*\mathcal{I}_n)*det(T_C^B)^{-1}=\chi_{M_B^B(\phi)}(x)\).
\qed
\ul{Folgerung:} Das Polynom \(\chi_{M_B^B(\phi)}(x)\) hängt nur von \(\phi\), aber nicht von der Wahl der Basis \(B\) ab. Es heißt das \ul{charakteristische Polynom} von \(\phi\) und wird mit \(\chi_\phi(x)\) bezeichnet.
\subsubsection{Satz}
Ein Element \(\lambda \in K\) ist genau dann ein Eigenwert von \(\phi\), wenn \(\chi_{\phi}(\lambda)=0\) gilt.\\
\ul{Beweis:}\\
Genau dann ist \(\lambda \in K\) ein Eigenwert von \(\phi\), wenn \(Ker(\phi-\lambda id_V)\neq \{0\}\) gilt. Dies ist genau dann der Fall, wenn \(\phi-\lambda id_V\) nicht bijektiv ist. Letztere Bedingung ist äquivalent zu \(det(\phi-\lambda id_V)=0\), also zu \(\chi_\phi(\lambda)=0\).
\qed
\subsubsection{Beispiel}
\begin{itemize}
\item[\circled{a}] Sei \(\phi:\mathbb{R}^2\rightarrow\mathbb{R}^2\) die Spiegelung an der Winkelhalbierenden \(G=\mathbb{R}*\begin{pmatrix}
1 \\
1
\end{pmatrix}\). Dann gilt: \(M_\xi^\xi(\phi)=\begin{pmatrix}
0 & 1 \\
1 & 0
\end{pmatrix}\) und  \(\chi_\phi(x)=x^2-1=(x-1)(x+1)\). Also besitzt \(\phi\) zwei Eigenwerte, nämlich \(\lambda_1=1\) und \(\lambda_2=-1\).\\
Für \(\lambda_1=1\) gilt \(Eig(\phi,\lambda_1)=Ker(\phi-\lambda_1 id_{\mathbb{R}^2})=Ker\begin{pmatrix}
-1 & 1 \\
1 & -1
\end{pmatrix} = \textless\begin{pmatrix}
1 \\
1
\end{pmatrix}\textgreater = G\).\\
Für \(\lambda_2=-1\) gilt \(Eig(\phi,\lambda_2)=Ker(\phi-\lambda_2 id_{\mathbb{R}^2})=Ker\begin{pmatrix}
1 & 1 \\
1 & 1
\end{pmatrix} = \textless\begin{pmatrix}
-1 \\
1
\end{pmatrix}\textgreater = G^\bot\).
\item[\circled{b}] Sei \(\psi:\mathbb{R}^2\rightarrow\mathbb{R}^2\) die Drehung um \(0\) um den Winkel \(\alpha \in ]0,\pi[\). Dann gilt \(M_\xi^\xi(\psi)=\begin{pmatrix}
cos(\alpha) & -sin(\alpha) \\
sin(\alpha) & cos(\alpha)
\end{pmatrix}\).\\
\(\chi_\psi(x)=det\begin{pmatrix}
cos(\alpha)-x & -sin(\alpha) \\
sin(\alpha) & cos(\alpha)-x
\end{pmatrix}=(cos(\alpha)-x)^2+sin^2(\alpha)=x^2-2cos(\alpha)x+sin^2(\alpha)+cos^2(\alpha)=x^2-2cos(\alpha)x+1\).\\
Dieses quadratische Polynom hat die Diskriminante \(\triangle=4cos^2(\alpha)-4=-4sin^2(\alpha)<0\). Somit besitzt \(\psi\) keine Eigenwerte.
\end{itemize}
\subsubsection{Bemerkung}
Ein Polynom \(f \in K[x]\backslash K\) braucht keine Nullstelle in \(K\) zu besitzen (z.B. \(x^2+1 \in \mathbb{R}[x])\).\\
Der \ul{Fundamentalsatz der Algebra} besagt, dass jedes Polynom \(f \in \mathbb{C} [x] \backslash \mathbb{C}\) eine Nullstelle besitzt und somit in ein Produkt von Linearfaktoren zerfällt.
\subsubsection{Satz (Eigenschaften des charakteristischen Polynoms)}
Sei \(\phi:V\rightarrow V\ K\)-linear und \(n=dim_K(V)\). Dann gilt:
\begin{itemize}
\item[\circled{a}] \(deg(\chi_\phi(x))=n\).
\item[\circled{b}] Der Gradkoeffizient von \(\chi_\phi(x)\) ist \((-1)^n\), also \(\chi_\phi(x)=(-1)^nx^n+(Monome\ kleineren\ Grades)\)
\item[\circled{c}] Der konstante Term von \(\chi_\phi(x)\) ist \(det(\phi)=\chi_\phi(0)\).
\end{itemize}
\ul{Beweis:}\\
\begin{itemize}
\item["\circled{a},\circled{b}"] Sei \(\mathcal{A}=(a_{ij})\in Mat_n(K)\) eine Darstellungsmatrix von \(\phi\). Dann gilt \(\chi_\phi(x)=det\begin{pmatrix}
a_{11}-x & & * \\
 & \ddots & \\
 * & & a_{nn}-x
\end{pmatrix}=(a_{11}-x)(a_{22}-x)*...*(a_{nn}-x)+(Monome\ niedriegeren\ Grades\ als\ n)=(-x)^n+(Monome\ niedriegeren\ Grades\ als\ n)\)
\item["\circled{c}"] \(\chi_\phi(0)=det(\phi-0*id_V)=det(\phi)\).
\end{itemize}
\qed
\subsubsection{Bemerkung (Der zweite Koeffizient von \(\chi_\phi(x)\))}
Schreibe \(\chi_\phi(x)=(-1)^nx^n+c_{n-1}x^{n-1}+...+c_1x+c_0\). Dann heißt \(Spur(\phi)=\underset{"Trace"}{Tr}(\phi)=(-1)^{n-1}c_{n-1}\) die \ul{Spur} von \(\phi\).\\
Sie hängt nicht ab von der Wahl einer Darstellungsmatrix \(\mathcal{A}=(a_{ij})\). Es gilt \(Spur(\mathcal{A})=a_{11}+a_{22}+...+a_{nn}\).\\
\ul{Beweis:}\\
Um einen Summanden \(c*x^{n-1}\) in \(det\begin{pmatrix}
a_{11}-x & & & * \\
 & a_{22}-x & & \\
 & & \ddots & \\
* & & & a_{nn}-x
\end{pmatrix}\) zu erhalten, muss man genau \(n-1\)mal den Faktor \(a_{ii}-x\) wählen. Also suchen wir den Koeffizienten von \(x^{n-1}\) in \((a_{11}-x)*...*(a_{nn}-x)\). Beim Ausmultiplizieren muss man \((n-1)\)mal \(-x\) wählen und einmal \(a_{ii}\). Also gilt \(c=(-1)^{n-1}(a_{11}+a_{22}+...+a_{nn})=(-1)^{n-1} Spur(\phi)\).
\qed
\end{document}






























