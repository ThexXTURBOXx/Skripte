\documentclass[a4paper]{article}

\usepackage[ngerman]{babel}
\usepackage[utf8]{inputenc}
\usepackage{amsthm}
\usepackage{amsmath}
\usepackage{amssymb}
\usepackage{tikz,tkz-euclide}
\usepackage{titlesec}
\usepackage{textcomp}
\usepackage[titles]{tocloft}
\usepackage{csquotes}
\usepackage[
  separate-uncertainty = true,
  multi-part-units = repeat
]{siunitx}

\usetkzobj{all}
\usetikzlibrary{shapes.misc}

\MakeOuterQuote{"}

\newcommand*\circled[1]{%
  \tikz[baseline=(C.base)]\node[draw,circle,inner sep=0.75pt](C) {#1};\!
}

\renewcommand{\thesubsection}{\arabic{subsection}}
\titleformat{\section}{\normalfont\Large\bfseries}{Kapitel \arabic{section}: }{0em}{}
\titleformat{\subsection}{\normalfont\large\bfseries}{§\arabic{subsection} }{0em}{}
\titleformat{\subsubsection}{\normalfont\bfseries}{\arabic{subsection}.\arabic{subsubsection} }{0em}{}
\renewcommand{\cftsubsecpresnum}{§}
\newlength\mylength
\settowidth\mylength{\cftsubsecpresnum}
\settowidth\mylength{\cftsubsecaftersnum}
\addtolength\cftsubsecnumwidth{\mylength}
\renewcommand{\cftsecpresnum}{Kapitel }
\renewcommand{\cftsecaftersnum}{: }
\settowidth\mylength{\cftsecpresnum}
\addtolength\cftsecnumwidth{\mylength}

\newcommand{\ul}{\underline}
\renewcommand{\qed}{\begin{flushright}
\ul{\(q.e.d.\)}
\end{flushright}}
\let\origphi\phi
\let\phi\varphi

\title{Lineare Algebra II\\Übungsblatt 1}
\author{Nico Mexis}
\date{\today}

\begin{document}
\maketitle
\newpage

\section*{Aufgabe 1)}
Sei \(K\) ein Körper, sei \(V\) ein endlich-dimensionaler \(K\)-Vektorraum und sei \(\phi:V\rightarrow V\) ein nilpotenter Endomorphismus, d.h. es gebe ein \(n \geq 1\) mit \(\phi^n = 0\). Finden Sie alle Eigenwerte von \(\phi\).\\\\
\ul{Lösung:}\\
Sei \(\lambda\) Eigenwert von \(\phi\) mit Eigenvektor \(v\).\\
\(\phi^n(v)=\phi^{n-1}(\phi(v))=\phi^{n-1}(\lambda\cdot v)=\lambda\cdot\phi^{n-1}(v)=...=\lambda^n\cdot v\).\\
Demnach gilt, falls \(\phi\) nilpotent ist:\\
\(\lambda^nv=\phi^n(v)=0\)\\
\(\Leftrightarrow \lambda\cdot v=0 \underbrace{\Leftrightarrow}_{v\neq 0}\lambda = 0\)\\
Somit ist \(0\) der einzige Eigenwert von \(\phi\).
\section*{Aufgabe 2)}
Sei \(K\) ein Körper, sei \(V\) ein endlich-dimensionaler \(K\)-Vektorraum und sei \(\phi:V\rightarrow V\) ein Endomorphismus, für den \(\phi^2+\phi\) den Eigenwert \(-1\) besitzt. Beweisen Sie, dass \(\phi^3\) dann den Eigenwert \(1\) besitzt.\\\\
\ul{Lösung:}\\
\(-1\cdot v=(\phi^2+\phi)(v)=\phi^2(v)+\phi(v)\\
\Leftrightarrow \phi(\phi^2(v)+\phi(v))=\phi(-v)\\
\Leftrightarrow \phi(\phi^2(v))+\phi(\phi(v))=-\phi(v)\\
\Leftrightarrow \phi^3(v)+\phi^2(v)=-\phi(v)\\
\Leftrightarrow \phi^3(v)=-\phi(v)-\phi^2(v)\\
\Leftrightarrow \phi^3(v)=-(\phi(v)+\phi^2(v))\\
\Leftrightarrow \phi^3(v)=-(-v)\\
\Leftrightarrow \phi^3(v)=1\cdot v\)\\
Also hat \(\phi^3\) den Eigenwert \(1\).
\qed
\section*{Aufgabe 3)}
Sei \(K\) ein Körper, sei \(V\) ein endlich-dimensionaler \(K\)-Vektorraum und seien \(\phi,\psi:V\rightarrow V\) zwei Endomorphismen  von \(V\). Beweisen Sie folgende Aussagen:
\subsection*{3a)}
Ist \(\lambda \in K\) ein Eigenwert von \(\psi\circ\phi\), so ist \(v\in V\backslash\{0\}\) ein Eigenvektor zum Eigenwert \(\lambda\) von \(\psi\circ\phi\) und gilt \(\phi(v)\neq 0\), so ist \(\phi(v)\) ein Eigenvektor von \(\phi\circ\psi\) zum Eigenwert \(\lambda\).\\\\
\ul{Lösung:}\\
\((\psi\circ\phi)(v)=\lambda v\\
\Leftrightarrow \phi((\psi\circ\phi)(v))=\phi(\lambda v)\\
\Leftrightarrow (\phi\circ\psi\circ\phi)(v)=\lambda\phi(v)\\
\Leftrightarrow (\phi\circ\psi)(\ul{\phi(v)})=\lambda\ul{\phi(v)}\).\\
Somit ist \(\phi(v)\) Eigenvektor von \(\phi\circ\psi\) zum Eigenwert \(\lambda\).
\qed
\subsection*{3b)}
Die beiden Endomorphismen \(\phi\circ\psi\) und \(\psi\circ\phi\) besitzen dieselben Eigenwerte.\\\\
\ul{Lösung:}\\
\((\phi\circ\psi)(v)=\lambda v\\
\Leftrightarrow \psi((\phi\circ\psi)(v))=\psi(\lambda v)\\
\Leftrightarrow (\psi\circ\phi\circ\psi)(v)=\lambda\psi(v)\\
\Leftrightarrow (\psi\circ\phi)(\psi(v))=\lambda\psi(v)\\
\Leftrightarrow (\psi\circ\phi)(w)=\lambda w\).\\
Somit ist \(\lambda\) Eigenwert von \(\phi\circ\psi\) und \(\psi\circ\phi\).
\qed
\section*{Aufgabe 4)}
Sei \(A\in Mat_2(\mathbb{R})\) eine symmetrische Matrix, d.h. es gelte \(A^{tr}=A\). Zeigen Sie, dass \(A\) einen Eigenwert besitzt.\\\\
\ul{Lösung:}\\
\(A=\begin{pmatrix}
a & b \\
b & c
\end{pmatrix}=A^{tr}\\
det\begin{pmatrix}
a-x & b \\
b & c-x
\end{pmatrix}=(a-x)(c-x)-b^2=x^2-ax-cx+ac-b^2\\
=x^2+(-a-c)x+(ac-b^2)\\
x_{1/2}=\frac{a+c\pm\sqrt{(-a-c)^2-4\cdot 1\cdot (ac-b^2)}}{2\cdot 1}\\
=\frac{a+c\pm\sqrt{a^2+2ac+c^2-4ac+4b^2}}{2}\\
=\frac{a+c\pm\sqrt{a^2-2ac+c^2+4b^2}}{2}\\
=\frac{a+c\pm\sqrt{(a-c)^2+4b^2}}{2}\)\\
Die Diskriminante lautet demnach \(\triangle=\underbrace{(a-c)^2}_{\geq0}+\underbrace{4b^2}_{\geq0}\).\\
Es existiert also ein Eigenwert genau dann, wenn der Nenner \(2\neq0\) ist, was offensichtlich gilt, und die Diskriminante \(\triangle\geq0\) ist.\\
\((a-c)^2+4b^2\) muss \(\geq0\) sein, da \(a,b,c\in \mathbb{R}\) gilt und somit \((a-c)^2\in\mathbb{R}_{\geq0}\) und \(4b^2\in\mathbb{R}_{\geq0}\) und somit auch \((a-c)^2+4b^2\in\mathbb{R}_{\geq0}\) gilt.
\qed
\section*{Aufgabe 5)}
Sei \(K\) ein Körper, sei \(A\in Mat_n(K)\), sei \(V=Mat_n(K)\), und sei \(\phi:V\rightarrow V\) der Endomorphismus, der gegeben ist durch \(\phi(B)=A\cdot B\) für jedes \(B\in V\). Beweisen Sie, dass \(\chi_\phi(z)=(\chi_A(z))^n\) gilt.\\\\
\ul{Lösung:}\\
\(\hspace*{212pt}j\)\\
\(\hspace*{212pt}\downarrow\)\\
Für die Standardbasismatrizen \(E_{ij}=\begin{pmatrix}
0 & & & &  \\
 & \ddots & & \mbox{\Large 0} & \\
 & & 1 & & \\
 & \mbox{\Large 0} & & \ddots & \\
 & & & & 0
\end{pmatrix}\leftarrow i\)\ \ \ gilt:\\
\(\phi(E_{ij})=A\cdot E_{ij}=\begin{pmatrix}
a_{11} & \hdots & \hdots & \hdots & a_{1n} \\
\vdots & \ddots & & & \vdots \\
\vdots & & \ddots & & \vdots \\
\vdots & & & \ddots & \vdots \\
a_{n1} & \hdots & \hdots & \hdots & a_{nn}
\end{pmatrix}
\begin{pmatrix}
0 & & & & \\
 & \ddots & & \mbox{\Large 0} & \\
 & & 1 & & \\
 & \mbox{\Large 0} & & \ddots & \\
 & & & & 0
\end{pmatrix}=\begin{pmatrix}
0 & \hdots & a_{1j} & \hdots & 0 \\
\vdots & & \vdots & & \vdots \\
\vdots & & \vdots & & \vdots \\
\vdots & & \vdots & & \vdots \\
0 & \hdots & a_{1j} & \hdots & 0
\end{pmatrix}=\sum_{k=1}^{n}a_{kj}E_{kj}\).\\
Wir wählen die Basis \(B=(E_{11},E_{12}...,E_{nn})\) und erhalten \(M_B(\phi)=\begin{pmatrix}
A & & 0 \\
 & \ddots & \\
0 & & A
\end{pmatrix}\).
\(\Rightarrow\chi_\phi(c)=det(\phi-x\cdot id_V)=det(M_B(\phi)-x\cdot \mathcal{I}_{n^2})=\\
det\begin{pmatrix}
A-x\cdot \mathcal{I}_n & & 0 \\
 & \ddots & \\
0 & & A-x\cdot \mathcal{I}_n
\end{pmatrix}=det(A-x\cdot \mathcal{I}_n)^n=\chi_A(x)^n\).
\qed
\end{document}






























