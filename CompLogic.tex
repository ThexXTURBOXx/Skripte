% !TeX spellcheck = en_US
\documentclass[a4paper]{article}

\usepackage[english]{babel}
\usepackage[utf8]{inputenc}
\usepackage{amsthm}
\usepackage{amsmath}
\usepackage{amssymb}
\usepackage{tikz,tkz-euclide}
\usepackage{titlesec}
\usepackage{gensymb}
\usepackage{textcomp}
\usepackage[titles]{tocloft}
\usepackage{csquotes}
\usepackage[babel]{microtype}
\usepackage{MnSymbol}
\usepackage{stmaryrd}
\usepackage{mathtools}
\usepackage{ulem}
\usepackage[shortlabels]{enumitem}
\usepackage{scalerel}
\usepackage{stackengine}
\usepackage[
  separate-uncertainty = true,
  multi-part-units = repeat
]{siunitx}

\usetkzobj{all}
\usetikzlibrary{shapes.misc}

\MakeOuterQuote{"}

\setcounter{secnumdepth}{4}

\renewcommand\hateq{\mathrel{\stackon[1.5pt]{=}{\stretchto{%
				\scalerel*[\widthof{=}]{\wedge}{\rule{1ex}{3ex}}}{0.5ex}}}}

\newcommand*\circled[1]{
  \tikz[baseline=(C.base)]\node[draw,circle,inner sep=0.75pt](C) {#1};\!
}

\newcommand*{\obot}{\perp\mkern-20.7mu\bigcirc}

\DeclarePairedDelimiter\abs{\lvert}{\rvert}
\DeclarePairedDelimiter\norm{\lVert}{\rVert}
\makeatletter
\let\oldabs\abs
\def\abs{\@ifstar{\oldabs}{\oldabs*}}
\let\oldnorm\norm
\def\norm{\@ifstar{\oldnorm}{\oldnorm*}}
\makeatother

\renewcommand{\thesubsection}{\arabic{subsection}}
\titleformat{\section}{\normalfont\Large\bfseries}{§\arabic{section}: }{0em}{}
\titleformat{\subsection}{\normalfont\large\bfseries}{§\arabic{subsection} }{0em}{}
\titleformat{\subsubsection}{\normalfont\bfseries}{\arabic{subsection}.\arabic{subsubsection} }{0em}{}
\renewcommand{\cftsubsecpresnum}{§}
\newlength\mylength
\settowidth\mylength{\cftsubsecpresnum}
\settowidth\mylength{\cftsubsecaftersnum}
\addtolength\cftsubsecnumwidth{\mylength}
\renewcommand{\cftsecpresnum}{§}
\renewcommand{\cftsecaftersnum}{: }
\settowidth\mylength{\cftsecpresnum}
\addtolength\cftsecnumwidth{\mylength}

\newcommand{\ul}{\underline}
\renewcommand{\proof}{\ul{Beweis:}\\}
\renewcommand{\qed}{\begin{flushright}
\ul{\(q.e.d.\)}
\end{flushright}}
\let\origphi\phi
\let\phi\varphi
\let\origepsilon\epsilon
\let\epsilon\varepsilon

\title{Computational Logic}
\author{Nico Mexis}
\date{\today}

\begin{document}
\maketitle
\newpage

\tableofcontents
\newpage

\section{What is logic?}
\ul{Logical Statement:} Sentence/proposition, which evaluates to $TRUE$ or $FALSE$\\
\ul{Logical connectives:} If $A$ and $B$ are logical statements, then $A\wedge B$ is another one\\
\ul{Logical system:}
\begin{enumerate}[(1)]
	\item \ul{Syntax:}
	\begin{itemize}
		\item Atomic formulas are connected by operations
		\item There is a set of rules, which tell you, if the formula is well-formed
	\end{itemize}
	\item \ul{Semantics:}
	\begin{itemize}
		\item There is a set of rules, which tell you, how to assign truth values to formulas, given the values of the atomic formulas
	\end{itemize}
	\item \ul{Logical calculus:}
	\begin{itemize}
		\item Always $TRUE$: Tautology
		\item Always $FALSE$: Unsatisfiable
		\item Sometimes $TRUE$: Satisfiable
		\item There is a set of rules such that, given a formula $F$ and an assignment for the truth values of the atomic formulas, it calculates the truth value of $F$.
		\item The algorithm should be able to decide if $F$ is a tautology/satisfiable
	\end{itemize}
\end{enumerate}
\section{Propositional Logic}
\ul{Syntax}:
Atomic formulas are called \ul{propositions} denoted by $A_0,A_1,\dots$ (or $A,B,C,\dots$)\\
A \ul{formula} in propositional logic is obtained by recursively applying the following rules:
\begin{enumerate}[(1)]
	\item An atomic formula is a formula
	\item Given a formula $F$, also $^\neg F$ is a formula ("not $F$")
	\item Given two formulas $F,G$, also $F\wedge G$ ("$F$ and $G$") and $F\vee G$ ("$F$ or $G$") are formulas
\end{enumerate}
Let $F,G$ be formulas.
\begin{enumerate}[(a)]
	\item Instead of $^\neg F\vee G$, we also write $F\Rightarrow G$ ("$F$ implies $G$")
	\item Instead of $(^\neg F\vee G)\wedge (F\vee ^\neg G)$ we also write $F\Leftrightarrow G$ ("$F$ is equivalent to $G$")
	\item Instead of $F_1\wedge\dots\wedge F_n$ we also write $\bigwedge_{i=1}^n F_i$ and instead of $F_1\vee\dots\vee F_n$ also $\bigvee_{i=1}^n F_i$
\end{enumerate}
\ul{Semantics}:
\begin{enumerate}[(a)]
	\item The set of truth values is $\{0,1\}$, where 0 is $FALSE$ and 1 is $TRUE$
	\item Let $M$ be a set of atomic formulas. A map $\alpha:M\rightarrow\{0,1\}$ is called a \ul{truth assignment}
	\item Let $\hat{M}$ be the set of all formulas in which only propositions of $M$ appear.\\
	Then we define $\hat{\alpha}:\hat{M}\rightarrow\{0,1\}$ recursively as follows:
	\begin{enumerate}[(1)]
		\item If $A\in M$, then we let $\hat{\alpha}(A)=\alpha(A)$
		\item If $\alpha(F)$ is defined, then we let $\hat{\alpha}(^\neg F)=1-\hat{\alpha}(F)$
		\item Given formulas $F,G$ for which $\hat{\alpha}(F),\hat{\alpha}(G)$ have been defined, we let $\hat{\alpha}(F\wedge G)=\begin{cases}
		1 & \text{if } \hat{\alpha}(F)=\hat{\alpha}(G)=1\\
		0 & \text{otherwise}
		\end{cases}$ and $\hat{\alpha}(F\vee G)=\begin{cases}
		1 & \text{if } \hat{\alpha}(F)=1 \text{ or } \hat{\alpha}(G)=1 \text{ or both}\\
		0 & \text{if } \hat{\alpha}(F)=\hat{\alpha}(G)=1
		\end{cases}$
	\end{enumerate}
\end{enumerate}
\begin{tabular}{ll}
$F$ "if" $G$ & $\hateq$ $G\Rightarrow F$\\
$F$ "only if" $G$ & $\hateq$ $F\Rightarrow G$\\
$F$ "if and only if" $G$ & $\hateq$ $F\Leftrightarrow G$\\
\end{tabular}\\
Let $F$ be a (propositional logic) formula, $M$ a set of propositions and $\alpha:M\rightarrow\{0,1\}$ a truth assignment.
\begin{enumerate}[(a)]
	\item The formula $F$ \ul{fits} with $\alpha$ or $\alpha$ is \ul{suitable} for $F$ if in $F$ only the propositions from $M$ appear.
	\item If $\alpha(F)=1$, then $F$ is called a \ul{model} for $\alpha$. We write $\alpha\models F$.
	\item Given a set of formulas $\mathcal{F}$, we write $\alpha\models \mathcal{F}$ if $\alpha\models F$ for every $F\in\mathcal{F}$.
	\item We say that $F$ is \ul{satisfiable} if there exists a truth assignment $\alpha$, which is suitable for $F$ and if $\alpha(F)=1$. Otherwise, we say that $F$ is \ul{unsatisfiable}.
	\item A formula $F$ is called a \ul{tautology} (or \ul{valid}) if $\alpha(F)=1$ for every suitable truth assignment $\alpha$.
\end{enumerate}
A formula $F$ is a \ul{tautology} if and only if $^\neg F$ is unsatisfiable.\\
Two formulas $F,G$ are called \ul{(semantically) equivalent}, if for all truth assignments $\alpha$, which are suitable for both $F$ and $G$, we have $\alpha(F)=\alpha(G)$. Notation: $F\equiv G$\\
\ul{The Fundamental Equivalences:}
Let $F,G,H$ be formulas.
\begin{enumerate}[(a)]
	\item $F\wedge F\equiv F$ and $F\vee F\equiv F$ (idempotency)
	\item $F\wedge G\equiv G\wedge F$ and $F\vee G\equiv G\vee F$ (commutativity)
	\item $(F\wedge G)\wedge H\equiv F\wedge(G\wedge H)$ and\\
	$(F\vee G)\vee H\equiv F\vee (G\vee H)$ (associativity)\\
	Hence we write $F_1\wedge\dots\wedge F_n$ or $F_1\vee \dots\vee F_n$.
	\item $F\wedge(F\vee G)\equiv F$ and $F\vee(F\wedge G)\equiv F$ (absorption)
	\item $F\wedge(G\vee H)\equiv (F\wedge G)\vee(F\wedge H)$ and\\
	$F\vee(G\wedge H)\equiv (F\vee G)\wedge(F\vee H)$ (distributive law)
	\item $^\neg$$^\neg F\equiv F$
	\item $^\neg(F\wedge G)\equiv ^\neg F\vee ^\neg G$ and\\
	$^\neg(F\vee G)\equiv ^\neg F\wedge ^\neg G$ (de Morgan's rules)
	\item If $F$ is a tautology, then $F\vee G\equiv F$ and $F\wedge G\equiv G$
	\item If $F$ is unsatisfiable, then $F\vee G\equiv G$ and $F\wedge G\equiv F$
\end{enumerate}
\ul{Substitution Theorem}:\\
Let $F_1,F_2$ be two equivalent formulas.\\
Let $G$ be a formula, which contains $F_1$ as a subformula.\\
Let $\tilde{G}$ be the formula obtained by replacing $F_1$ in $G$ by $F_2$.\\
Then we have $G\equiv \tilde{G}$.
\begin{enumerate}[(a)]
	\item A \ul{literal} is an atomic formula or the negation of an atomic formula ($A_i$ or $^\neg A_i$)
	\item A formula $F$ is said to be in \ul{conjunctive normal form} (CNF), if it is of the form $$F=(L_{11}\vee L_{12}\vee\dots\vee L_{1n_1})\wedge\dots\wedge(L_{k1}\vee L_{k2}\vee\dots\vee L_{kn_k})$$ where the $L_{ij}$ are literals ("$F$ is a conjunction of disjunctions of literals").
	\item We say that $F$ is in \ul{disjunctive normal form} (DNF) if $$F=(L_{11}\wedge L_{12}\wedge\dots\wedge L_{1n_1})\vee\dots\vee(L_{k1}\wedge L_{k2}\wedge\dots\wedge L_{kn_k})$$ with literals $L_{ij}$.
\end{enumerate}
\ul{Algorithm:}\\
Let $F$ be a formula. Consider the following sequence of instructions:
\begin{enumerate}[(1)]
	\item Replace all occurrences of "$\Rightarrow$" and "$\Leftrightarrow$" by their definition
	\item Replace each subformula of the form $^\neg$$^\neg G$ by $G$.
	\item Replace in $F$ every subformula of the form $^\neg (G\vee H)$ by $^\neg G\wedge ^\neg H$. If a subformula $^\neg$$^\neg K$ results, apply step (2).
	\item Replace in $F$ every subformula of the form $^\neg (G\wedge H)$ by $^\neg G\vee ^\neg H$. If a subformula $^\neg$$^\neg K$ results, apply step (2).
	\item Repeat (3) and (4) as often as possible.
	\item Replace in $F$ every subformula of the form $G\vee (H\wedge K)$ by $(G\vee H)\wedge (G\vee K)$
	\item Replace in $F$ every subformula of the form $(G\wedge H)\vee K$ by $(G\vee K)\wedge (H\vee K)$
	\item Repeat (6) and (7) as often as possible. Then return $F$ and stop.
\end{enumerate}
This is an algorithm, which returns a formula $\tilde{F}$ in CNF, such that $\tilde{F}\equiv F$.
\end{document}






























