\documentclass[a4paper]{article}

\usepackage[ngerman]{babel}
\usepackage[utf8]{inputenc}
\usepackage{amsthm}
\usepackage{amsmath}
\usepackage{amssymb}
\usepackage{tikz,tkz-euclide}
\usepackage{titlesec}
\usepackage{gensymb}
\usepackage{textcomp}
\usepackage[titles]{tocloft}
\usepackage{csquotes}
\usepackage[babel]{microtype}
\usepackage{MnSymbol}
\usepackage{stmaryrd}
\usepackage{mathtools}
\usepackage{ulem}
\usepackage[shortlabels]{enumitem}
\usepackage{scalerel}
\usepackage{stackengine}
\usepackage[
  separate-uncertainty = true,
  multi-part-units = repeat
]{siunitx}

\usetkzobj{all}
\usetikzlibrary{shapes.misc}

\MakeOuterQuote{"}

\setcounter{secnumdepth}{4}

\renewcommand\hateq{\mathrel{\stackon[1.5pt]{=}{\stretchto{%
				\scalerel*[\widthof{=}]{\wedge}{\rule{1ex}{3ex}}}{0.5ex}}}}

\newcommand*\circled[1]{
  \tikz[baseline=(C.base)]\node[draw,circle,inner sep=0.75pt](C) {#1};\!
}

\newcommand*{\obot}{\perp\mkern-20.7mu\bigcirc}

\DeclarePairedDelimiter\abs{\lvert}{\rvert}
\DeclarePairedDelimiter\norm{\lVert}{\rVert}
\makeatletter
\let\oldabs\abs
\def\abs{\@ifstar{\oldabs}{\oldabs*}}
\let\oldnorm\norm
\def\norm{\@ifstar{\oldnorm}{\oldnorm*}}
\makeatother

\renewcommand{\thesubsection}{\arabic{subsection}}
\titleformat{\section}{\normalfont\Large\bfseries}{Kapitel \arabic{section}: }{0em}{}
\titleformat{\subsection}{\normalfont\large\bfseries}{§\arabic{subsection} }{0em}{}
\titleformat{\subsubsection}{\normalfont\bfseries}{\arabic{subsection}.\arabic{subsubsection} }{0em}{}
\renewcommand{\cftsubsecpresnum}{§}
\newlength\mylength
\settowidth\mylength{\cftsubsecpresnum}
\settowidth\mylength{\cftsubsecaftersnum}
\addtolength\cftsubsecnumwidth{\mylength}
\renewcommand{\cftsecpresnum}{Kapitel }
\renewcommand{\cftsecaftersnum}{: }
\settowidth\mylength{\cftsecpresnum}
\addtolength\cftsecnumwidth{\mylength}

\newcommand{\ul}{\underline}
\renewcommand{\proof}{\ul{Beweis:}\\}
\renewcommand{\qed}{\begin{flushright}
\ul{\(q.e.d.\)}
\end{flushright}}
\let\origphi\phi
\let\phi\varphi
\let\origepsilon\epsilon
\let\epsilon\varepsilon

\title{Vorlage}
\author{Nico Mexis}
\date{\today}

\begin{document}
\maketitle
\newpage

\tableofcontents
\newpage

\section{Grundbegriffe}
\begin{tabular}{c|l}
$\Sigma$ & Alphabet: endliche, nicht-leere Menge von Zeichen/Symbolen\\
$w$ & Wort über Alphabet $\Sigma$: endliche Folge von Zeichen aus $\Sigma$ \\
$\epsilon$ & leeres Wort\\
$\Sigma^*$ & Menge aller Wörter über $\Sigma$\\
$\Sigma^+$ & $\Sigma^*\backslash\{\epsilon\}$\\
$\Sigma^n$ & Menge aller Wörter über $\Sigma$ der Länge \(n\)
\end{tabular}\\
$w_1,\ w_2$ Wörter $\Rightarrow$ $w_1\cdot w_2$ oder $w_1 w_2$ Konkatenation\\
$w^n=\underbrace{www\dots w}_n$\\
$\abs{w}$=Anzahl der Zeichen von $w$\\
$w=uvx$ $\Rightarrow$
$\begin{cases}
u & \text{Präfix}\\
v & \text{Teilwort}\\
w & \text{Suffix}
\end{cases}$\\
$L\subseteq\Sigma^*$ formale Sprache\\
$L_1L_2=\{w_1w_2\ |\ w_1\in L_1,\ w_2\in L_2\}$\\
$L^n=\underbrace{LLL\dots L}_n$\\
Seien $L,\ L_1,\ L_2\subseteq \Sigma^*$\\

TODO NACHHOLEN

\section{TODO Überschrift Folie}
Zustand: Momentaufnahme eines Systems zu einem Zeitpunkt\\
Übergang: Änderung des Zustands - spontan/aufgrund externer Eingaben\\
Ein (deterministischer) endlicher Automat (DEA) ist ein Tupel $$M=\left(Q,\Sigma,\delta,s,F\right)$$
wobei gilt:
\begin{itemize}
\item $Q$ ist eine endliche Menge von Zuständen
\item $\Sigma$ ist ein endliches Eingabealphabet
\item $\delta:Q\times \Sigma \rightarrow Q$ ist die Zustandsübergangsfunktion
\item $s\in Q$ ist der Startzustand
\item $F\subseteq Q$ ist eine Menge von akzeptierenden Zuständen oder Finalzuständen
\end{itemize}
Schreibweise:\\
Definiere $\hat{\delta}:Q\times\Sigma^*\rightarrow Q$ induktiv über die Länge des Wortes $x$:\\
$\hat{\delta}(q,\epsilon)=q$\\
$\hat{\delta}(q,xa)=\delta(\hat{\delta}(q,x),a)$\\
Jede reguläre Sprache wird von einem DEA akzeptiert.\\
$L_1\cup L_2,\ L_1L_2,\ L_1^*$ werden von einem DEA akzeptiert.\\\\
Sei $M=\left(Q,\Sigma,\delta,s,F\right)$ ein Automat mit $L(M)=A$.\\
Dann akzeptiert der Automat $M'=\left(Q,\Sigma,\delta,s,Q\backslash F\right)$ die Sprache $L(M')=A^C$.\\\\
$L(M_3)=L(M_1)\cap L(M_2)$
\end{document}






























