\documentclass[a4paper]{article}

\usepackage[ngerman]{babel}
\usepackage[utf8]{inputenc}
\usepackage{amsthm}
\usepackage{amsmath}
\usepackage{amssymb}
\usepackage{tikz,tkz-euclide}
\usepackage{titlesec}
\usepackage{gensymb}
\usepackage{textcomp}
\usepackage[titles]{tocloft}
\usepackage{csquotes}
\usepackage[babel]{microtype}
\usepackage{MnSymbol}
\usepackage{stmaryrd}
\usepackage{mathtools}
\usepackage{ulem}
\usepackage[shortlabels]{enumitem}
\usepackage{scalerel}
\usepackage{stackengine}
\usepackage[
  separate-uncertainty = true,
  multi-part-units = repeat
]{siunitx}

\usetkzobj{all}
\usetikzlibrary{shapes.misc}

\MakeOuterQuote{"}

\setcounter{secnumdepth}{4}

\renewcommand\hateq{\mathrel{\stackon[1.5pt]{=}{\stretchto{%
				\scalerel*[\widthof{=}]{\wedge}{\rule{1ex}{3ex}}}{0.5ex}}}}

\newcommand*\circled[1]{
  \tikz[baseline=(C.base)]\node[draw,circle,inner sep=0.75pt](C) {#1};\!
}

\newcommand*{\obot}{\perp\mkern-20.7mu\bigcirc}

\DeclarePairedDelimiter\abs{\lvert}{\rvert}
\DeclarePairedDelimiter\norm{\lVert}{\rVert}
\makeatletter
\let\oldabs\abs
\def\abs{\@ifstar{\oldabs}{\oldabs*}}
\let\oldnorm\norm
\def\norm{\@ifstar{\oldnorm}{\oldnorm*}}
\makeatother

\newcommand{\ul}{\underline}
\renewcommand{\proof}{\ul{Beweis:}\\}
\renewcommand{\qed}{\begin{flushright}
\ul{\(q.e.d.\)}
\end{flushright}}
\let\origphi\phi
\let\phi\varphi
\let\origepsilon\epsilon
\let\epsilon\varepsilon
\let\blank\sqcup

\title{Vorlage}
\author{Nico Mexis}
\date{\today}

\begin{document}
\maketitle
\newpage

\tableofcontents
\newpage

\section{Formale Sprachen}
\begin{tabular}{c|l}
$\Sigma$ & Alphabet: endliche, nicht-leere Menge von Zeichen/Symbolen\\
$w$ & Wort über Alphabet $\Sigma$: endliche Folge von Zeichen aus $\Sigma$ \\
$\epsilon$ & leeres Wort\\
$\Sigma^*$ & Menge aller Wörter über $\Sigma$\\
$\Sigma^+$ & $\Sigma^*\backslash\{\epsilon\}$\\
$\Sigma^n$ & Menge aller Wörter über $\Sigma$ der Länge \(n\)
\end{tabular}\\
$w_1,\ w_2$ Wörter $\Rightarrow$ $w_1\cdot w_2$ oder $w_1 w_2$ Konkatenation\\
$w^n=\underbrace{www\dots w}_n$\\
$\abs{w}$=Anzahl der Zeichen von $w$\\
$w=uvx$ $\Rightarrow$
$\begin{cases}
u & \text{Präfix}\\
v & \text{Teilwort}\\
w & \text{Suffix}
\end{cases}$\\
$L\subseteq\Sigma^*$ formale Sprache\\
$L_1L_2=\{w_1w_2\ \vert\ w_1\in L_1,\ w_2\in L_2\}$\\
$L^n=\underbrace{LLL\dots L}_n$\\
Seien $L,\ L_1,\ L_2\subseteq \Sigma^*$\\
$L^*:=\bigcup_{i\geq 0}L^i$\\
$L^+:=\bigcup_{i\geq 1}L^i$\\
$L^c:=\Sigma^*\backslash L$\\
$L_1\backslash L_2:=\{w\in\Sigma^*\ \vert\ \exists z\in L_2:wz\in L_1\}$\\
auch: $\cup,\cap,\backslash$\\
$L$ regulär, wenn eine der folgenden Punkte gilt:
\begin{itemize}
	\item Verankerung:
	\begin{itemize}
		\item $L=\{a\}$ mit $a\in\Sigma$ oder
		\item $L=\varnothing$ oder
		\item $L=\{\epsilon\}$ oder
	\end{itemize}
	\item Induktion:
	\begin{itemize}
		\item $L=L_1\cup L_2$ oder
		\item $L=L_1\cdot L_2$ oder
		\item $L=L_1^*$
	\end{itemize}
	für zwei Sprachen $L_1,L_2$
\end{itemize}
$\epsilon=\{\epsilon\},\ a=\{a\}$\\
$^*$ vor $\cdot$ vor $\cup$
\section{Deterministische endliche Automaten}
Zustand: Momentaufnahme eines Systems zu einem Zeitpunkt\\
Übergang: Änderung des Zustands - spontan/aufgrund externer Eingaben\\
Ein (deterministischer) endlicher Automat (DEA) ist ein Tupel $$M=\left(Q,\Sigma,\delta,s,F\right)$$
wobei gilt:
\begin{itemize}
\item $Q$ ist eine endliche Menge von Zuständen
\item $\Sigma$ ist ein endliches Eingabealphabet
\item $\delta:Q\times \Sigma \rightarrow Q$ ist die Zustandsübergangsfunktion
\item $s\in Q$ ist der Startzustand
\item $F\subseteq Q$ ist eine Menge von akzeptierenden Zuständen oder Finalzuständen
\end{itemize}
Schreibweise:\\
Definiere $\hat{\delta}:Q\times\Sigma^*\rightarrow Q$ induktiv über die Länge des Wortes $x$:\\
$\hat{\delta}(q,\epsilon)=q$\\
$\hat{\delta}(q,xa)=\delta(\hat{\delta}(q,x),a)$\\
Jede reguläre Sprache wird von einem DEA akzeptiert.\\
$L_1\cup L_2,\ L_1L_2,\ L_1^*$ werden von einem DEA akzeptiert.\\\\
Sei $M=\left(Q,\Sigma,\delta,s,F\right)$ ein Automat mit $L(M)=A$.\\
Dann akzeptiert der Automat $M'=\left(Q,\Sigma,\delta,s,Q\backslash F\right)$ die Sprache $L(M')=A^c$.\\\\
$L(M_3)=L(M_1)\cap L(M_2)$ wird von einem DEA mit $M_i=(Q_i,\Sigma,\delta_i,s_i,F_i)$ akzeptiert.\\
$Q_3=Q_1\times Q_2$\\
$\delta_3((p,q),a)=(\delta_1(p,a),\delta_2(q,a))$\\
$s_3=(s_1,s_2)$\\
$F_3=F_1\times F_2$\\\\
$L_1\cup L_2=(L_1^c\cap L_2^c)^c$\\
oder einfacher: $F_3=(F_1\times Q_2)\cup (Q_1\times F_2)$, Rest wie bei Durchschnitt
\section{Nichtdeterministische endliche Automaten}
$L_1\cdot L_2$ wird von einem NEA akzeptiert, bei dem man die Finalzustände von $L_1$ mit dem Startzustand von $L_2$ verschmilzt.\\
Sei $M=(Q,\Sigma,\Delta,S,F)$ ein NEA mit Überführungsfunktion $\Delta:Q\times\Sigma\rightarrow\mathcal{P}(Q)$\\
Definiere $\hat{\Delta}:\mathcal{P}(Q)\times\Sigma^*\rightarrow\mathcal{P}(Q)$:\\
$\hat{\Delta}(A,\epsilon):=A$\\
$\hat{Delta}(A,xa)=\bigcup_{q\in\hat{\Delta}(A,x)}\Delta(q,a)$\\
Die vom NEA $M$ akzeptierte Sprache ist $L(M)=\{x\in\Sigma^*\ \vert\ \hat{\Delta}(S,x)\cap F\neq\varnothing\}$\\\\
Für $x,y\in\Sigma^*$ und $A\subseteq Q$ gilt $\hat{\Delta}(A,xy)=\hat{\Delta}(\hat{\Delta}(A,x),y)$\\
Für jede indizierte Familie von Mengen $A_i\subseteq Q$ gilt $\hat{\Delta}(\bigcup_iA_i,x)=\bigcup_i\hat{\Delta}(A_i,x)$\\
Zwei endliche Automaten heißen \ul{äquivalent}, wenn sie dieselbe Sprache akzeptieren\\
\ul{Satz von Rabin, Scott}: Zu jedem NEA gibt es einen äquivalenten DEA\\
Sei $N=\{Q_N,\Sigma,\Delta_N,S_N,F_N\}$ NEA und $M=\{Q_M,\Sigma,\delta_M,s_M,F_M\}$ äquivalenter DEA, wobei gilt:\\
$Q_M=\mathcal{P}(Q_N)$\\
$\delta_M(A,a)=\hat{\Delta}(A,a)$\\
$s_M=S_N$\\
$F_M=\{A\subseteq Q_N\ \vert\ A\cap F_N\neq\varnothing\}$
\section{Automatensprachen und Regularität}
Erlaube auch Spontanübergänge in NEAs (einlesen von $\epsilon$)\\
Zu jedem NEA mit $\epsilon$-Übergänge gibt es einen NEA ohne, der dieselbe Sprache akzeptiert und nicht mehr Zustände hat.\\
VNEA wie NEA, aber Übergänge sind Regex\\\\
\ul{Pumping Lemma}: Sei $L$ eine reguläre Sprache. Dann existiert eine Zahl $n\in\mathbb{N}$, sodass für jedes Wort $w\in L$ mit $\abs{w}>n$ eine Darstellung $w=uvx$ mit $\abs{uv}\leq n$, $v\neq\epsilon$ existiert, für die auch $uv^ix\in L$ ist für alle $i\in\mathbb{N}_0$.
\section{Minimierung von Automaten}
Gegeben: DEA $M=(Q,\Sigma,\delta,s,F)$ für Sprache $A$. Minimierung in 2 Schritten:
\begin{enumerate}[(1)]
	\item Entferne überflüssige Zustände\\
	Zustand $q$ ist überflüssig, wenn es kein Wort $x\in\Sigma^*$ gibt mit $\hat\delta(s,x)=q$ (via Breitensuche beginnend bei $s$, entferne alle übrigen Zustände)
	\item Identifiziere äquivalente Zustände
\end{enumerate}
\ul{Äquivalenz}: $p\approx q\Leftrightarrow\forall x\in\Sigma^*:\hat\delta(p,x)\in F\Leftrightarrow\hat\Delta(q,x)\in F$\\
$M/\approx\ =(Q',\Sigma,\delta',s',F')$ mit
\begin{itemize}
	\item $Q'=\{\left[p\right]\ \vert\ p\in Q\}$
	\item $\delta'(\left[p\right],a)=\left[\delta(p,a)\right]$
	\item $s'=\left[s\right]$
	\item $F'=\{\left[p\right]\ \vert\ p\in F\}$
\end{itemize}
\section{Myhill-Nerode-Relationen}
Automat $M$ induziert Äquivalenzrelation $\equiv_M$ auf $\Sigma^*$ durch: $x\equiv_M y:\Leftrightarrow\hat{\delta}(s,x)=\hat{\delta}(s,y)$\\
Eine Äquivalenzrelation $\equiv$ auf $\Sigma*$ heißt \ul{Myhill-Nerode-Relation} für eine reguläre Sprache $R$, wenn sie folgende Eigenschaften erfüllt:
\begin{enumerate}[(i)]
	\item Rechts-Kongruenz: $x\equiv y$ $\Rightarrow$ $xa\equiv ya\ \forall a\in\Sigma$
	\item Verfeinerung von $R$: $x\equiv x$ $\Rightarrow$ $(x\in R\Leftrightarrow y\in R)$
	\item $\equiv$ hat einen endlichen Index
\end{enumerate}
Definiere DEA $M_\equiv=(Q,\Sigma,\delta,s,F)$ mit
\begin{itemize}
	\item $Q:=\{\left[x\right]\ \vert\ x\in\Sigma^*\}$
	\item $s:=\left[\epsilon\right]$
	\item $F:=\{\left[x\right]\ \vert\ x\in R\}$
	\item $\delta(\left[x\right],a):=\left[xa\right]$
\end{itemize}
$M\mapsto \equiv_M$ und $\equiv\mapsto M_\equiv$ sind invers:
\begin{itemize}
	\item Ist $\equiv$ eine Myhill-Nerode-Relation für $R$, so ist $\equiv_{M_\equiv}$ identisch zu $\equiv$
	\item Ist $M$ DEA für $R$ ohne überflüssige Zustände, so ist $M_{\equiv_M}$ isomorph zu $M$.
\end{itemize}
Für eine beliebige Sprache $R$ definiere Äquivalenzrelation $\equiv_R$ auf $\Sigma^*$: $$x\equiv_R y\ :\Leftrightarrow\ \forall z\in\Sigma^*(xz\in R\Leftrightarrow yz\in R)$$
Diese erfüllt die Eigenschaften (i) und (ii).\\
Sei $R\subseteq\Sigma^*$. Dann sind die folgenden Aussagen äquivalent:
\begin{enumerate}[(a)]
	\item $R$ ist regulär
	\item Es existiert eine Myhill-Nerode-Relation für $R$
	\item Die Relation $\equiv_R$ hat endlichen Index
\end{enumerate}
Jede Myhill-Nerode-Relation hat mindestens so viele Äquivalenzklassen wie $\equiv_R$. Insbesondere ist der Automat $M_{\equiv_R}$ minimal.
\section{Formale Sprachen und Grammatiken}
Eine Grammatik ist ein 4-Tupel $G=(V,\Sigma,P,S)$\\
\begin{itemize}
	\item $V$ ist endliche Menge von \textit{Variablen}
	\item $\Sigma$ ist endliche Menge von \textit{Terminalsymbolen}
	\item $P$ ist endliche Menge von \textit{Regeln} oder \textit{Produktionen}\\
	Formal: $P\subseteq (V\cup\Sigma)^+\times(V\cup\Sigma)^*$, $P$ endlich\\
	Für $(I,r)\in P$ schreibe $I\rightarrow r$
	\item $S\in V$ ist die Startvariable
\end{itemize}
Wenn ein Wort $w$ das Teilwort $I$ enthält, darf $I$ in $w$ durch $r$ ersetzt werden.\\
Schreibe $w\underset{G}{\rightarrow} z$, wenn $w$ durch Anwendung \textit{einer} Ableitungsregel in z verwandelt wird.\\
Schreibe $w\overset{*}{\underset{G}{\rightarrow}} z$, wenn $w$ durch Anwendung \textit{beliebig vieler} Ableitungsregeln in z verwandelt wird.\\
Wort $w\in (V\cup\Sigma)^*$, das noch Variablen enthält, heißt Satzform.\\
Zu Grammatik $G$ gehörige Sprache: $L(G)=\{w\in\Sigma^*\ \vert\ S\overset{*}{\underset{G}{\rightarrow}} w\}$\\
In jedem Ableitungsschritt am weitesten links stehende Variable ersetzt: "Linksableitung"\\
\ul{Die Chomsky-Hierarchie}:
\begin{itemize}
	\item[Typ 0] Jede Grammatik ist automatisch vom Typ 0.
	\item[Typ 1] Eine Grammatik ist vom Typ 1 oder \textit{kontextsensitiv}, wenn für alle Regeln $w_1\rightarrow w_2\in P$ gilt: $\abs{w_1}\leq\abs{w_2}$
	\item[Typ 2] Eine Typ 1-Grammatik ist vom Typ 2 oder \textit{kontextfrei}, wenn für alle Regeln $w_1\rightarrow w_2\in P$ gilt: $w_1\in V$ ist eine einzelne Variable
	\item[Typ 3] Eine Typ 2-Grammatik ist vom Typ 3 oder \textit{regulär}, wenn zusätzlich für alle Regeln $w_1\rightarrow w_2\in P$ gilt: $w_2\in\Sigma\cup\Sigma V$
\end{itemize}
Sonderregel für $\epsilon$: Falls $\epsilon\in L(G)$ erwünscht die Regel $S\rightarrow\epsilon$ aber dann darf $S$ auf keiner rechten Seite einer Produktion vorkommen (Ersetze $S$ immer durch $S'$ und füge $S\rightarrow\epsilon\ \vert\ \textit{andere Regeln}$ hinzu)\\
Wenn $A\rightarrow\epsilon$ in Typ 2 oder Typ3:
\begin{enumerate}[1.]
	\item Partitioniere $V$ in $V_1=\{A\in V\ \vert\ A\overset{*}{\underset{G}{\rightarrow}}\epsilon\}$ und $V_2=V\backslash V_1$
	\item Entferne alle Regeln der Form $A\rightarrow\epsilon$
	\item Für jede Regel $B\rightarrow xAy$ mit $B\in V$, $A\in V_1$, $xy\in (V\cup\Sigma)^+$ füge Regel $B\rightarrow xy$ hinzu
	\item Füge Regel für $\epsilon$ ggf. wieder hinzu (mit vorheriger Sonderregel)
\end{enumerate}
TODO linear Folie\\
Typ 3 $\subsetneq$ Typ 2 $\subsetneq$ Typ 1 $\subsetneq$ Typ 0\\
Nicht jede Sprache besitzt Grammatik\\
Die Menge der Typ 3-Sprachen ist genau die der regulären Sprachen\\
\begin{tabular}{|r|c|c|c|c|}\hline
	& Typ 0 & Typ 1 & Typ 2 & Typ 3\\\hline
	Wortproblem & & & \ & $\times$\\\hline
	Endlichkeitsproblem & & & & $\times$\\\hline
	Schnittproblem & & & & $\times$\\\hline
	Äquivalenzproblem & & & & $\times$\\\hline
\end{tabular}\\
Eine kontextfreie Grammatik $G=(V,\Sigma,P,S)$ ist in \ul{Chomsky-Normalform}, wenn alle Produktionen von der Form $A\rightarrow BC$ oder $A\rightarrow a$ sind mit $A,B,C\in V$, $a\in\Sigma$.\\
Eine kontextfreie Grammatik $G=(V,\Sigma,P,S)$ ist in \ul{Greibach-Normalform}, wenn alle Produktionen von der Form $A\rightarrow aB_1B_2\dots B_k$ sind mit $k \geq 0$, $A,B_1,\dots,B_k\in V$, $a\in \Sigma$.\\
Für jede kontextfreie Grammatik (KFG) existiert eine KFG $G'$ in Chomsky-Normalform und eine KFG $G''$ in Greibach-Normalform, sodass $L(G'')=L(G')=L(G)\backslash\{\epsilon\}$.\\
\ul{Pumping-Lemma für kontextfreie Sprachen}: Für jede kontextfreie Sprache $L$ existiert ein $K\geq 0$, sodass für jedes $z\in L$ der Länge mindestens $k$ gilt: Es existiert Zerlegung $z=uvwxy$ mit $vx\neq\epsilon$, $\abs{vwx}\leq k$ und für alle $i\geq 0$ gilt $uv^iwx^iy\in L$.\\
TODO VL\\
Ein \ul{nichtdeterministischer Kellerautomat} (NPDA) ist ein Tupel $M=(Q,\Sigma,\Gamma,\delta,s,\bot,F)$. Dabei ist
\begin{itemize}
	\item $A$ endliche Zustandsmenge
	\item $\Sigma$ endliches Alphabet
	\item $\Gamma$ endliches Stack-Alphabet
	\item $s\in Q$ Startzustand
	\item $\bot\in\Gamma$ Initialisierung des Stack
	\item $\delta:Q\times (\Sigma\cup\{\epsilon\})\times\Gamma\rightarrow 2^{Q\times\Gamma^*}$
	\item $F\subseteq Q$ Menge der akzeptierenden Endzustände, $F=\varnothing$ möglich
\end{itemize}
Eine \ul{Konfiguration eines PDA} ist ein Tripel $(q,w,\alpha)\in Q\times\Sigma^*\times\Gamma^*$, $w\in\Sigma^*$ noch nicht gelesener Teil der Eingabe, $\alpha\in\Gamma^*$ der Stack-Inhalt.\\
$M$ akzeptiert $x$ durch akzeptierenden Endzustand, wenn $(s,x,\bot)\overset{*}{\underset{M}{\rightarrow}}(q,\epsilon,\gamma)$ mit $q\in F,\gamma\in\Gamma^*$\\
$M$ akzeptiert $x$ durch leeren Stack, wenn $(s,x,\bot)\overset{*}{\underset{M}{\rightarrow}}(q,\epsilon,\epsilon)$ mit $q\in Q$ beliebig\\
Beide Varianten sind äquivalent (Umbauen des Automaten möglich)\\
Ein PDA ist deterministisch (DPDA), wenn $\abs{\delta(q,a,Z)}+\abs{\delta(q,\epsilon,Z)}\leq 1$ für alle $q\in Q,a\in\Sigma,Z\in\Gamma$.\\
PDAs sind echt mächtiger als DPDAs\\
Eine Sprache ist genau dann kontextfrei, wenn es einen Kellerautomaten gibt, der sie akzeptiert.
\section{Berechenbarkeit und Turingmaschinen}
Eine (partielle) Funktion $f:\mathbb{N}^k\rightarrow\mathbb{N}$ soll als \ul{berechenbar} angesehen werden, wenn es ein Rechenverfahren/einen Algorithmus gibt, der $f$ berechnet. D.h., Gestartet mit Eingabe $(n_1,\dots,n_k)\in\mathbb{N}^k$ soll der Algorithmus nach endlich vielen Schritten mit Ausgabe von $f(n_1,\dots,n_k)$ stoppen.\\
\ul{Partielle Funktion}: Funktion ist an manchen Stellen undefiniert, schreibe: $f(n_1,\dots,n_k)=\bot$. Bei TODO\\
Eine Turingmaschine ist gegeben durch ein 7-Tupel $M=(Q,\Sigma,\Gamma,\delta,s,\blank,F)$ mit:
\begin{itemize}
	\item $Q$ eine endliche Zustandsmenge
	\item $\Sigma$ ein endliches Eingabealphabet
	\item $s\in Q$ der Startzustand
	\item $\blank$ ein Blanksymbol mit $\blank\notin\Sigma$
	\item $\Gamma$ ein endliches Bandalphabet mit $\Sigma\cup\{\blank\}\subseteq\Gamma$
	\item $\delta:Q\times\Gamma\rightarrow Q\times\Gamma\times\{L,R,N\}$ die Überführungsfunktion (deterministisch)\\
	$\delta:Q\times\Gamma\rightarrow\mathcal{P}(Q\times\Gamma\times\{L,R,N\})$ (nicht-deterministisch)
	\item $F\subseteq Q$ die Menge der Endzustände
\end{itemize}
Eine \ul{Konfiguration} einer Turingmaschine ist ein Wort $k\in\Gamma^*Q\Gamma^*$, welches eine "Momentaufnahme" der Turingmaschine beschreibt.\\
$k=\alpha q\beta$ bedeutet:
\begin{itemize}
	\item nicht-leerer bzw. schon besuchter Bandinhalt ist $\alpha\beta$
	\item Maschine befindet sich in Zustand $q$
	\item Schreib-/Lesekopf befindet sich auf erstem Zeichen von $\beta$
\end{itemize}
Start der Turingmaschine: Eingabe $x\in\Sigma^*$ steht auf dem Band, Lese-/Schreibkopf steht auf erstem Zeichen von $x$ $\rightarrow$ Startkonfiguration: $sx$\\
Definiere Übergangsrelation $\vdash$ auf Konfigurationen
$$a_1\dots a_mqb_1\dots b_n\vdash\begin{cases}
a_1\dots a_mq'cb_2\dots b_n & \delta(q,b_1)=(q',c,N),m\geq 0,n\geq 1\\
a_1\dots a_mcq'b_2\dots b_n & \delta(q,b_1)=(q',c,R),m\geq 0,n\geq 2\\
a_1\dots a_{m-1}q'a_mcb_2\dots b_n & \delta(q,b_1)=(q',c,L),m\geq 1,n\geq 1
\end{cases}$$
Zwei Sonderfälle:\\
$a_1\dots a_mqb_1\vdash a_1\dots a_mcq'\blank$ falls $\delta(q,b_1)=(q',c,R)$\\
$qb_1\dots b_n\vdash cq'\blank b_1\dots a_n$ falls $\delta(q,b_1)=(q',c,L)$\\
Sei $M=(Q,\Sigma,\Gamma,\delta,s,\blank,F)$ eine Turingmaschine.\\
$M$ akzeptiert ein Wort $x\in\Sigma^*$, wenn sie bei Eingabe $x$ in einem Zustand aus $F$ stoppt, d.h. wenn $sx\vdash^*\alpha z\beta$ mit $z\in F$\\
Die von $M$ akzeptierte Sprache $L(M)$ ist definiert als $L(M)=\{x\in\Sigma^*\ \vert\ sx\vdash^*\alpha e\beta\text{ mit }e\in F\}$\\
Turingmaschine hält genau auf den Wörtern aus $L(M)$ in einem Endzustand\\
Für alle anderen kann sie halten (dann in einem Zustand, der kein Endzustand ist), oder auch gar nicht halten.\\
Turingmaschine, die bei jeder Eingabe hält, heißt \ul{total}.\\
Eine Sprache $L\subseteq\Sigma^*$ heißt \ul{semi-entscheidbar}, wenn es eine Turingmaschine $M$ gibt mit $L=L(M)$.\\
Eine Sprache $L\subseteq\Sigma^*$ heißt \ul{rekursiv} oder \ul{entscheidbar}, wenn es eine totale Turingmaschine M gibt mit $L=L(M)$.\\
Die von allgemeinen Turingmaschinen akzeptierbaren Sprachen sind genau die Typ-0-Sprachen (egal ob nicht-/deterministisch).\\
Die von linear beschränkten Turingmaschinen (nur nicht-deterministisch, verlässt den Bandbereich, auf dem die Eingabe steht, nicht, LBA) akzeptierbaren Sprachen sind genau die Typ-1-Sprachen.\\
TODO Berechenbarkeit\\
Eine Sprache $A$ ist genau dann \ul{entscheidbar}, wenn die charakteristische Funktion $\chi_A$ von $A$ \ul{berechenbar} ist mit $$\chi_A:\Sigma^*\rightarrow\{0,1\}, \chi_A(w)=\begin{cases}
1 & w\in A\\
0 & w\notin A
\end{cases}$$
Eine Sprache $A$ ist genau dann \ul{semi-entscheidbar}, wenn die "halbe" charakteristische Funktion $\chi_A'$ von $A$ \ul{berechenbar} ist mit $$\chi_A':\Sigma^*\rightarrow\{0,1\}, \chi_A'(w)=\begin{cases}
1 & w\in A\\
\bot & w\notin A
\end{cases}$$
\ul{Church'sche These}: Die Menge der (Turing-)berechenbaren Funktionen ist genau die Menge der im intuitiven Sinne überhaupt berechenbaren Funktionen.\\
Zu jeder Mehrband-Turingmaschine $M$ gibt es eine Einband-Turingmaschine $M'$ mit $L(M)=L(M')$ bzw. so, dass $M'$ dieselbe Funktion berechnet wie $M$.\\
TODO Maschinen hintereinander
Eine Typ-1-Grammatik ist in Kuroda-Normalform, wenn alle Regeln eine der folgenden Formen haben:\\
$A\rightarrow a$, $A\rightarrow B$, $A\rightarrow BC$, $AB\rightarrow CD$\\
Für jede Typ-1-Grammatik gibt es eine Grammatik $G'$ in Kuroda-Normalform mit $L(G')=L(G)\backslash\{\epsilon\}$\\
TODO VL Halteproblem\\
Die universelle Sprache ($L_u=\{wv\ \vert\ v\in L(T_w)\}$) ist semi-entscheidbar.\\
$L_1,L_2$ entscheidbar $\Rightarrow$ $L_1\cap L_2,L_!\cup L_2,L_1^c$ entscheidbar\\
$L_1,L_2$ semi-entscheidbar $\Rightarrow$ $L_1\cap L_2,L_1\cup L_2$ semi-entscheidbar\\
Seien $A\subseteq\Sigma^*$ und $B\subseteq \Delta^*$ zwei Sprachen. Eine \ul{Reduktion} von $A$ auf $B$ ist eine Abbildung $f:\Sigma^*\rightarrow\Delta^*$, die
\begin{enumerate}[1)]
	\item total und berechenbar ist, sodass
	\item $w\in A$ $\Leftrightarrow$ $f(w)\in B$
\end{enumerate}
Wir schreiben $A\leq_m B$, wenn es eine Reduktion von $A$ auf $B$ gibt.\\
Seien $A,B$ zwei Sprachen und $A\leq_m B$.
\begin{enumerate}[(i)]
	\item Ist $B$ semi-entscheidbar, so ist $A$ es auch. Äquivalent, ist $A$ nicht semi-entscheidbar, so ist $B$ es auch nicht.
	\item Ist $B$ entscheidbar, so ist $A$ es auch. Äquivalent, ist $A$ nicht entscheidbar, so ist $B$ es auch nicht.
\end{enumerate}
\ul{Satz von Rice}: Jede \ul{nicht-triviale} Eigenschaft der \ul{semi-entscheidbaren Mengen} ist unentscheidbar.\\
Sei $\Sigma$ ein Alphabet. Eine Eigenschaft der semi-entscheidbaren Mengen ist eine Abbildung $P:\{\text{semi-entscheidbare Teilmengen von }\Sigma^*\}\rightarrow\{T,\bot\}$\\
Nicht-trivial = ist nicht immer falsch oder immer wahr\\
Sei $\varnothing\subsetneq\mathcal{S}\subsetneq\mathcal{R}$ eine nicht-triviale Teilmenge der berechenbaren Funktionen. Dann ist $C(\mathcal{S})=\{\langle M\rangle\ \vert\ \text{die von }M\text{ berechnete Funktion liegt in } \mathcal{S}\}$ unentscheidbar.\\
Eine Eigenschaft $P:\{\text{semi-entscheidbare Mengen}\}\rightarrow\{\top,\bot\}$ heißt \ul{monoton}, wenn für alle semi-entscheidbaren Mengen $A,B$ gilt: $A\subseteq B\Rightarrow P(A)\leq P(B)$ ($\leq$ heißt "kleiner oder gleich" in der Ordnung $\bot\leq \top$)\\
Keine nicht-monotone Eigenschaft der semi-entscheidbaren Mengen ist semi-entscheidbar\\
Eine Sprache $L$ ist \ul{rekursiv aufzählbar}, wenn eine surjektive, totale berechenbare Funktion $f:\mathbb{N}\rightarrow L$ existiert.\\
Eine Sprache ist genau dann rekursiv aufzählbar, wenn sie semi-entscheidbar ist.\\
TODO Hälfte der VL\\
Sei $T$ eine Turingmaschine, die eine Funktion $t:\Sigma^*\times\Sigma^*\rightarrow\Sigma^*$ berechnet. Dann existiert eine Turingmaschine $R$, die eine Funktion $r:\Sigma^*\rightarrow\Sigma^*$ berechnet, sodass für jedes $w$ gilt: $r(w)=r(\langle R\rangle, w)$\\
\ul{Bedeutung}: $R$ arbeitet genauso wie $T$, aber seine Beschreibung $\langle R\rangle$ wird automatisch mit übergeben.\\
\ul{Beschreibung von $SELF$} (Turingmaschine, die sich selbst baut):
\begin{enumerate}
	\item Erlange (via Rekursionssatz) eigene Beschreibung von $\langle SELF\rangle$
	\item Drucke $\langle SELF\rangle$
\end{enumerate}
Rekursionssatz $\Rightarrow$ Programm mit Rekursion kann auch durch normale Turingmaschine erledigt werden\\
Für eine Turingmaschine $M$ ist die Länge der Beschreibung $\langle M\rangle$ die Anzahl der Symbole von $\langle M\rangle$.\\
Eine Turingmaschine $M$ ist minimal, wenn es keine äquivalente Turingmaschine mit kürzerer Beschreibung gibt.\\
Seo $MIN_{TM}=\{\langle M\rangle\ \vert\ M\text{ ist eine minimale TM}\}$ die Menge der minimalen Beschreibungen von Turingmaschinen.\\
$MIN_{TM}$ ist nicht semi-entscheidbar.\\

\end{document}































