\documentclass[a4paper]{article}

\usepackage[ngerman]{babel}
\usepackage[utf8]{inputenc}
\usepackage{amsthm}
\usepackage{amsmath}
\usepackage{amssymb}
\usepackage{tikz,tkz-euclide}
\usepackage{titlesec}
\usepackage{gensymb}
\usepackage{textcomp}
\usepackage[titles]{tocloft}
\usepackage{csquotes}
\usepackage[babel]{microtype}
\usepackage{MnSymbol}
\usepackage{stmaryrd}
\usepackage{mathtools}
\usepackage{ulem}
\usepackage[shortlabels]{enumitem}
\usepackage{scalerel}
\usepackage{stackengine}
\usepackage[
  separate-uncertainty = true,
  multi-part-units = repeat
]{siunitx}

\usetkzobj{all}
\usetikzlibrary{shapes.misc}

\MakeOuterQuote{"}

\setcounter{secnumdepth}{4}

\renewcommand\hateq{\mathrel{\stackon[1.5pt]{=}{\stretchto{%
				\scalerel*[\widthof{=}]{\wedge}{\rule{1ex}{3ex}}}{0.5ex}}}}

\newcommand*\circled[1]{
  \tikz[baseline=(C.base)]\node[draw,circle,inner sep=0.75pt](C) {#1};\!
}

\newcommand*{\obot}{\perp\mkern-20.7mu\bigcirc}

\DeclarePairedDelimiter\abs{\lvert}{\rvert}
\DeclarePairedDelimiter\norm{\lVert}{\rVert}
\makeatletter
\let\oldabs\abs
\def\abs{\@ifstar{\oldabs}{\oldabs*}}
\let\oldnorm\norm
\def\norm{\@ifstar{\oldnorm}{\oldnorm*}}
\makeatother

\newcommand{\ul}{\underline}
\renewcommand{\proof}{\ul{Beweis:}\\}
\renewcommand{\qed}{\begin{flushright}
\ul{\(q.e.d.\)}
\end{flushright}}
\let\origphi\phi
\let\phi\varphi
\let\origepsilon\epsilon
\let\epsilon\varepsilon

\title{Vorlage}
\author{Nico Mexis}
\date{\today}

\begin{document}
\maketitle
\newpage

\tableofcontents
\newpage

\section{Formale Sprachen}
\begin{tabular}{c|l}
$\Sigma$ & Alphabet: endliche, nicht-leere Menge von Zeichen/Symbolen\\
$w$ & Wort über Alphabet $\Sigma$: endliche Folge von Zeichen aus $\Sigma$ \\
$\epsilon$ & leeres Wort\\
$\Sigma^*$ & Menge aller Wörter über $\Sigma$\\
$\Sigma^+$ & $\Sigma^*\backslash\{\epsilon\}$\\
$\Sigma^n$ & Menge aller Wörter über $\Sigma$ der Länge \(n\)
\end{tabular}\\
$w_1,\ w_2$ Wörter $\Rightarrow$ $w_1\cdot w_2$ oder $w_1 w_2$ Konkatenation\\
$w^n=\underbrace{www\dots w}_n$\\
$\abs{w}$=Anzahl der Zeichen von $w$\\
$w=uvx$ $\Rightarrow$
$\begin{cases}
u & \text{Präfix}\\
v & \text{Teilwort}\\
w & \text{Suffix}
\end{cases}$\\
$L\subseteq\Sigma^*$ formale Sprache\\
$L_1L_2=\{w_1w_2\ \vert\ w_1\in L_1,\ w_2\in L_2\}$\\
$L^n=\underbrace{LLL\dots L}_n$\\
Seien $L,\ L_1,\ L_2\subseteq \Sigma^*$\\
$L^*:=\bigcup_{i\geq 0}L^i$\\
$L^+:=\bigcup_{i\geq 1}L^i$\\
$L^c:=\Sigma^*\backslash L$\\
$L_1\backslash L_2:=\{w\in\Sigma^*\ \vert\ \exists z\in L_2:wz\in L_1\}$\\
auch: $\cup,\cap,\backslash$\\
$L$ regulär, wenn eine der folgenden Punkte gilt:
\begin{itemize}
	\item Verankerung:
	\begin{itemize}
		\item $L=\{a\}$ mit $a\in\Sigma$ oder
		\item $L=\varnothing$ oder
		\item $L=\{\epsilon\}$ oder
	\end{itemize}
	\item Induktion:
	\begin{itemize}
		\item $L=L_1\cup L_2$ oder
		\item $L=L_1\cdot L_2$ oder
		\item $L=L_1^*$
	\end{itemize}
	für zwei Sprachen $L_1,L_2$
\end{itemize}
$\epsilon=\{\epsilon\},\ a=\{a\}$\\
$^*$ vor $\cdot$ vor $\cup$
\section{Deterministische endliche Automaten}
Zustand: Momentaufnahme eines Systems zu einem Zeitpunkt\\
Übergang: Änderung des Zustands - spontan/aufgrund externer Eingaben\\
Ein (deterministischer) endlicher Automat (DEA) ist ein Tupel $$M=\left(Q,\Sigma,\delta,s,F\right)$$
wobei gilt:
\begin{itemize}
\item $Q$ ist eine endliche Menge von Zuständen
\item $\Sigma$ ist ein endliches Eingabealphabet
\item $\delta:Q\times \Sigma \rightarrow Q$ ist die Zustandsübergangsfunktion
\item $s\in Q$ ist der Startzustand
\item $F\subseteq Q$ ist eine Menge von akzeptierenden Zuständen oder Finalzuständen
\end{itemize}
Schreibweise:\\
Definiere $\hat{\delta}:Q\times\Sigma^*\rightarrow Q$ induktiv über die Länge des Wortes $x$:\\
$\hat{\delta}(q,\epsilon)=q$\\
$\hat{\delta}(q,xa)=\delta(\hat{\delta}(q,x),a)$\\
Jede reguläre Sprache wird von einem DEA akzeptiert.\\
$L_1\cup L_2,\ L_1L_2,\ L_1^*$ werden von einem DEA akzeptiert.\\\\
Sei $M=\left(Q,\Sigma,\delta,s,F\right)$ ein Automat mit $L(M)=A$.\\
Dann akzeptiert der Automat $M'=\left(Q,\Sigma,\delta,s,Q\backslash F\right)$ die Sprache $L(M')=A^c$.\\\\
$L(M_3)=L(M_1)\cap L(M_2)$ wird von einem DEA mit $M_i=(Q_i,\Sigma,\delta_i,s_i,F_i)$ akzeptiert.\\
$Q_3=Q_1\times Q_2$\\
$\delta_3((p,q),a)=(\delta_1(p,a),\delta_2(q,a))$\\
$s_3=(s_1,s_2)$\\
$F_3=F_1\times F_2$\\\\
$L_1\cup L_2=(L_1^c\cap L_2^c)^c$\\
oder einfacher: $F_3=(F_1\times Q_2)\cup (Q_1\times F_2)$, Rest wie bei Durchschnitt
\section{Nichtdeterministische endliche Automaten}
$L_1\cdot L_2$ wird von einem NEA akzeptiert, bei dem man die Finalzustände von $L_1$ mit dem Startzustand von $L_2$ verschmilzt.\\
Sei $M=(Q,\Sigma,\Delta,S,F)$ ein NEA mit Überführungsfunktion $\Delta:Q\times\Sigma\rightarrow\mathcal{P}(Q)$\\
Definiere $\hat{\Delta}:\mathcal{P}(Q)\times\Sigma^*\rightarrow\mathcal{P}(Q)$:\\
$\hat{\Delta}(A,\epsilon):=A$\\
$\hat{Delta}(A,xa)=\bigcup_{q\in\hat{\Delta}(A,x)}\Delta(q,a)$\\
Die vom NEA $M$ akzeptierte Sprache ist $L(M)=\{x\in\Sigma^*\ \vert\ \hat{\Delta}(S,x)\cap F\neq\varnothing\}$\\\\
Für $x,y\in\Sigma^*$ und $A\subseteq Q$ gilt $\hat{\Delta}(A,xy)=\hat{\Delta}(\hat{\Delta}(A,x),y)$\\
Für jede indizierte Familie von Mengen $A_i\subseteq Q$ gilt $\hat{\Delta}(\bigcup_iA_i,x)=\bigcup_i\hat{\Delta}(A_i,x)$\\
Zwei endliche Automaten heißen \ul{äquivalent}, wenn sie dieselbe Sprache akzeptieren\\
\ul{Satz von Rabin, Scott}: Zu jedem NEA gibt es einen äquivalenten DEA\\
Sei $N=\{Q_N,\Sigma,\Delta_N,S_N,F_N\}$ NEA und $M=\{Q_M,\Sigma,\delta_M,s_M,F_M\}$ äquivalenter DEA, wobei gilt:\\
$Q_M=\mathcal{P}(Q_N)$\\
$\delta_M(A,a)=\hat{\Delta}(A,a)$\\
$s_M=S_N$\\
$F_M=\{A\subseteq Q_N\ \vert\ A\cap F_N\neq\varnothing\}$
\section{Automatensprachen und Regularität}
Erlaube auch Spontanübergänge in NEAs (einlesen von $\epsilon$)\\
Zu jedem NEA mit $\epsilon$-Übergänge gibt es einen NEA ohne, der dieselbe Sprache akzeptiert und nicht mehr Zustände hat.\\
VNEA wie NEA, aber Übergänge sind Regex\\\\
\ul{Pumping Lemma}: Sei $L$ eine reguläre Sprache. Dann existiert eine Zahl $n\in\mathbb{N}$, sodass für jedes Wort $w\in L$ mit $\abs{w}>n$ eine Darstellung $w=uvx$ mit $\abs{uv}\leq n$, $v\neq\epsilon$ existiert, für die auch $uv^ix\in L$ ist für alle $i\in\mathbb{N}_0$.
\section{Minimierung von Automaten}
Gegeben: DEA $M=(Q,\Sigma,\delta,s,F)$ für Sprache $A$. Minimierung in 2 Schritten:
\begin{enumerate}[(1)]
	\item Entferne überflüssige Zustände\\
	Zustand $q$ ist überflüssig, wenn es kein Wort $x\in\Sigma^*$ gibt mit $\hat\delta(s,x)=q$ (via Breitensuche beginnend bei $s$, entferne alle übrigen Zustände)
	\item Identifiziere äquivalente Zustände
\end{enumerate}
\ul{Äquivalenz}: $p\approx q\Leftrightarrow\forall x\in\Sigma^*:\hat\delta(p,x)\in F\Leftrightarrow\hat\Delta(q,x)\in F$\\
$M/\approx\ =(Q',\Sigma,\delta',s',F')$ mit
\begin{itemize}
	\item $Q'=\{\left[p\right]\ \vert\ p\in Q\}$
	\item $\delta'(\left[p\right],a)=\left[\delta(p,a)\right]$
	\item $s'=\left[s\right]$
	\item $F'=\{\left[p\right]\ \vert\ p\in F\}$
\end{itemize}
\section{Myhill-Nerode-Relationen}
Automat $M$ induziert Äquivalenzrelation $\equiv_M$ auf $\Sigma^*$ durch: $x\equiv_M y:\Leftrightarrow\hat{\delta}(s,x)=\hat{\delta}(s,y)$\\
Eine Äquivalenzrelation $\equiv$ auf $\Sigma*$ heißt \ul{Myhill-Nerode-Relation} für eine reguläre Sprache $R$, wenn sie folgende Eigenschaften erfüllt:
\begin{enumerate}[(i)]
	\item Rechts-Kongruenz: $x\equiv y$ $\Rightarrow$ $xa\equiv ya\ \forall a\in\Sigma$
	\item Verfeinerung von $R$: $x\equiv x$ $\Rightarrow$ $(x\in R\Leftrightarrow y\in R)$
	\item $\equiv$ hat einen endlichen Index
\end{enumerate}
Definiere DEA $M_\equiv=(Q,\Sigma,\delta,s,F)$ mit
\begin{itemize}
	\item $Q:=\{\left[x\right]\ \vert\ x\in\Sigma^*\}$
	\item $s:=\left[\epsilon\right]$
	\item $F:=\{\left[x\right]\ \vert\ x\in R\}$
	\item $\delta(\left[x\right],a):=\left[xa\right]$
\end{itemize}
$M\mapsto \equiv_M$ und $\equiv\mapsto M_\equiv$ sind invers:
\begin{itemize}
	\item Ist $\equiv$ eine Myhill-Nerode-Relation für $R$, so ist $\equiv_{M_\equiv}$ identisch zu $\equiv$
	\item Ist $M$ DEA für $R$ ohne überflüssige Zustände, so ist $M_{\equiv_M}$ isomorph zu $M$.
\end{itemize}
Für eine beliebige Sprache $R$ definiere Äquivalenzrelation $\equiv_R$ auf $\Sigma^*$: $$x\equiv_R y\ :\Leftrightarrow\ \forall z\in\Sigma^*(xz\in R\Leftrightarrow yz\in R)$$
Diese erfüllt die Eigenschaften (i) und (ii).\\
Sei $R\subseteq\Sigma^*$. Dann sind die folgenden Aussagen äquivalent:
\begin{enumerate}[(a)]
	\item $R$ ist regulär
	\item Es existiert eine Myhill-Nerode-Relation für $R$
	\item Die Relation $\equiv_R$ hat endlichen Index
\end{enumerate}
Jede Myhill-Nerode-Relation hat mindestens so viele Äquivalenzklassen wie $\equiv_R$. Insbesondere ist der Automat $M_{\equiv_R}$ minimal.
\section{Formale Sprachen und Grammatiken}
Eine Grammatik ist ein 4-Tupel $G=(V,\Sigma,P,S)$\\
\begin{itemize}
	\item $V$ ist endliche Menge von \textit{Variablen}
	\item $\Sigma$ ist endliche Menge von \textit{Terminalsymbolen}
	\item $P$ ist endliche Menge von \textit{Regeln} oder \textit{Produktionen}\\
	Formal: $P\subseteq (V\cup\Sigma)^+\times(V\cup\Sigma)^*$, $P$ endlich\\
	Für $(I,r)\in P$ schreibe $I\rightarrow r$
	\item $S\in V$ ist die Startvariable
\end{itemize}
Wenn ein Wort $w$ das Teilwort $I$ enthält, darf $I$ in $w$ durch $r$ ersetzt werden.\\
Schreibe $w\underset{G}{\rightarrow} z$, wenn $w$ durch Anwendung \textit{einer} Ableitungsregel in z verwandelt wird.\\
Schreibe $w\overset{*}{\underset{G}{\rightarrow}} z$, wenn $w$ durch Anwendung \textit{beliebig vieler} Ableitungsregeln in z verwandelt wird.\\
Wort $w\in (V\cup\Sigma)^*$, das noch Variablen enthält, heißt Satzform.\\
Zu Grammatik $G$ gehörige Sprache: $L(G)=\{w\in\Sigma^*\ \vert\ S\overset{*}{\underset{G}{\rightarrow}} w\}$\\
In jedem Ableitungsschritt am weitesten links stehende Variable ersetzt: "Linksableitung"\\
\ul{Die Chomsky-Hierarchie}:
\begin{itemize}
	\item[Typ 0] Jede Grammatik ist automatisch vom Typ 0.
	\item[Typ 1] Eine Grammatik ist vom Typ 1 oder \textit{kontextsensitiv}, wenn für alle Regeln $w_1\rightarrow w_2\in P$ gilt: $\abs{w_1}\leq\abs{w_2}$
	\item[Typ 2] Eine Typ 1-Grammatik ist vom Typ 2 oder \textit{kontextfrei}, wenn für alle Regeln $w_1\rightarrow w_2\in P$ gilt: $w_1\in V$ ist eine einzelne Variable
	\item[Typ 3] Eine Typ 2-Grammatik ist vom Typ 3 oder \textit{regulär}, wenn zusätzlich für alle Regeln $w_1\rightarrow w_2\in P$ gilt: $w_2\in\Sigma\cup\Sigma V$
\end{itemize}
Sonderregel für $\epsilon$: Falls $\epsilon\in L(G)$ erwünscht die Regel $S\rightarrow\epsilon$ aber dann darf $S$ auf keiner rechten Seite einer Produktion vorkommen (Ersetze $S$ immer durch $S'$ und füge $S\rightarrow\epsilon\ \vert\ \textit{andere Regeln}$ hinzu)\\
Wenn $A\rightarrow\epsilon$ in Typ 2 oder Typ3:
\begin{enumerate}[1.]
	\item Partitioniere $V$ in $V_1=\{A\in V\ \vert\ A\overset{*}{\underset{G}{\rightarrow}}\epsilon\}$ und $V_2=V\backslash V_1$
	\item Entferne alle Regeln der Form $A\rightarrow\epsilon$
	\item Für jede Regel $B\rightarrow xAy$ mit $B\in V$, $A\in V_1$, $xy\in (V\cup\Sigma)^+$ füge Regel $B\rightarrow xy$ hinzu
	\item Füge Regel für $\epsilon$ ggf. wieder hinzu (mit vorheriger Sonderregel)
\end{enumerate}
TODO linear Folie\\
Typ 3 $\subsetneq$ Typ 2 $\subsetneq$ Typ 1 $\subsetneq$ Typ 0\\
Nicht jede Sprache besitzt Grammatik\\
\ul{Wortproblem}: Ist ein Wort in Grammatik? Für Typ 1 entscheidbar\\
Die Menge der Typ 3-Sprachen ist genau die der regulären Sprachen
\end{document}






























